\section {Classification of issues in Solidity code}

We classify Solidity code issues as follows (based on~\cite{Henney2003}):
\begin{itemize}
	\item \textbf{Security} issues lead to exploits by a malicious user account or contract;
	\item \textbf{Functional} issues cause the violation of the intended functionality\footnote{Though without a specification we only assume what the intended functionality is.};
	\item \textbf{Operational} issues lead to run-time problems, e.g.,~bad performance;
	\item \textbf{Developmental} issues make code difficult to understand and improve.
\end{itemize}
We differentiate between functional and security issues: the former pose problems even without an adversary (though an external malicious actor can aggravate the situation), while the latter do not. Our primary sources are~\cite{Consensys2016, Solidity2017, Atzei2017, Delmolino2016, Chen2017, OpenZeppelin2017a}.
Table~\ref{AllPatternsTable} presents a summary of all issues.
We describe the patterns in more detail in Sections~\ref{sec:SecurityIssues} --~\ref{sec:DevelopmentalIssues}.

\definecolor{lightlightgray}{gray}{0.90}

\begin{table}[]
	\centering
	\caption{Code issues detected by SmartCheck. \\ Gray background -- false positives possible. }
%	\resizebox{0.5\textwidth}{!}{
	\begin{tabular}{|p{0.25\linewidth}|p{0.1\linewidth}|p{0.65\linewidth}|}
	%\begin{tabular}{|c|c|c|}
		\hline
		\textbf{Name} & \textbf{Severity} & \textbf{Description} \\
		\hline
		%% SOLIDITY_BALANCE_EQUALITY
		\usevalue SolidityBalanceEquality:name ~(\ref{SolidityBalanceEquality}) % do not remove the space before ~
		&
		\usevalue SolidityBalanceEquality:severityWord 
		&
		Adversary can manipulate contract logic by forcibly sending it ether.
		Use non-strict inequality on balances
		\\
		\hline
		%% SOLIDITY_UNCHECKED_CALL
		\rowcolor{lightlightgray}
		\usevalue SolidityUncheckedCall:name ~(\ref{SolidityUncheckedCall})
		&
		\usevalue SolidityUncheckedCall:severityWord 
		&
		The return value is not checked.
		Always check return values of functions
		\\
		\hline
		%% SOLIDITY_DOS_WITH_THROW
		\rowcolor{lightlightgray}
		\usevalue SolidityDosWithThrow:name ~(\ref{SolidityDosWithThrow})
		&
		\usevalue SolidityDosWithThrow:severityWord 
		&
		Expect external calls to deliberately \texttt{throw}
		\\
		\hline
		%% SOLIDITY_SEND
		\rowcolor{lightlightgray}
		\usevalue SoliditySend:name ~(\ref{SoliditySend})
		&
		\usevalue SoliditySend:severityWord 
		&
		The return value of \texttt{send} should be checked.
		Use \texttt{transfer}, which is equivalent to \texttt{if~(!send())~throw;}
		\\
		\hline
		%% SOLIDITY_REENTRANCY_EXTERNAL_CALL
		\rowcolor{lightlightgray}
		\usevalue SolidityReentrancyExternalCall:name ~(\ref{SolidityReentrancyExternalCall})
		&
		\usevalue SolidityReentrancyExternalCall:severityWord 
		&
		External contracts should be called after all local state updates
		\\
		\hline
		%% SOLIDITY_MALICIOUS_LIBRARIES
		\rowcolor{lightlightgray}
		\usevalue SolidityMaliciousLibraries:name ~(\ref{SolidityMaliciousLibraries})
		&
		\usevalue SolidityMaliciousLibraries:severityWord 
		&
		Using external libraries may be dangerous.
		Avoid external code dependencies, audit all code that is part of the project
		\\
		\hline
		%% SOLIDITY_TX_ORIGIN
		\usevalue SolidityTxOrigin:name ~(\ref{SolidityTxOrigin})
		&
		\usevalue SolidityTxOrigin:severityWord 
		&
		A malicious contract can act on a user's behalf.
		Use \texttt{msg.sender} for authentication
		\\
		\hline
		%% SOLIDITY_CALL_VALUE
		\rowcolor{lightlightgray}
		\usevalue SolidityCallValue:name ~(\ref{SolidityCallValue})
		&
		\usevalue SolidityCallValue:severityWord 
		&
		\texttt{a.call.value()()} forwards all gas, allowing the callee to call back.
		Use \texttt{a.transfer()}: it only provides the callee with 2300~gas (insufficient for a callback)
		\\
		\hline
		%% SOLIDITY_INTEGER_DIVISION
		\rowcolor{lightlightgray}
		\usevalue SolidityIntegerDivision:name ~(\ref{SolidityIntegerDivision})
		&
		\usevalue SolidityIntegerDivision:severityWord 
		&
		The quotient is rounded down.
		Account for it, especially for ether and token amounts
		\\
		\hline
		%% SOLIDITY_LOCKED_MONEY
		\rowcolor{lightlightgray}
		\usevalue SolidityLockedMoney:name ~(\ref{SolidityLockedMoney})
		&
		\usevalue SolidityLockedMoney:severityWord 
		&
		The contract receives ether, but there is no way to withdraw it.
		Implement a function to withdraw or reject payments
		\\
		\hline
		%% SOLIDITY_UNCHECKED_MATH
		\usevalue SolidityUncheckedMath:name ~(\ref{SolidityUncheckedMath})
		&
		\usevalue SolidityUncheckedMath:severityWord 
		&
		Without extra checks, integer over- and underflow is possible.
		Use \hbox{SafeMath}
		\\
		\hline
		%% SOLIDITY_TIMESTAMP_DEPENDENCE
		\rowcolor{lightlightgray}
		\usevalue SolidityTimestampDependence:name ~(\ref{SolidityTimestampDependence})
		&
		\usevalue SolidityTimestampDependence:severityWord 
		&
		Miners can alter timestamps.
		Make critical code independent of the environment
		\\
		\hline
		%% SOLIDITY_VAR
		\rowcolor{lightlightgray}
		\usevalue SolidityVar:name ~(\ref{SolidityVar})
		&
		\usevalue SolidityVar:severityWord 
		&
		Type inference choses the smallest integer type possible.
		Explicitly specify types
		\\
		\hline
		%% SOLIDITY_BYTES_BYTE
		\usevalue SolidityBytesByte:name ~(\ref{SolidityBytesByte})
		&
		\usevalue SolidityBytesByte:severityWord 
		&
		\texttt{byte[]} requires more gas than \texttt{bytes}
		\\
		\hline
		%% SOLIDITY_GAS_LIMIT_AND_LOOPS
		\rowcolor{lightlightgray}
		\usevalue SolidityGasLimitAndLoops:name ~(\ref{SolidityGasLimitAndLoops})
		&
		\usevalue SolidityGasLimitAndLoops:severityWord 
		&
		Expensive computation inside loops may exceed the block gas limit.
		Avoid loops with big or unknown number of steps
		\\
		\hline
		%% SOLIDITY_ERC20_API_VIOLATION
		\rowcolor{lightlightgray}
		\usevalue SolidityErc20ApiViolation:name ~(\ref{SolidityErc20ApiViolation})
		&
		\usevalue SolidityErc20ApiViolation:severityWord 
		&
		The contract \texttt{throw}s where the ERC20 standard expects a \texttt{bool}.
		Return \texttt{false} instead
		\\
		\hline
		%% SOLIDITY_PRAGMAS_VERSION
		\usevalue SolidityPragmasVersion:name ~(\ref{SolidityPragmasVersion})
		&
		\usevalue SolidityPragmasVersion:severityWord 
		&
		Contract compiles with future compiler versions.
		Specify the exact compiler version
		\\
		\hline
		%% SOLIDITY_PRIVATE_MODIFIER
		\rowcolor{lightlightgray}
		\usevalue SolidityPrivateModifier:name ~(\ref{SolidityPrivateModifier})
		&
		\usevalue SolidityPrivateModifier:severityWord 
		&
		The \texttt{private} modifier does not hide the variable's value, only prevents external contracts from editing it
		\\
		\hline
		%% SOLIDITY_REDUNDANT_FALLBACK_REJECT
		\usevalue SolidityRedundantFallbackReject:name ~(\ref{SolidityRedundantFallbackReject})
		&
		\usevalue SolidityRedundantFallbackReject:severityWord 
		&
		The payment rejection fallback is redundant.
		Remove the function to save space: payments are rejected automatically
		\\
		\hline
		%% SOLIDITY_STYLE_GUIDE_VIOLATION
		\usevalue SolidityStyleGuideViolation:name ~(\ref{SolidityStyleGuideViolation})
		&
		\usevalue SolidityStyleGuideViolation:severityWord 
		&
		Unfamiliar capitalization style causes confusion.
		Start function names with lowercase, events with uppercase
		\\
		\hline
		%% SOLIDITY_VISIBILITY
		\usevalue SolidityVisibility:name ~(\ref{SolidityVisibility})
		&
		\usevalue SolidityVisibility:severityWord 
		&
		Functions are \texttt{public} by default.
		Avoid ambiguity: explicitly declare visibility level
		\\
		\hline
	\end{tabular}
	\label{AllPatternsTable}
\end{table}


\subsection{Security issues} \label{sec:SecurityIssues}


%% SOLIDITY_BALANCE_EQUALITY
\subsubsection{\let\letcs\texapiletcs \usevalue SolidityBalanceEquality:name \let\letcs\etoolboxletcs}
\label{SolidityBalanceEquality}

Avoid checking for strict balance equality: an adversary can forcibly send ether to any account by mining or via \texttt{selfdestruct}.
\begin{lstlisting}[language=Solidity]
if (this.balance == 42 ether) { /* ... */} // bad
if (this.balance >= 42 ether) { /* ... */} // good
\end{lstlisting}

The pattern detects comparison expressions with \texttt{==} which contain \texttt{this.balance} as either left- or right-hand side.

%% SOLIDITY_UNCHECKED_CALL
\subsubsection{\let\letcs\texapiletcs \usevalue SolidityUncheckedCall:name \let\letcs\etoolboxletcs} \label{SolidityUncheckedCall}

Expect calls to external contract to fail.
When \texttt{send}ing ether, check for the return value and handle errors.
The recommended way of doing ether transfers is \texttt{transfer} (see Section~\ref{SoliditySend}).
\begin{lstlisting}[language=Solidity]
addr.send(42 ether); // bad
if (!addr.send(42 ether)) revert; // better
addr.transfer(42 ether); // good
\end{lstlisting}

The pattern detects an external function call (\texttt{call}, \texttt{delegatecall}, or \texttt{send}) which is not inside an \texttt{if}-statement.

%% SOLIDITY_DOS_WITH_THROW
\subsubsection{\let\letcs\texapiletcs \usevalue SolidityDosWithThrow:name \let\letcs\etoolboxletcs} \label{SolidityDosWithThrow}

A conditional statement (\texttt{if}, \texttt{for}, \texttt{while}) should not depend on an external call: the callee may permanently fail (\texttt{throw} or \texttt{revert}), preventing the caller from completing the execution.

In the following example, the caller expects the oracle to return an integer value (\texttt{badOracle.answer()}), but the actual oracle implementation may throw an exception in some or all cases.

\begin{lstlisting}[language=Solidity]
function dos(address oracleAddr) public {
	badOracle = Oracle(oracleAddr);
	if (badOracle.answer() < 42) { revert; }
	// ...
}
\end{lstlisting}

This rule contains multiple patterns:
\begin{itemize}
	\item an \texttt{if}-statement with an external function call in the condition and a \texttt{throw} or a \texttt{revert} in the body;
	\item a \texttt{for}- or an \texttt{if}-statement with an external function call in the condition.
\end{itemize}

%% SOLIDITY_SEND
\subsubsection{\let\letcs\texapiletcs \usevalue SoliditySend:name \let\letcs\etoolboxletcs} \label{SoliditySend}

The recommended way to perform ether payments is \texttt{addr.transfer(x)}, which automatically throws an exception if the transfer is unsuccessful, preventing the problem described in Section~\ref{SolidityUncheckedCall}.
The pattern detects the \texttt{send} keyword.

\paragraph{2020 update}
As of 2020, the best practice is that \texttt{call} is preferred to \texttt{send} and \texttt{transfer} (see comment in~\ref{SolidityCallValue}).

%% SOLIDITY_REENTRANCY_EXTERNAL_CALL
\subsubsection{\let\letcs\texapiletcs \usevalue SolidityReentrancyExternalCall:name \let\letcs\etoolboxletcs} \label{SolidityReentrancyExternalCall}

%A function call from contract \texttt{A} to contract \texttt{B} (including ether transfers) passes control from \texttt{A} to \texttt{B}, which can call \texttt{A} back before returning.
Consider the following code:

\begin{lstlisting}[language=Solidity]
pragma solidity 0.4.19;
contract Fund {
	mapping(address => uint) balances;
	function withdraw() public {
		if (msg.sender.call.value(balances[msg.sender])())
		balances[msg.sender] = 0;
	}
}
\end{lstlisting}

The contract at \texttt{msg.sender} can get multiple refunds and retrieve all \texttt{Fund}'s ether by recursively calling \texttt{withdraw} before its share is set to \texttt{0}.
Besides, it can modify the state of some third contract, which \texttt{Fund} depends on.
Use the "checks -- effects -- interactions" pattern: first check the invariants, then update the internal state, then communicate with external entities (see also Section~\ref{SoliditySend}):
\begin{lstlisting}[language=Solidity]
function withdraw() public {
	uint balance = balances[msg.sender];	
	balances[msg.sender] = 0;
	msg.sender.transfer(balance);
	// state reverted, balance restored if transfer fails
}
\end{lstlisting}

The pattern detects an external function call which is followed by an internal function call.

%% SOLIDITY_MALICIOUS_LIBRARIES
\subsubsection{\let\letcs\texapiletcs \usevalue SolidityMaliciousLibraries:name \let\letcs\etoolboxletcs} \label{SolidityMaliciousLibraries}

Third-party libraries can be malicious.
Avoid external dependencies or ensure that third-party code implements only the intended functionality.
The pattern simply detects the \texttt{library} keyword (and thus produces some false positives).

%% SOLIDITY_TX_ORIGIN
\subsubsection{\let\letcs\texapiletcs \usevalue SolidityTxOrigin:name \let\letcs\etoolboxletcs} \label{SolidityTxOrigin}

Contracts can call each other's public functions.
\texttt{tx.origin} is the first account in the call chain (always an externally owned one, i.e.,~not a contract); \texttt{msg.sender} is the immediate caller.
For instance, in a call chain \texttt{A} $\rightarrow$ \texttt{B} $\rightarrow$ \texttt{C}, from the \texttt{C}'s viewpoint, \texttt{tx.origin} is \texttt{A}, and \texttt{msg.sender} is \texttt{B}.

Use \texttt{msg.sender} instead of \texttt{tx.origin} for authentication.
Consider a wallet:

\begin{lstlisting}[language=Solidity]
pragma solidity 0.4.19;
contract TxWallet {
    address private owner;
    function TxWallet() { owner = msg.sender; }
    function transferTo(address dest, uint amount) public {
        require(tx.origin == owner);	// authentication
        dest.transfer(amount);
    }
}
\end{lstlisting}

User sends ether to the address of the \texttt{TxAttackerWallet}, which forwards the call to a vulnerable implementation of \texttt{TxWallet} and obtains all funds, acting as the user (\texttt{tx.origin}):

\begin{lstlisting}[language=Solidity]
pragma solidity 0.4.19;
interface TxWallet {
    function transferTo(address dest, uint amount);
}
contract TxAttackerWallet {
    address private owner;
    function TxAttackerWallet() { owner = msg.sender; }
    function() payable {
        TxWallet(target).transferTo(owner, msg.sender.balance);
    }
}
\end{lstlisting}

The pattern detects the environmental variable \texttt{tx.origin}.

%% SOLIDITY_CALL_VALUE
\subsubsection{\let\letcs\texapiletcs \usevalue SolidityCallValue:name \let\letcs\etoolboxletcs} \label{SolidityCallValue}

Solidity provides many ways to transfer ether (see Section~\ref{SoliditySend}).
\texttt{addr.call.value(x)()} transfers \texttt{x} ether and forwards all gas to \texttt{addr}, potentially leading to vulnerabilities like re-entrancy (see Section~\ref{SolidityReentrancyExternalCall}).
The recommended way to transfer ether is \texttt{addr.transfer(x)}, which only provides the callee with a "stipend" of 2300 gas.
The pattern detects functions whose name is \texttt{call.value} and whose argument list is empty.

\paragraph{2020 update}
In 2019, Ethereum underwent the Istanbul upgrade.
Among other modifications, gas prices for some operations were increased~\cite{EIP1884}.
As a result, the call subsidy of $2\,300$ gas is often not sufficient to handle the call.
Therefore, it is recommended to forward more than $2\,300$ gas (probably all available gas) when calling external contracts.
In the light of this pattern, \texttt{transfer} is no longer preferred over \texttt{send}.
Both these commands are considered undesirable.
The preferred way to interact with external contracts is \texttt{call}.


\subsection{Functional issues} \label{sec:FunctionalIssues}

%% SOLIDITY_INTEGER_DIVISION
\subsubsection{\let\letcs\texapiletcs \usevalue SolidityIntegerDivision:name \let\letcs\etoolboxletcs} \label{SolidityIntegerDivision}

Solidity supports neither floating-point nor decimal types.
For integer division, the quotient is rounded down.
Account for it, especially when calculating ether or token amounts.
The pattern detects division (\texttt{/}) where the numerator and the denominator are number literals.

%% SOLIDITY_LOCKED_MONEY
\subsubsection{\let\letcs\texapiletcs \usevalue SolidityLockedMoney:name \let\letcs\etoolboxletcs} \label{SolidityLockedMoney}

Contracts programmed to receive ether should implement a way to withdraw it, i.e.,~call \texttt{transfer}, \texttt{send}, or \texttt{call.value} at least once.
The pattern detects contracts that contain a \texttt{payable} function but contain neither of the withdraw-enabling functions mentioned above.

\paragraph{2020 update}
In Solidity 0.5.0, the address type has been split into \texttt{address} and \texttt{address payable}~\cite{Solidity050}.
Ether can only be sent to \texttt{address payable}.
A contract developer is thus forced to consider where money can be sent from their contract.
However, the issue captured in this pattern persists on the receiving side.
It is still possible to write a contract that receives money but cannot send it elsewhere.

%% SOLIDITY_UNCHECKED_MATH
\subsubsection{\let\letcs\texapiletcs \usevalue SolidityUncheckedMath:name \let\letcs\etoolboxletcs} \label{SolidityUncheckedMath}

Solidity is prone to integer over- and underflow\footnote{Referred to as simply overflow for brevity.}.
Overflow leads to unexpected effects and can lead to loss of funds if exploited by a malicious account.
Use the SafeMath library\footnote{See Section~\ref{SolidityMaliciousLibraries} for advice on library usage.} that checks for overflows~\cite{OpenZeppelin2017}.
The pattern detects arithmetic operations \texttt{+}, \texttt{-}, \texttt{*}, which are not inside a conditional statement.
This rule was temporarily muted for testing (Section~\ref{SectionResults}) due to a high false positive rate.


%% SOLIDITY_TIMESTAMP_DEPENDENCE
\subsubsection{\let\letcs\texapiletcs \usevalue SolidityTimestampDependence:name \let\letcs\etoolboxletcs} \label{SolidityTimestampDependence}

Miners can manipulate environmental variables and are likely to do so if they can profit from it.
Consider a lottery that distributes prizes depending on whether \texttt{now} (alias for \texttt{block.timestamp}) is odd or even:

\begin{lstlisting}[language=Solidity]
	if (now % 2 == 0) winner = pl1; else winner = pl2;
\end{lstlisting}

A miner can tweak the timestamp and gain unfair advantage.
%Use block numbers and average time between blocks to estimate the current time.
Use secure sources of randomness, such as RANDAO~\cite{RANDAO2017}.
The pattern detects the environmental variable \texttt{now}.

%% SOLIDITY_VAR
\subsubsection{\let\letcs\texapiletcs \usevalue SolidityVar:name \let\letcs\etoolboxletcs} \label{SolidityVar}

Solidity supports type inference: the type of \texttt{i} in \texttt{var i = 42;} is the smallest integer type sufficient to store the right-hand side value (\texttt{uint8}).
Consider a \texttt{for}-loop:

\begin{lstlisting}[language=Solidity]
for (var i = 0; i < array.length; i++) { /*...*/ }
\end{lstlisting}

The type of \texttt{i} is inferred to \texttt{uint8}.
If \texttt{array.length} is bigger than \texttt{256}, an overflow will occur.
Explicitly define the type when declaring integer variables:

\begin{lstlisting}[language=Solidity]
for (uint256 i = 0; i < array.length; i++) { /*...*/ }
\end{lstlisting}

The pattern detects assignments where the left-hand side is a \texttt{var} and the right-hand side is an integer (matches \texttt{\^{}[0-9]+\$}).

\paragraph{2020 update}
Implicit type inference from \texttt{var} was deprecated in Solidity 0.5.0~\cite{Solidity050}.
The pattern no longer applies.

\subsection{Operational issues} \label{sec:OperationalIssues}

%% SOLIDITY_BYTES_BYTE
\subsubsection{\usevalue SolidityBytesByte:name } \label{SolidityBytesByte}

Use \texttt{bytes} instead of \texttt{byte[]} for lower gas consumption.
The pattern detects the construction \texttt{byte[]}.

%% SOLIDITY_GAS_LIMIT_AND_LOOPS
\subsubsection{\usevalue SolidityGasLimitAndLoops:name } \label{SolidityGasLimitAndLoops}

Ethereum is a very resource-constrained environment.
Prices per computational step are orders of magnitude higher than with centralized cloud providers.
Moreover, Ethereum miners impose a limit on the total number of gas consumed in a block.
In the following example, if \texttt{array.length} is large enough, the function exceeds the block gas limit, and transactions calling it will never be confirmed:

\begin{lstlisting}[language=Solidity]
for (uint256 i = 0; i < array.length; i++) { costlyF(); }
\end{lstlisting}

This becomes a security issue, if an external actor influences \texttt{array.length}.
E.g.,~if \texttt{array} enumerates all registered addresses, and registration is open, an adversary can register many addresses, causing denial of service.
The rule includes two patterns:
\begin{itemize}
	\item a \texttt{for}-statement with a function call or an identifier inside the condition;
	\item a \texttt{while}-statement with a function call inside the condition.
\end{itemize}

\subsection{Developmental issues} \label{sec:DevelopmentalIssues}

%% SOLIDITY_ERC20_API_VIOLATION
\subsubsection{\usevalue SolidityErc20ApiViolation:name } \label{SolidityErc20ApiViolation}

ERC20 is the de-facto standard API for implementing \textit{tokens} -- transferable units of value managed by a contract.
Exchanges and other third-party services may struggle to integrate a token that does not conform to it.
Certain ERC20 functions (\texttt{approve}, \texttt{transfer}, \texttt{transferFrom}) return a \texttt{bool} indicating whether the operation succeeded.
Throwing exceptions (\texttt{revert}, \texttt{throw}, \texttt{require}, \texttt{assert}) is not recommended inside these functions.
Note that library functions may also throw exceptions (see Section~\ref{SolidityMaliciousLibraries}).

\begin{lstlisting}[language=Solidity]
function transferFrom(address _spender, uint _value)
returns (bool success) {
	require (_value < 20 wei);
	// ...
}
\end{lstlisting}

The pattern detects a contract inherited from a contract with a name including the word "token", which may throw exceptions from inside one of the functions mentioned above.

%% SOLIDITY_PRAGMAS_VERSION
\subsubsection{\usevalue SolidityPragmasVersion:name } \label{SolidityPragmasVersion}

Solidity source files indicate the versions of the compiler they can be compiled with:
\begin{lstlisting}[language=Solidity]
pragma solidity ^0.4.19;	// bad: 0.4.19 and above
pragma solidity 0.4.19;		// good: 0.4.19 only
\end{lstlisting}
It is recommended to follow the latter example, as future compiler versions may handle certain language constructions in a way the developer did not foresee.
The pattern detects the version operator~\texttt{\^{}} in the \texttt{pragma} directive.

%% SOLIDITY_PRIVATE_MODIFIER
\subsubsection{\usevalue SolidityPrivateModifier:name } \label{SolidityPrivateModifier}

Contrary to a popular misconception, the \texttt{private} modifier does not make a variable invisible.
Miners have access to all contracts' code and data.
Developers must account for the lack of privacy in Ethereum.
The pattern detects state variable declarations with a \texttt{private} modifier.

%% SOLIDITY_REDUNDANT_FALLBACK_REJECT
\subsubsection{\usevalue SolidityRedundantFallbackReject:name } \label{SolidityRedundantFallbackReject}

Contracts should reject unexpected payments (see Sections~\ref{SolidityBalanceEquality} and~\ref{SolidityLockedMoney}).
Before Solidity 0.4.0, it was done manually:
\begin{lstlisting}[language=Solidity]
function () payable { throw; }
\end{lstlisting}
Starting from Solidity 0.4.0, contracts without a fallback function automatically revert payments, making the code above redundant.
The pattern detects the described construction (only if the \texttt{pragma} directive indicates the compiler version not lower than 0.4.0).


%% SOLIDITY_STYLE_GUIDE_VIOLATION
\subsubsection{\usevalue SolidityStyleGuideViolation:name } \label{SolidityStyleGuideViolation}

In Solidity, function\footnote{Except for constructors: they must share the name with the contract and thus usually start with an uppercase letter.} and event names usually start with a lower- and uppercase letter respectively:
\begin{lstlisting}[language=Solidity]
function Foo(); // bad
event logFoo(); // bad
function foo(); // good
event LogFoo(); // good
\end{lstlisting}
Violating the style guide decreases readability and leads to confusion.
The pattern detects the described constructions.

%% SOLIDITY_VISIBILITY
\subsubsection{\usevalue SolidityVisibility:name } \label{SolidityVisibility}

The default function visibility level in Solidity is \texttt{public}.
Explicitly define function visibility to prevent confusion.
\begin{lstlisting}[language=Solidity]
function foo() { /*...*/ } // bad
function foo() public { /*...*/ } // good
function bar() private { /*...*/ } // good
\end{lstlisting}

The pattern detects function and variable definitions without a visibility modifier.

\paragraph{2020 update}
Since Solidity 0.5.0, a visibility modifier is mandatory for all functions~\cite{Solidity050}.
The pattern no longer applies.

