\frontmatter % Use roman page numbering style (i, ii, iii, iv...) for the pre-content pages

\pagestyle{plain} % Default to the plain heading style until the thesis style is called for the body content

%----------------------------------------------------------------------------------------
%	TITLE PAGE
%----------------------------------------------------------------------------------------

\begin{titlepage}
\begin{center}

\vspace*{.06\textheight}
{\scshape\LARGE \univname\par}\vspace{1.5cm} % University name
\textsc{\Large Doctoral Thesis}\\[0.5cm] % Thesis type

\HRule \\[0.4cm] % Horizontal line
{\huge \bfseries \ttitle\par}\vspace{0.4cm} % Thesis title
\HRule \\[1.5cm] % Horizontal line
 
\begin{minipage}[t]{0.4\textwidth}
\begin{flushleft} \large
\emph{Author:}\\
\href{https://s-tikhomirov.github.io/about/}{\authorname} % Author name - remove the \href bracket to remove the link
\end{flushleft}
\end{minipage}
\begin{minipage}[t]{0.4\textwidth}
\begin{flushright} \large
\emph{Supervisor:} \\
\href{https://www.cryptolux.org/index.php/Alex\_Biryukov}{\supname} % Supervisor name - remove the \href bracket to remove the link  
\end{flushright}
\end{minipage}\\[3cm]
 
\vfill

\large \textit{A thesis submitted in fulfillment of the requirements\\ for the degree of \degreename}\\[0.3cm] % University requirement text
\textit{in the}\\[0.4cm]
\groupname\\\deptname\\[2cm] % Research group name and department name
 
\vfill

{\large \today}\\[4cm] % Date
%\includegraphics{Logo} % University/department logo - uncomment to place it
 
\vfill
\end{center}
\end{titlepage}

%----------------------------------------------------------------------------------------
%	DECLARATION PAGE
%----------------------------------------------------------------------------------------
\iffalse % not seen in other theses in our group
\begin{declaration}
\addchaptertocentry{\authorshipname} % Add the declaration to the table of contents
\noindent I, \authorname, declare that this thesis titled, \enquote{\ttitle} and the work presented in it are my own. I confirm that:

\begin{itemize} 
\item This work was done wholly or mainly while in candidature for a research degree at this University.
\item Where any part of this thesis has previously been submitted for a degree or any other qualification at this University or any other institution, this has been clearly stated.
\item Where I have consulted the published work of others, this is always clearly attributed.
\item Where I have quoted from the work of others, the source is always given. With the exception of such quotations, this thesis is entirely my own work.
\item I have acknowledged all main sources of help.
\item Where the thesis is based on work done by myself jointly with others, I have made clear exactly what was done by others and what I have contributed myself.\\
\end{itemize}
 
\noindent Signed:\\
\rule[0.5em]{25em}{0.5pt} % This prints a line for the signature
 
\noindent Date:\\
\rule[0.5em]{25em}{0.5pt} % This prints a line to write the date
\end{declaration}
\fi

\cleardoublepage

%----------------------------------------------------------------------------------------
%	QUOTATION PAGE
%----------------------------------------------------------------------------------------

\vspace*{0.2\textheight}

\noindent\enquote{\itshape 
	Imagine there was a base metal as scarce as gold but with the following properties:
	
	- boring grey in colour
	
	- not a good conductor of electricity
	
	- not particularly strong, but not ductile or easily malleable either
	
	- not useful for any practical or ornamental purpose 
	
	and one special, magical property:
	
	- can be transported over a communications channel.}\bigbreak

\hfill Satoshi Nakamoto
% https://bitcointalk.org/index.php?topic=583.msg11405#msg11405

%----------------------------------------------------------------------------------------
%	ABSTRACT PAGE
%----------------------------------------------------------------------------------------

\begin{abstract}
\addchaptertocentry{\abstractname} % Add the abstract to the table of contents

Bitcoin is the first digital currency without a trusted third party.
This revolutionary protocol inspired multiple alternative projects that aim to address its limitations such as scalability and privacy.
This new area of research at the intersection of computer science and economics is often characterized by the term \textit{blockchain}.

This thesis explores the security and privacy of blockchain systems.

A cryptocurrency is based on a peer-to-peer network.
Performance, resilience, and privacy of the P2P layer are important for the protocol as a whole.
In Part~\ref{Part1Privacy}, we study the P2P networks of Bitcoin and selected privacy-focused cryptocurrencies.
We introduce a new attack on privacy that allows an attacker to link transactions issued by the same node.
We test the efficiency of the attack in real networks, successfully linking our own transactions.
We provide a separate study of privacy characteristics of mobile cryptocurrency wallets.
We discover that very few wallets follow the best practices regarding their users' privacy.

The architecture of Bitcoin and similar cryptocurrencies emphasizes security but severely limits the transaction throughput.
Off-chain protocols address this issue.
Part~\ref{Part2Lightning} is dedicated to the Lightning Network (LN) -- a prominent Bitcoin-based off-chain protocol.
Lightning performs transactions off-chain, but allows for on-chain dispute resolution.
This ensures low latency while inheriting most of Bitcoin's security guarantees.
We introduce a probing attack that allows to quickly discover user balances in the LN.
We analyze the likelihood of various privacy attacks on the LN depending on a number of parameters.
We describe a limitation on the number of concurrent LN payments and quantify its effects on the transaction throughput.

Bitcoin allows only a limited means to define how coins can be spent.
Ethereum is a blockchain network with a focus on programmability.
It allows to write programs in a Turing-complete language and permanently store them on-chain.
Such programs are called smart contracts and are usually written in a high-level language Solidity.
Part~\ref{Part3Ethereum} explores the security and privacy of smart contracts in Ethereum.
We propose Findel -- a Solidity-based declarative domain-specific language for financial contracts.
We classify the vulnerabilities in real-world Ethereum contracts.
We present SmartCheck -- a static analysis tool for bug detection in Solidity.
We describe an Ethereum-based cryptographic protocol for a more privacy-preserving regulation compliance.
	
\end{abstract}

%----------------------------------------------------------------------------------------
%	ACKNOWLEDGEMENTS
%----------------------------------------------------------------------------------------

\begin{acknowledgements}
\addchaptertocentry{\acknowledgementname} % Add the acknowledgements to the table of contents
This work would not be possible without the help and support of many people.
This is an incomplete list of those to whom I would like to express my gratitude.

First of all, I am deeply grateful my advisor, Prof.~Alex~Biryukov, for the opportunity to pursue my research interests and the invaluable guidance throughout this journey.
I appreciate Prof.~Volker M{\"u}ller, A-Prof.~Andrew Miller, Prof.~Matteo Maffei, Dr.~Patrick McCorry, and A-Prof.~Arthur Gervais agreeing to serve as jury members.
I thank my co-authors: Dmitry Khovratovich, Ekaterina Voskresenskaya, Ivan Ivanitskiy, Ramil Takhaviev, Evgeny Marchenko, Yaroslav Alexandrov, Pedro Moreno-Sanchez, Matteo Maffei, Ren{\'e} Pickhardt, and Mariusz Nowostawski.
I have learned a lot during our fruitful collaborations.
A special thanks goes to Pedro Moreno-Sanchez and Matteo Maffei for inviting me to spend three months on a research visit at TU~Wien in the beautiful city of Vienna.

I thank my colleagues for the insightful and fun conversations: Aleksei Udovenko, Brian Shaft, Christof Beierle, Dag Arne Osvik, Daniel Dinu, Daniel Feher, Dmitry Khovratovich, Giuseppe Vitto, Johann Gro{\ss}sch{\"a}dl, Luan Cardoso dos Santos, L{\'e}o Perrin, Qingju Wang, Ritam Bhaumik, Shange Fu, Vesselin Velichkov, and Yann Le Corre.

The country of Luxembourg and the University of Luxembourg generously provided me with excellent research conditions.
I specifically thank Fabienne Schmitz, Ida Ienna, and Catherine Violet for their assistance in administrative matters.

A huge thanks goes to my co-hosts for the Basic Block Radio podcast Ivan Ivanitskiy, Sergei Pavlin, and Alexander Seleznev, as well as to all our guests and listeners, for helping spread the word about decentralized technologies for the Russian-speaking audience.

My deepest gratitude goes to my family for their endless love and support.
I want to specifically thank my parents for encouraging me to pursue education abroad.

Finally, I thank Satoshi Nakamoto.
Your ingenious invention continues to amaze and inspire me.
I admire your wisdom and humility.
It is an honor to be a part of the movement you started.

\end{acknowledgements}

%----------------------------------------------------------------------------------------
%	LIST OF CONTENTS/FIGURES/TABLES PAGES
%----------------------------------------------------------------------------------------

\tableofcontents % Prints the main table of contents

\listoffigures % Prints the list of figures

\listoftables % Prints the list of tables

%----------------------------------------------------------------------------------------
%	ABBREVIATIONS
%----------------------------------------------------------------------------------------

\begin{abbreviations}{ll} % Include a list of abbreviations (a table of two columns)

\textbf{PoW} & Proof of work \\
\textbf{PoS} & Proof of stake \\
\textbf{LN} & Lightning Network \\
\textbf{CPU} & Central processing unit \\
\textbf{GPU} & Graphics processing unit \\
\textbf{FPGA} & Field-programmable gate array \\
\textbf{ASIC} & Application-specific integrated circuit \\
\textbf{P2P} & Peer-to-peer \\

\end{abbreviations}

%----------------------------------------------------------------------------------------
%	PHYSICAL CONSTANTS/OTHER DEFINITIONS
%----------------------------------------------------------------------------------------

%\begin{constants}{lr@{${}={}$}l} % The list of physical constants is a three column table

% The \SI{}{} command is provided by the siunitx package, see its documentation for instructions on how to use it

%Speed of Light & $c_{0}$ & \SI{2.99792458e8}{\meter\per\second} (exact)\\
%Constant Name & $Symbol$ & $Constant Value$ with units\\

%\end{constants}

%----------------------------------------------------------------------------------------
%	SYMBOLS
%----------------------------------------------------------------------------------------

%\begin{symbols}{lll} % Include a list of Symbols (a three column table)

%$a$ & distance & \si{\meter} \\
%$P$ & power & \si{\watt} (\si{\joule\per\second}) \\
%Symbol & Name & Unit \\

%\addlinespace % Gap to separate the Roman symbols from the Greek

%$\omega$ & angular frequency & \si{\radian} \\

%\end{symbols}

%----------------------------------------------------------------------------------------
%	DEDICATION
%----------------------------------------------------------------------------------------

%\dedicatory{For/Dedicated to/To my\ldots} 