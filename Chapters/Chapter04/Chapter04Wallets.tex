\chapter{Privacy of cryptocurrency wallets}

\label{Chapter04Wallets}


%Mobile devices have surpassed desktops by Internet traffic.
Smartphones have become the primary computing device for many millions of people and play an increasingly important role in the cryptocurrency ecosystem.
In this Chapter, we study the privacy of mobile cryptocurrency wallets for Android -- the most prevalent mobile operating system.
\footnote{This Chapter is based on~\cite{Biryukov2019}.}
We systematize privacy-related characteristics of popular wallets.
We study $23$~selected wallets in more detail.
We compare their privacy-related properties and analyze their source code, both manually and using static analysis tools.
Our results indicate that most mobile wallets do not follow important privacy guidelines.
Privacy conscious users are rather limited in their choice of a mobile wallet for Bitcoin, and even more so -- for privacy-focused cryptocurrencies.


\section{Minimal privacy criteria} \label{section:Ch04Initialprivacycriteria}

We argue that the following privacy criteria are minimally necessary for a mobile wallet:
\begin{enumerate}
	\item No registration required. Otherwise, the user's transactions can be linked together and deanonymized by the wallet provider. 
	\item Open-sourced code. A malicious closed-source application can track users or even steal their funds. Open-sourced code decreases the risk of backdoors or other unintended functionality. 
	\item Private keys generated and stored locally. A wallet that stores the keys on a server requires full trust.
	\item P2P networking. The wallet should request blockchain data and broadcast transactions to the P2P network directly. Otherwise, a remote server can deanonymize and censor transactions.
\end{enumerate}
We check the first criterion by attempting to generate a receiving address in the wallet without registering or providing any personal information.
If we succeed, we proceed to check the next criteria.
We consider the second criterion met if there is a publicly available code repository with a fully fledged Android application linked from the wallet's official website.\footnote{Strictly speaking, \textit{reproducible builds} are required to assure that the file in the Play~Store is not modified compared to the repository. We do not check this in this work.}
We check the third and the fourth criteria based on the official documentation and the source code, if available.

We compile a list of wallets recommended on the official websites of the considered cryptocurrencies: Bitcoin, Dash, Monero, and Zcash.
Additionally, we consider the most popular wallets (based on the publicly available approximate number of Play~Store downloads).
The following wallets satisfy these criteria (Bitcoin unless specified): Bitcoin wallet, Bither, BRD, Dash wallet (Dash), Electrum\footnote{Electrum is originally a desktop application; the official GitHub repository gives instructions on how to generate an APK file using Kivy GUI. Electrum wallet relies on an independent network of nodes (Electrum servers) to receive blockchain data and broadcast transactions. Though Electrum servers are not genuine cryptocurrency P2P nodes, we consider Electrum satisfy the P2P criteria, as a user can technically choose which Electrum servers to connect to, including their own.}, Monerujo (Monero), Simple Bitcoin (marked in bold in Table~\ref{tab:minimal-criteria}).
No multi-currency wallets and no Zcash wallets satisfy these criteria.

\begin{table*}
	\normalsize
	\caption{Minimal privacy criteria for selected wallets}
	\centering
	%\resizebox{\textwidth}{!}{% use resizebox with textwidth
	\begin{tabular}{ | l | c c c c | c | c | c | c | }
		\hline
		Wallet & \multicolumn{4}{|c|}{\rot{Coin support}} & \rot{No registration} & \rot{Open-source} & \rot{Local private keys} & \rot{P2P networking} \\ 
		\hline
		& \rot{Bitcoin} & \rot{Dash} & \rot{Monero } & \rot{Zcash} &  &  &  &  \\ 
		\hline
		Abra & + & + & + & + & - & - & + & ? \\
		% requires name, email, phone number
		% DOES store users txs linked: provides CSV tx history download
		% https://support.abra.com/hc/en-us/articles/360002528832-How-can-I-obtain-my-2017-transaction-history-
		\hline
		Airbitz & + & - & - & - & - & + & + & ? \\
		\hline
		ArcBit & + & - & - & - & + & + & + & - \\
		% Many Google play reviews: txs disabled (whaat?); website down
		% or Reddit: https://www.reddit.com/r/Bitcoin/comments/8wrydq/arcbit_help/
		\hline
		Bitcoin.com & + & - & - & - & + & + & + & - \\
		\hline
		\textbf{Bitcoin wallet} & + & - & - & - & + & + & + & + \\
		\hline
		\textbf{Bither} & + & - & - & - & + & - & + & + \\
		\hline
		BTC.com & + & - & - & - & - & + & + & - \\
		\hline
		\textbf{BRD} & + & - & - & - & + & + & + & + \\
		% a.k.a. Breadwallet
		\hline
		Coin.space & + & - & - & - & + & + & + & - \\
		% their website: "Server-free environment fully localizes each installed application" (what does it even mean?)
		% bitcoin.org: "This wallet relies on a centralized service by default. This means a third party must be trusted to not hide or simulate payments."
		\hline
		Coinomi & + & + & - & + & + & + & + & - \\
		% the repository declared inactive, code last updated in April~2016
		\hline
		Copay & + & - & - & - & + & + & + & - \\
		\hline
		\textbf{Dash wallet} & - & + & - & - & + & + & + & + \\
		% fork of Schildbach (not marked as such)
		% connects to peers but why on port 70210?
		% * The app now builds reproducible. (Google play)
		\hline
		Edge & + & + & + & - & - & - & + & - \\
		% registration required (choose username and password) but email is not
		% can still presumably link transactions ("username already exists" error)
		% even have a whitepaper https://edge.app/wp-content/uploads/2018/02/2017-10-Edge-White-Paper.pdf
		% use centralized Electrum servers for broadcast
		\hline
		\textbf{Electrum} & + & - & - & - & + & + & + & + \\
		% https://github.com/spesmilo/electrum/tree/master/electrum/gui/kivy
		\hline
		Ethos & + & + & + & + & - & - & ? & ? \\
		\hline
		GreenBits & + & - & - & - & + & + & + & - \\
		% GreenBits is a native Android Bitcoin wallet for GreenAddress' wallet service. SPV Validation client-side
		\hline
		%Guarda~\cite{Guarda} & + & - & - & + & + & - & + & - \\
		% replaces by multi-wallet released 2018-12-03 
		% https://play.google.com/store/apps/details?id=com.crypto.multiwallet
		%\hline
		Jaxx & + & + & - & + & + & - & + & - \\
		% https://support.decentral.ca/hc/en-us/articles/217877528-Is-Jaxx-Open-Source-
		% "The Android .apk's code can be unpacked and scrutinized in full as well."
		\hline
		Mobi.me & + & + & + & + & - & - & ? & ? \\
		% "You can sign up for Mobi through the app; all you need is a mobile phone number."
		% requires way too many permissions (contacts, locations, ...), doesn't launch without permission to calls (!)
		\hline
		\textbf{Monerujo} & - & - & + & - & + & + & + & + \\
		% "Account added" in net log, uses port 4331 (wspipe) - ?
		\hline
		Mycelium & + & - & - & - & + & + & + & - \\
		% "Ultra fast connection to the Bitcoin network through _our_ super nodes" - can use own node?
		% Connect to own node: No (only on testnet, feature in development): https://github.com/mycelium-com/wallet-android/issues/408
		% "For enhanced privacy and availability you can connect to our super nodes via a tor-hidden service ( .onion address)"
		\hline
		Samourai & + & - & - & - & + & + & + & +* \\
		\hline
		\textbf{Simple Bitcoin} & + & - & - & - & + & + & + & + \\
		% "you can explicitly instruct Simple Bitcoin Wallet to use [your node] by providing your full node's IP address and port"
		\hline
		Zelcore & + & - & - & + & - & - & + & - \\
		\hline
	\end{tabular}
	%}
	\label{tab:minimal-criteria}
\end{table*}


\section{Analysis of selected wallets} \label{section:Ch04Analysis}

The following wallets have passed the minimal privacy criteria: Bitcoin Wallet, Bither, BRD, Dash~Wallet, Electrum, Monerujo, and Simple~Bitcoin.
% Samourai, despite English spelling being "Samurai"
We add Samourai wallet to our list due to its heavy emphasis on privacy.
\footnote{Strictly speaking, Samourai did not pass our initial test: it connects to a remote node via RPC, not P2P, and requires full control over it~\cite{SamouraiRPC}. We mark it with “+*” in Table~\ref{tab:all-wallets}.}
We not study these wallets in more detail.


\subsection{Manual inspection}

\paragraph{Independent installation}
The most prevalent way of installing Android applications is the Google Play store, which requires a Google account.
A privacy-conscious user may want to avoid linking their cryptocurrency activity with their Google profile.
Alternative ways to install Android applications are F-Droid (an independent application store for free and open source applications~\cite{FDroid}) and manual installation from an APK file.
Out of the seven~wallets we considered, three are available in F-Droid, and five~can be installed from an APK file (see Table~\ref{tab:installation}).

\begin{table*}
	\normalsize
	\caption{Alternative installation methods of selected wallets}
	\centering
	\begin{tabular}{| l | l l l l l l l | l l l l l l l |}
		\hline
		& \rot{Bitcoin Wallet} & \rot{Bither} & \rot{BRD} & \rot{Dash wallet} & \rot{Electrum} & \rot{Monerujo} & \rot{Simple Bitcoin } & \rot{Bitcoin.com} & \rot{Mycelium} & \rot{Coinomi} & \rot{Jaxx} & \rot{Copay} & \rot{Airbitz} & \rot{Samourai} \\
		%\hline
		%Play Store, '000 & 1000 & 10 & 100 & 100 & 100 & 10 & 100 & 1000 & 500 & 500 & 500 & 500 & 100 & 10 \\
		\hline
		F-Droid & + & - & - & - & - & + & + & - & - & - & - & - & - & - \\
		APK & + & + & - & - & + & + & + & + & + & + & - & - & - & - \\
		\hline
	\end{tabular}
	%}
	\label{tab:installation}
\end{table*}

\paragraph{Permissions}
Android \textit{permissions} restrict access to certain user information and device functionality.
Each application declares the necessary permissions in its \textit{manifest file}.
The user grants permissions at installation time or at runtime\footnote{Starting from Android~6.0.}.
Permissions that give access to critical functionality or personal information are refereed to as \textit{dangerous}~\cite{Android}.

Airbitz requires the highest number of permissions ($15$).
Two applications require access to both coarse and fine location (Airbitz, BRD), and sending SMS (BRD, Samourai).
Electrum requests the lowest number of permissions -- four, three of which are dangerous.
All wallets require at least one dangerous permission -- camera access.
This is usually required to scan cryptocurrency addresses presented as QR codes.
The summary of the requested permissions are presented in Table~\ref{tab:permissions} (dangerous permissions in bold).

\begin{table*}
	\normalsize
	\caption{Permissions of selected wallets}
	\centering
	%\resizebox{\textwidth}{!}{% use resizebox with textwidth
	\begin{tabular}{ | l | l l l l l l l | l l l l l l l |}
		\hline
		& \rot{Bitcoin Wallet} & \rot{Bither} & \rot{BRD} & \rot{Dash wallet} & \rot{Electrum} & \rot{Monerujo} & \rot{Simple Bitcoin } & \rot{Bitcoin.com} & \rot{Mycelium} & \rot{Coinomi} & \rot{Jaxx} & \rot{Copay} & \rot{Airbitz} & \rot{Samourai} \\
		%\hline
		% GreenBits: 50
		\hline
		\textbf{read storage} & + & + & + & + & + & + & + & + & + & + & + & + & + & + \\
		\textbf{modify storage} & - & + & + & + & + & + & + & + & + & + & + & + & + & + \\
		\textbf{take pictures} & + & + & + & + & + & + & + & + & + & + & + & + & + & + \\
		\textbf{coarse location} & - & - & + & - & - & - & - & - & + & - & - & - & + & - \\
		\textbf{fine location} & - & - & + & - & - & - & - & - & - & - & - & - & + & - \\
		\textbf{send SMS} & - & - & + & - & - & - & - & - & - & - & - & - & - & + \\
		view connections & + & + & + & + & - & - & - & + & + & + & + & + & + & + \\
		Bluetooth & + & + & - & + & - & - & - & - & - & + & - & - & + & - \\
		full network access & + & + & + & + & + & + & + & + & + & + & + & + & + & + \\
		control NFC & + & - & - & + & - & + & - & - & + & + & - & - & + & - \\
		run at startup & + & + & - & + & - & - & - & - & - & + & - & - & + & + \\
		control vibration & + & + & - & + & - & - & + & - & + & + & - & + & - & + \\
		prevent sleeping & + & + & + & + & - & + & - & + & + & + & - & + & + & - \\
		upd component usg & - & - & + & - & - & - & - & - & - & - & - & - & - & - \\
		receive data & - & - & + & - & - & - & - & + & + & + & - & + & - & - \\
		background work & - & - & + & - & - & - & - & - & - & - & - & - & - & - \\
		display settings & - & - & + & - & - & - & - & - & - & - & - & - & - & - \\
		disable screen lock & - & - & + & - & - & - & - & - & - & - & - & - & - & - \\
		retrieve running apps & - & + & - & - & - & - & - & - & - & - & - & - & - & - \\
		record audio & - & + & - & - & - & - & - & - & - & - & - & - & - & - \\
		view Wi-Fi & - & + & - & - & - & - & - & - & - & - & - & - & - & - \\
		send sticky broadcast & - & + & - & - & - & - & - & - & - & - & - & - & - & - \\
		read phone status & - & - & - & - & - & - & - & + & + & - & - & + & - & + \\
		read service config & - & - & - & - & - & - & - & + & - & - & - & - & - & - \\
		license check & - & - & - & - & - & - & - & - & + & - & - & - & - & - \\
		find accounts & - & - & - & - & - & - & - & - & - & - & - & - & + & - \\
		read contact card & - & - & - & - & - & - & - & - & - & - & - & - & + & - \\
		Bluetooth settings & - & - & - & - & - & - & - & - & - & - & - & - & + & - \\
		use accounts & - & - & - & - & - & - & - & - & - & - & - & - & + & - \\
		receive SMS & - & - & - & - & - & - & - & - & - & - & - & - & - & + \\
		reroute out calls & - & - & - & - & - & - & - & - & - & - & - & - & - & + \\
		\hline
		Total permissions & $9$ & $13$ & $14$ & $10$ & $4$ & $6$ & $5$ & $9$ & $12$ & $11$ & $5$ & $9$ & $15$ & $11$ \\
		\hline
	\end{tabular}
	%}
	\label{tab:permissions}
\end{table*}


\paragraph{Privacy policies}
According to Google~Play~store rules, all Android applications that handle "personal or sensitive user data" must have a privacy policy.
We compare the privacy policies of the selected wallets.

Privacy policies of some wallets (Bitcoin wallet~\cite{BitcoinWalletPrivacyPolicy}, Dash wallet~\cite{DashWalletPrivacyPolicy}, Bither~\cite{BitherWalletPrivacyPolicy}, Monerujo~\cite{MonerujoPrivacyPolicy}) are quite concise and only justify the use of some of the required permissions.
Privacy policies of BRD~\cite{BRDPrivacyPolicy} and Electrum~\cite{ElectrumPrivacyPolicy} are much more elaborate.
Bither privacy policy provides the rationale behind a permission to record audio: it allows for "collecting ambience entropy for XRandom".
BRD uses cookies and trackers, as well as third-party providers for analytics: Google Analytics and Firebase.
These tools allow wallet developers to collect and analyze crash reports, and collect application usage patterns.
This may be used to link users' activity with unique identifiers of their devices.
%GreenBits is the only wallet which reserves the right to collect users' Bitcoin addresses.
Simple Bitcoin does not have a privacy policy.
The link from Google Play refers to the wallet's official website~\cite{SimpleBitcoin}, which does not specify whether the application collects, stores or transmits the users' personal data.
Monerujo privacy policy notes that the application uses the exchange rate information provided by the public API of \url{kraken.com}, and an exchange service \url{xmr.to}, which are subjects to their own privacy policies (marked with (+) in Table~\ref{tab:privacy-policies}).
Mycelium transmits a report to the developers' server in case of a crash.
The developers claim they "took care that it does not contain unnecessary privacy relevant information".
%It is not clear which information is being transmitted.
The Samourai wallet collects users' IP addresses "with replaced last byte".
This can hardly be considered anonymization, as one may still infer the approximate location from the first three bytes of the IP address (marked with "+*" in Table~\ref{tab:privacy-policies}).
Airbitz broadcasts transactions through Electrum servers; a user may choose a server.
In many cases the link to the privacy policy from the wallet's Play~Store page leads to the privacy policy of the corresponding \textit{website}, not the \textit{wallet}.
Table~\ref{tab:privacy-policies} summarizes the finding regarding the privacy policies.

\begin{table*}
	\normalsize
	\caption{Privacy policies of selected wallets: information that the developers may obtain}
	\centering
	\begin{tabular}{| l | l l l l l l l | l l l l l l l |}
		\hline
		& \rot{Bitcoin Wallet} & \rot{Bither} & \rot{BRD} & \rot{Dash wallet} & \rot{Electrum} & \rot{Monerujo} & \rot{Simple Bitcoin } & \rot{Bitcoin.com} & \rot{Mycelium} & \rot{Coinomi} & \rot{Jaxx} & \rot{Copay} & \rot{Airbitz} & \rot{Samourai} \\
		\hline
		IP address & - & - & + & - & + & (+) & ? & - & ? & + & - & + & + & +* \\
		browser version & - & - & + & - & - & (+) & ? & - & ? & + & - & ? & + & + \\
		pages visited & - & - & + & - & - & (+) & ? & - & ? & + & - & ? & + & + \\
		time of visit & - & - & + & - & + & (+) & ? & - & ? & + & - & ? & + & + \\
		%time spent on pages & - & - & + & - & - & (+) & ? & - & ? & + & - & ? & + & + \\
		unique device ID & - & - & + & - & - & (+) & ? & - & ? & - & - & ? & ? & - \\
		other diagnostics & - & - & + & - & + & (+) & ? & - & ? & + & + & ? & + & + \\
		type of device & - & - & + & - & - & (+) & ? & - & ? & - & - & ? & + & + \\
		OS type & - & - & + & - & + & (+) & ? & - & ? & + & - & ? & + & + \\
		location & - & - & + & - & - & (+) & ? & - & ? & + & - & ? & - & - \\
		device name & - & - & - & - & + & - & ? & - & ? & - &-  & ? & ? & - \\
		app configuration & - & - & - & - & + & - & ? & - & ? & - & - & ? & - & - \\
		pages visited before & - & - & - & - & - & (+) & ? & - & ? & + & - & ? & - & - \\
		browser plug-ins & - & - & - & - & - & (+) & ? & - & ? & - & - & ? & - & - \\
		time zone & - & - & - & - & - & (+) & ? & - & ? & - & - & ? & - & - \\
		"clickstream" & - & - & - & - & - & (+) & ? & - & ? & - & - & ? & - & - \\
		cookies & - & - & + & - & - & - & ? & - & ? & - & + & ? & + & + \\
		analytics & - & - & + & - & + & - & ? & - & - & - & - & ? & + & + \\
		\hline
	\end{tabular}
	%}
	\label{tab:privacy-policies}
\end{table*}


\paragraph{Networking}
%We inspected the source code of the open sourced P2P wallets (see Table~\ref{tab:source-code}).
All wallets with P2P transaction broadcasting except Monerujo use hard-coded DNS seeds to bootstrap.
%BRD adds its own DNS seed.
%\footnote{\texttt{seed.breadwallet.com.}; does not resolve at the time of writing.}
Simple Bitcoin adds one random node from a hard-coded list to a list of peers obtained via bootstrapping.
\footnote{\url{5.9.104.252}, \url{213.133.103.56}, \url{213.133.99.89}; all unreachable as of 2018.}
Electrum connects to two random servers from a hard-coded list of $52$~servers.
It requests the transaction history from a single server and checks it against block headers sent by other servers.
Monerujo lets the user either choose from three hard-coded URLs which resolve to a list of publicly available nodes~\footnote{\url{node.moneroworld.com:18089}, \url{node.xmrbackb.one}, \url{node.xmr.be}} or provide the credentials for connecting to a custom node.

We tried to connect Bitcoin wallets with P2P broadcast to our own full node.
Bitcoin wallet and Simple~Bitcoin did connect,\footnote{Version strings: \texttt{/bitcoinj:0.14.7/Bitcoin Wallet:6.29/}, \texttt{/bitcoinj:0.14.4/Bitcoin:1.075/}.} BRD did not.
Bither did not provide this option.

Wallets with P2P networking have a \textit{network monitor} which displays the IP addresses of connected nodes.
BRD shows only the IP of the "primary node" (without specifying what it means).
Electrum shows the address of the server used to query the transaction history (other servers are used to check it).
Simple~Bitcoin only shows the number of connected peers.
Different wallets try to establish different number of connections.

The summary of the networking characteristics is presented in Table~\ref{tab:connectivity-issues}.

\begin{table*}
	\normalsize
	\caption{Networking characteristics of selected wallets}
	\centering
	\begin{tabular}{| l | l l l l l l l | l l l l l l l |}
		\hline
		& \rot{Bitcoin Wallet} & \rot{Bither} & \rot{BRD} & \rot{Dash wallet} & \rot{Electrum} & \rot{Monerujo} & \rot{Simple Bitcoin } & \rot{Bitcoin.com} & \rot{Mycelium} & \rot{Coinomi} & \rot{Jaxx} & \rot{Copay} & \rot{Airbitz} & \rot{Samourai} \\
		\hline
		Trusted node & + & - & - & + & + & + & + & - & - & - & - & - & + & + \\
		F-Droid download & + & - & - & - & - & + & + & - & - & - & - & - & - & - \\
		APK download & + & + & - & - & + & + & + & + & + & + & - & - & - & - \\
		Uses BitcoinJ & + & - & - & + & - & - & + & - & + & + & - & - & - & + \\
		Net monitor & + & + & $\pm$ & + & $\pm$ & + & - &  &  &  &  &  &  &  \\
		Connections & 4-6 & 6 & 3 & 4-6 & 2 & 1 & 10 &  &  &  &  &  &  &  \\
		\hline
	\end{tabular}
	%}
	\label{tab:connectivity-issues}
\end{table*}

\iffalse
Table~\ref{tab:source-code} lists the classes containing implementations of the networking functionality, including transaction broadcasting.
\begin{table}
	\normalsize
	\caption{References to transaction broadcast implementations}
	\centering
	\begin{tabular}{ | l | l | }
		\hline
		BitcoinJ (library) & \texttt{TransactionBroadcast}, \texttt{EnoughAvailablePeers}  \\
		\hline
		Bitcoin wallet & \texttt{BlockchainService} \\    % line 810
		\hline
		Bither & \texttt{PeerManager} \\    % publishTransaction
		\hline
		BRD & \texttt{BPeer}, \texttt{BPeerManager} \\
		\hline
		Dash wallet & \texttt{BlockchainServiceImpl} \\
		\hline
		%GreenBits & \texttt{SendFragment} \\    % line 1123: broadcast via server!
		%\hline
		Electrum & \texttt{WalletTool} \\
		\hline
		Monerujo & \texttt{WalletService} \\    % line 345 - 380
		% boolean success = pendingTransaction.commit("", true);
		% commit implemented natively (JNI) in monerujo.cpp, line 1248.
		% Not exactly clear where is the "broadcast" part, but if the wallet maintains connections only to one "remote node" it doesn't matter
		\hline
		Simple Bitcoin & \texttt{Utils} \\  % line 580
		\hline
	\end{tabular}
	\label{tab:source-code}
\end{table}
\fi


\subsection{Static analysis}

\subsubsection*{FlowDroid}

We scanned the wallets with FlowDroid~\cite{Arzt2014} -- an open source static analysis tool\footnote{The source code is available at~\cite{FlowDroid}.}
It uses data flow analysis to detect execution paths which transfer data from \textit{sources} (functions that may return sensitive data) into \textit{sinks} (functions that send data elsewhere).
The analyses of Bitcoin wallet, Bither, and Monerujo were stopped after a 2~hour timeout.
Samourai was not scanned due to a technical limitation: the application is labeled as unreleased, which prevented us from downloading the APK.
FlowDroid detected potential data leaks in $8$~out of $14$~applications (see Table~\ref{tab:static-analysis}).

\subsubsection*{SmartDec Scanner}
We scanned the wallets with a proprietary static analysis tool SmartDec~Scanner~\cite{SmartDec2018}.
%\footnote{The reference is not provided in this version of the paper to maintain anonymity for the review.}.
We manually inspected the results and summarized the most prevalent privacy-related issues detected by the tool, roughly in decreasing order of potential threat.
Note that these issues do not directly lead to an exploit.

\paragraph{Leak to external storage}
Android provides internal and external storage\footnote{Historically, external storage was assumed to be on a removable memory card, which now may not be the case.}.
An application can only access its own directory in the internal storage.
External storage is available for other applications.

Sensitive data should only be kept in the internal storage.
Android automatically backs up data and settings of applications which did not opt out\footnote{By setting \texttt{android:allowBackup="false"} in the Manifest file.}.
Automatic backups should be disabled for privacy-critical applications.
%Only Electrum and Simple~Bitcoin allow backup (other wallets explicitly forbid it).
%Simple Bitcoin also stores scanned QR~codes in the external storage.

\paragraph{XSS attacks via Javascript in WebView}
One method of developing dynamic user interfaces on Android is using JavaScript inside a \texttt{WebView} -- an Android component that displays web pages.
By default, execution of JavaScript code in \texttt{WebView} is disabled, but this setting can be overridden (\texttt{setJavaScriptEnabled(true)}).
Executing malicious\footnote{Even trusted code, e.g.,~from the application's resources, may contain unintended side effects or bugs, as well as implicitly leak information.} JavaScript code may lead to cross-site scripting (XSS) and other attacks.
We detect two instances of this issue (in \texttt{FragmentSupport} and \texttt{WebViewActivity} classes) in BRD.
In both cases, the warning from Android Lint -- a static analyzer built into the standard Android development environment -- is suppressed.

\paragraph{Insecure connection}
Parameters of a TLS connection in Java are specified in the \texttt{X509TrustManager} class.
Its methods can be overridden to accept all certificates (not only those authenticated by a chain of signatures up to a trusted root CA).
This may lead to a man-in-the-middle attack.
Connections between Electrum servers, unlike the Bitcoin protocol, are encrypted and authenticated with TLS.
Bitcoin wallet and Dash wallet, though communicating with the respective networks via their P2P protocols, use Electrum servers for querying the balance when sweeping a paper wallet.
% (see the \texttt{X509TrustManager}-inherited \texttt{RequestWalletBalanceTask} class).
Bither defines HTTP URLs of its own API (\url{bither.net}) and a block explorer \url{blockchain.info} (class \texttt{BitherUrl}), which can lead to displaying incorrect balances or fake transactions in case of a man-in-the-middle attack.
Simple Bitcoin uses five hard-coded URLs to query current fees.
One of them\footnote{\url{http://api.blockcypher.com/v1/btc/main}.} uses an unencrypted connection.
This may lead to incorrect fee estimation (and thus a potential denial of service attack, as transactions with very low fees may never be confirmed), though this scenario requires four other APIs served over HTTPS to fail simultaneously.

\paragraph{Leak into logs}
Android applications can write to their own log.
Applications can also read their own logs with the \texttt{READ\_LOGS} permission.
Prior to Android~4.0, this permission also granted access to logs of other applications.

It is possible to access logs on a rooted device, or using developer tools.
All wallets log details about their operation, including error messages, which may include sensitive data (e.g.,~the IP address of a trusted node).
This issue is present to some extent in all wallets, as all wallets use logging in exception handlers.
One may find all occurrences of this issue by searching for the methods of the \texttt{Log} class, \texttt{print}, \texttt{println}, and exception handlers with \texttt{ex.printStackTrace()}.
Further investigation is needed to determine the impact and probability of data leaks.

The summary of the results obtained with static analysis (both FlowDroid and SmartDec~Scanner) is presented in Table~\ref{tab:static-analysis}.

\begin{table*}
	\normalsize
	\caption{Static analysis of selected wallets}
	\centering
	\begin{tabular}{| l | l l l l l l l | l l l l l l l |}
		\hline
		& \rot{Bitcoin Wallet} & \rot{Bither} & \rot{BRD} & \rot{Dash wallet} & \rot{Electrum} & \rot{Monerujo} & \rot{Simple Bitcoin } & \rot{Bitcoin.com} & \rot{Mycelium} & \rot{Coinomi} & \rot{Jaxx} & \rot{Copay} & \rot{Airbitz} & \rot{Samourai} \\
		\hline
		Leaks (FlowDroid) & 0 & 4 & 3 & 1 & 0 & 2 & 1 & 0 & 6 & 4 & 0 & 0 & 4 & ? \\
		Leak to ext. storage & - & - & - & - & + & - & + & - & - & - & + & + & + & - \\
		XSS WebView & - & - & + & - & - & - & - & + & - & + & + & + & + & - \\
		Insecure conn. & + & + & - & + & - & - & + & - & - & - & - & - & - & - \\
		Leak into logs & + & + & + & + & + & + & + & + & + & + & + & + & + & + \\
		\hline
	\end{tabular}
	%}
	\label{tab:static-analysis}
\end{table*}
%%%%%%%%%%%%%%%%%%%%%%%%%%%%%%%%%%%%%%%%%%%%%%%%%%



\section{Conclusion} \label{section:Ch04Conclusion}

We studied Android wallets for Bitcoin and the major privacy focused cryptocurrencies.
Most wallets do not satisfy our minimal privacy criteria.
Many wallets obtain dangerous permissions and potentially leak users' private information.
Static analysis revealed defects in their source code.
%P2P wallets do not implement privacy enhancing broadcast mechanisms such as diffusion, which is used by Bitcoin~Core.

%Considering the influx of new people into the blockchain space, secure and usable lightweight blockchain software is needed.
%Research in human-computer interaction is needed to make cryptocurrencies usable by general public.
%Cryptographically sound systems may fail to gain traction due to usability issues~\cite{Ruoti2015}.

Secure development practices are are especially important for mobile wallets, which store lots of personal data and can be easily lost or stolen.
Mobile developers should require as few permissions as possible, open-source the code, provide alternative installation methods (F-Droid, direct APK download), and implement additional privacy-preserving measures to prevent deanonymization of users.