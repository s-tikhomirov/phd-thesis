\section{Related work}
\label{sec:related-work}

%We can roughly divide the related literature in two groups: academic analyses of the concept of payment-channel networks (PCN)
%and experimental analyses of the Lightning Network data.

Multiple research works have shed light on various aspects of payment-channel networks, such as security~\cite{Malavolta2019, Kiayias2019}, privacy~\cite{Malavolta2017, HerreraJoancomarti2019}, concurrency~\cite{Malavolta2017}, routing~\cite{Malavolta2017a, Roos2018, Sivaraman2018, Prihodko2016}, liquidity~\cite{Dandekar2011,MorenoSanchez2018}, efficiency~\cite{Decker2018}, and incentive compatibility~\cite{Engelmann2017}.
These works mainly share the lack of a quantitative analysis of the impact of their findings in the current LN.
%In tour work , we have empirically analyzed the potential severity of the wormhole attack, 
%as well as attacks on value privacy and relationship anonymity, in the current Lightning Network. 

A group of papers more closely related to ours conveys experimental analyses of various aspects of the LN.
Herrera-Joancomart\'{i} et al.~\cite{HerreraJoancomarti2019} describe an adversarial 
strategy to determine the current balance of a channel in the network.
Tang et al.~\cite{Tang2019} study 
the tradeoffs between balance privacy and routing effectiveness. 
Martinazzi~\cite{Martinazzi2019} and Seres et al.~\cite{Seres2019} study the evolution of topological aspects of the LN graph.
Conoscenti et al.~\cite{Conoscenti2019} study the dependency of the LN on payment hubs 
and the rebalancing mechanisms that ameliorate the effect of depleted channels.
Tochner et al.~\cite{Tochner2019} analyze a DoS attack vector based on route hijacking. 
P{\'{e}}rez{-}Sol{\`{a}} et al.~\cite{PerezSola2019} introduce the LockDown attack where the adversary 
prevents a LN node from transacting by depleting the capacity in all its channels.
In comparison, our HTLC depletion attack achieves the same result (a victim node can not forward payments), but exploits the HTLC limit at each channel rather than its capacity.
%\todo[inline]{Pedro:I would not put this text in a footnote. We need to say why related work is different from this}
%Our experiments show that the isolation of a node (or part of a network) 
%exploiting the HTLC limit is cheaper for the adversary than the attack proposed by P{\'{e}}rez{-}Sol{\`{a}} et al. 
Finally, concurrently to our research, Mizrahi and Zohar~\cite{Mizrahi2020} study the HTLC limit and its effects.
Their work, however, does not account for the way LN handles payments below the dust limit.
%\todo[inline]{Pedro: Same here. What's the difference with this work? Aren't they missing the point that 1satoshi payments are not carried out with HTLC?}


%their estimated cost of attacking one node is $15$~EUR ($16.5$~USD), whereas the cost of attacking the whole LN in our case is estimated at $3000$~USD, i.e.,~around $0.6$~USD per node.
