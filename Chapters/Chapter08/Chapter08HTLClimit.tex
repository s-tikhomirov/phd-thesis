\chapter{Throughput limitation of the Lightning network}

\label{Chapter08HTLClimit}

In this Section, we describe an inherent limitation on the number of concurrent payments Lightning channels can handle.\footnote{This Chapter is based on~\cite{Tikhomirov2020a}.}
Due to the size limitations on Bitcoin transactions, a payment channel can hold only a certain number of concurrent unresolved HTLCs.
An attacker can create unresolved payments between two nodes under their control.
This attack blocks the capacity of channels along the route by depleting their "HTLC slots."
This effect may be critical for micro-payment applications -- a significant use case for the LN\@.
This limitation has been pointed out~\cite{EmelyanenkoK2017} but never quantitatively analyzed.

We study the management of concurrent payments in the LN and quantify its negative effect on scalability.
We observe that for micropayments, the forwarding capability of up to $50\%$~of channels is under-utilized.
This phenomenon not only hinders scalability but also opens the door for DoS attacks.
We estimate that a network-wide DoS attack costs within $1.64M$~USD, while isolating the biggest community from the rest of the network costs only $250k$~USD\@.
We also evaluate the evolution of this phenomenon over the two years of LN existence, based on the historical data.
We finally discuss potential countermeasures.


\section{Background} \label{max-htlc-background}

Bitcoin~Core imposes a $100$~KB transaction size limit~\cite{StandardTxBitcoinSE, BitcoinCoreMaxTxWeight}.
More precisely, transaction size is one of the requirements for a \textit{standard} transaction.
Bitcoin~Core nodes, which make up $97\%$~of the network~\cite{CoinDance}, do not propagate non-standard transactions, which are therefore unlikely to get confirmed.
An LN channel cannot contain more than $966$~unsettled HTLCs~\cite{BOLT2Rationale}.
This limit ensures that both counterparties can close the channel using one standard Bitcoin transaction.
We call this limitation the \textit{HTLC limit}.

Despite the perceived focus on micropayments, LN does not fully support payments of very small value.
Every HTLC increases the weight of a potential closing transaction and therefore increases the on-chain fees.
Redeeming very small outputs on-chain can be more expensive than their value.
Therefore, BOLT specifications prescribe that nodes negotiate the \textit{dust limit} before opening a channel.
Nodes do not create HTLCs for payments below this limit.
Such non-existent outputs are called \textit{trimmed HTLCs}~\cite{BOLT3Trimmed}.
Out of the three most popular LN implementations, c-lightning and Eclair use the default dust limit of~$546$~satoshis.
LND estimates the dust limit dynamically based on the current on-chain fees.
We thus assume $546$~satoshis as the dust limit.


\section{The HTLC limit effect on LN scalability}

This section estimates the effect of the HTLC limit on the number of concurrent channel updates.
We use the same dataset \textit{LN20} as in Chapter~\ref{Chapter07LNattacks}.
We derive a series of historical snapshots representing the state of the LN on the first day of each month from April~2018 to February~2020.
We refer to this dataset as \textit{LNHist}.

Let $C$ be the total network capacity (i.e., the sum of all channels' capacities).
We only consider amounts above the dust limit $D$.
Let $a_\textit{avg}$ be the average payment amount.
We define the limit on concurrent updates based solely on capacity as $u_\textit{cap} := C / a_\textit{avg}$.
In contrast, the limit on concurrent updates considering the HTLC limit is $u_\textit{HTLC} = N * 966$, where $N$ is the number of channels.
We remark here that $u_\textit{HTLC}$ does not depend on payment amounts.

Given those two values, we define the \textit{effective update rate} $ur_\textit{eff}$ as the ratio between the actual limit on concurrent payments considering the HTLC limit and the theoretical limit based solely on capacity:

\[ur_\textit{eff} = \frac{min(u_\textit{cap}, u_\textit{HTLC})}{u_\textit{cap}}\]

Note that the effective update rate $ur_\textit{eff}$ depends on the average payment amount, as shown in Figure~\ref{fig:effective-channel-updates}.
Starting from $D$, the effective number of concurrent updates diverges from what could be theoretically possible without the HTLC limit.
We observe that $2\,677$~satoshis ($0.27$~USD) is the \textit{borderline amount}: for higher average payment amounts, the limiting factor for the number of concurrent channel updates is channel capacity.
For amounts between $D$~and~$2\,677$~satoshis, the limiting factor is the HTLC limit.

\begin{figure}[tb]
	\centering
	\includegraphics[width=0.8\columnwidth]{effective-channel-updates}
	\caption{Ratio between the current limit on concurrent channel updates and the theoretically possible capacity-based limit.}
	\label{fig:effective-channel-updates}
\end{figure}

\paragraph{Affected channels}
The $ur_\textit{eff}$ is an aggregated measurement that does not shed light on how the HTLC limit affects individual channels.
We now study how many channels are affected by the HTLC limit.
The number of affected channels depends on the average payment amount $a_\textit{avg}$.
For high values of~$a_\textit{avg}$, it is more likely that the channel's capacity limits its effective update rate, whereas the HTLC limit determines the update rate cap for small values of~$a_\textit{avg}$.
We quantify this as follows.
Given a fixed average payment amount $a_\textit{avg}$, we consider a channel \textit{affected} by the HTLC limit if $u_{\textit{HTLC},a_\textit{avg}} < u_{\textit{cap},a_\textit{avg}}$, i.e, $u_{\textit{eff},a_\textit{avg}} < 100\%$~(\cref{fig:affected-channels}).

\begin{figure}[tb]
	\centering
	\includegraphics[width=0.8\columnwidth]{affected-channels}
	\caption{Share of affected channels for different payment amounts.}
	\label{fig:affected-channels}
\end{figure}


\paragraph{The effect of the HTLC limit over time}

We study the effect of the HTLC limit on the LN using our historical snapshots \emph{LNHist}.
For each monthly snapshot and four assumed average payment amounts, we calculate the share of channels affected by the HTLC limit (\cref{fig:historic-htlc-limited-share}).
As expected, the HTLC limit becomes a more pressing issue with smaller payment amounts, if they are above the dust limit.
We also observe that the share of affected channels has been increasing in the early months of the LN's existence and has remained stable since mid-2019.

We finally study how the \textit{borderline amount} has changed over time.
As~\cref{fig:historic-borderline-amount} shows, the HTLC limit finds its inflection point in payment amount at approximately $2\,500$~satoshis, with the borderline amount stabilizing in mid-2019, after the initial growth.

\begin{figure}[tb]
	\centering
	\includegraphics[width=0.8\columnwidth]{historic-htlc-limited-share}
	\caption{Historic share of HTLC-limited channels.}
	\label{fig:historic-htlc-limited-share}
\end{figure}

\begin{figure}[tb]
	\centering
	\includegraphics[width=0.8\columnwidth]{historic-borderline-amount}
	\caption{Historic borderline amounts.}
	\label{fig:historic-borderline-amount}
\end{figure}


\section{Depleting the Lightning Network}

The HTLC limit enables a network-wide DoS attack.
An adversary connects to both endpoints of the target channel and forwards multiple small payments to itself, but does not finalize them.
After $966$~HTLCs are added, the channel loses its ability to forward payments until some HTLCs expire.
The attacker can thereby deplete a channel, making it unusable.

The cost of this attack depends on the minimum payment amount.
We assume it equal to the dust limit of~$546$~satoshis (the default value in 2 out of 3 major LN implementations).

We calculate the total capital requirements for an attacker to block the LN completely.
To block all $31\,084$~channels, the attacker would send, in the worst case, $966$~payments of~$546$~satoshi to each channel.
This brings the total capital requirements to approximately $163.9482$~BTC ($1.64M$~USD).

Each HTLC defines a timeout, after which the funds are returned to the sender if the receiver provides no preimage.
From our dataset, we see that HTLC timeouts are long: $75.44$~blocks on average.
At a block creation rate of~$10$~minutes per block, this implies that an average HTLC can block the capacity for around $12$~hours.
Therefore, the attacker can render channels useless for around $12$~hours using the same HTLC parameters as regular LN users.

While this rough upper bound estimate suggests a relatively high attack cost, the following optimizations make it more affordable.


\paragraph{Targeting highest-capacity channels}
The attack impact can be maximized by targeting the highest-capacity channels.
For example, it requires $0.05$~BTC to block 10~top channels with combined capacity of~$17.91$~BTC (\cref{fig:block-top-channels}).

\begin{figure}[tb]
	\centering
	\includegraphics[width=0.8\columnwidth]{block-top-channels}
	\caption{Effectiveness of targeting highest-capacity channels.}
	\label{fig:block-top-channels}
\end{figure}

\paragraph{Real HTLC limit}
Our calculations are based on the maximum number of concurrent HTLCs ($483$) as defined by BOLT specifications.
LN implementations may choose other values.
In particular, Eclair and c-lightning enforce a lower default HTLC limit ($30$).
Lower limits mean that in the real network, the attacker would need to create fewer HTLCs to block channels between c-lightning and Eclair nodes than between LND nodes (which by default support $483$~concurrent HTLCs per channel).
LND makes up $91\%$~of the nodes in the network, and Eclair is another $1\%$~\cite{Mizrahi2020}.
That brings the real average HTLC limit to $442.23$~and lowers the attack cost by~$8.44\%$.

\paragraph{Multi-hop payments}
Our estimation assumes single-hop payments.
An attacker can leverage multi-hop payments to multiply the effect of the committed capital, connecting to both ends of a $20$-hop~\cite{Bolt4OnionRouting} payment path and performing a payment to itself that never gets completed.
This technique resembles capacity-based griefing attacks~\cite{PerezSola2019,Rohrer2019} but entails much lower capital requirements.

\paragraph{Optimizing the attack based on communities}
The attacker may wish to prevent different parts of the network from transacting to each other.
To evaluate this possibility, we first divide the network into \textit{communities} using the Clauset-Newman-Moore greedy modularity maximization algorithm~\cite{Clauset2004}.
Then we consider a scenario where the attacker tries to block the channels that connect communities rather than channels within communities.
For a chosen number $N$~of the largest communities, we calculate how many channels the attacker has to block to split the network into at least $N+1$~parts: the $N$~largest communities and the rest of the network (\cref{fig:isolate-communities}).
We infer, e.g.,~that the attacker needs to block $4\,670$~channels ($13\%$~of all channels) to isolate the largest community from the rest of the network, locking up $25$~BTC ($250k$~USD) -- or around $2.8\%$~of the total LN capacity.

\begin{figure}[tb]
	\centering
	\includegraphics[width=0.8\columnwidth]{isolate-communities}
	\caption{Number of channels to cut to isolate the largest communities.}
	\label{fig:isolate-communities}
\end{figure}


\section{Discussion}
Our simplistic model does not fully reflect all the details of payment handling.
In particular, we do not account for multi-hop payments and payments that try multiple paths before succeeding.
We do not reflect the unequal forwarding ability of a unit of capacity at a well-connected node, as opposed to a poorly connected one.
We also do not account for non-public channels, which may account for $28\%$~of all channels~\cite{BitMEXPrivateChannels}.
Nevertheless, our approach allows us to calculate the effect of the HTLC limit, as we derive both estimations (capacity-based limit and HTLC limit)  under the same assumptions.

The negative effect of the HTLC limit manifests itself at low payment amounts.
This threatens potential LN applications involving micro-payments, such as paying for online content~\cite{Poon2016}.
Our calculations show that for payments of~$1\,000$~satoshis ($0.01$~USD), the network-wide rate of concurrent channel updates is $60\%$~lower than it could have been based solely on capacity limitations.

The low value for the default minimum payment amount and the reduced number of in-flight payments open a DoS attack vector with a moderate cost for the adversary.
Note that the capital in the attacker's channels will be recouped after the HTLCs time out.
Moreover, the unequal distribution of connectivity in the current LN paves the way for optimized attacks where the attacker focuses on high-capacity or inter-community channels to disrupt the transfer of value across the network.


\section{Countermeasures}
One of the limiting factors for payment throughput is the total available capacity.
This limitation is overcome by opening new channels, a countermeasure naturally implemented with the growing LN adoption.
The HTLC limit issue is more challenging as it comes from the limitations of the Bitcoin and Lightning protocols themselves.
Therefore, more fundamental changes are needed to reduce the information required to carry out the functionality encoded in HTLCs.
One countermeasure involves replacing HTLCs with atomic multi-hop locks (AMHL)~\cite{Malavolta2019}.
While an HTLC requires a digital signature, hash value, and a timelock, AMHL only requires a digital signature and a timelock while providing the same functionality.
This countermeasure would reduce the number of bytes required per in-flight payment and increase the number of payments handled concurrently.
While not removing the limitation on the number of concurrent payments, this countermeasure raises this limit, reducing its harmful effect.

Another countermeasure is implementing payments across multiple channels with small \textit{packetized payments} that are not secured by HTLCs~\cite{Robinson2019}.
Instead, this mechanism relies on economic incentives: if a counterparty misbehaves, an honest party stops the interaction and only loses a negligible value.

Firewall-like solutions have been proposed to protect Lightning nodes against HTLC-based DoS attempts~\cite{Jager2020}.


\section{Conclusion}

We have quantitatively analyzed the negative effect on scalability produced by the limit on concurrent payments in the LN\@.
The LN's limited concurrency implies that an adversary can block the complete network investing around $1.64M$~USD ($18.5\%$~of the network capacity).
In comparison to~\cite{PerezSola2019}, our HTLC depletion attack achieves the same result (a victim node cannot forward payments) but exploits the HTLC limit at each channel rather than its capacity.
Compared to~\cite{Mizrahi2020}, we properly account for the way LN handles payments below the dust limit.
The attack cost can be substantially reduced by targeting highly valuable channels (e.g., high-capacity channels or those connecting the network's largest communities).
