\chapter{SmartCheck: Static analysis of Ethereum smart contracts}

\label{Chapter11SmartCheck}

%\usepackage{yax}

% \usevalue SolidityUncheckedMath:name

\let\letcs\texapiletcs % switch to texapi definition

\setparameter SolidityBalanceEquality :
id 				= SOLIDITY\_BALANCE\_EQUALITY
name 			= {Balance equality}
severity 		= 2
severityWord 	= medium
occur			= 113
occurWord		= $113$

\setparameter SolidityBytesByte :
id 				= SOLIDITY\_BYTES\_BYTE
name 			= {Byte array}
severity 		= 1
severityWord 	= low
occur			= 7
occurWord		= $7$

\setparameter SolidityCallValue :
id 				= SOLIDITY\_CALL\_VALUE
name 			= {Transfer forwards all gas}
severity 		= 3
severityWord 	= high
occur			= 275
occurWord		= $275$

\setparameter SolidityDosWithThrow :
id 				= SOLIDITY\_DOS\_WITH\_THROW
name 			= {DoS by external contract}
severity 		= 3
severityWord 	= high
occur			= 7864
occurWord		= $7\,864$

\setparameter SolidityErc20ApiViolation :
id 				= SOLIDITY\_ERC20\_API\_VIOLATION
name 			= {Token API violation}
severity 		= 1
severityWord 	= low
occur			= 1410
occurWord		= $1\,410$

\setparameter SolidityGasLimitAndLoops :
id 				= SOLIDITY\_GAS\_LIMIT\_AND\_LOOPS
name 			= {Costly loop}
severity 		= 2
severityWord 	= medium
occur			= 2610
occurWord		= $2\,610$

\setparameter SolidityIntegerDivision :
id 				= SOLIDITY\_INTEGER\_DIVISION
name 			= {Integer division}
severity 		= 1
severityWord 	= low
occur			= 1727
occurWord		= $1\,727$

\setparameter SolidityLockedMoney :
id 				= SOLIDITY\_LOCKED\_MONEY
name 			= {Locked money}
severity 		= 2
severityWord 	= medium
occur			= 530
occurWord		= $530$

\setparameter SolidityMaliciousLibraries :
id 				= SOLIDITY\_MALICIOUS\_LIBRARIES
name 			= {Malicious libraries}
severity 		= 1
severityWord 	= low
occur			= 1395
occurWord		= $1\,395$

\setparameter SolidityPragmasVersion :
id 				= SOLIDITY\_PRAGMAS\_VERSION
name 			= {Compiler version not fixed}
severity 		= 1
severityWord 	= low
occur			= 3699
occurWord		= $3\,699$

\setparameter SolidityPrivateModifier :
id 				= SOLIDITY\_PRIVATE\_MODIFIER
name 			= {\texttt{private} modifier}
severity 		= 1
severityWord 	= low
occur			= 1223
occurWord		= $1\,223$

\setparameter SolidityRedundantFallbackReject :
id 				= SOLIDITY\_REDUNDANT\_FALLBACK\_REJECT
name 			= {Redundant fallback function}
severity 		= 1
severityWord 	= low
occur			= 64
occurWord		= $64$

\setparameter SolidityReentrancyExternalCall :
id 				= SOLIDITY\_REENTRANCY\_EXTERNAL\_CALL
name 			= {Re-entrancy}
severity 		= 3
severityWord 	= high
occur			= 4015
occurWord		= $4\,015$

\setparameter SoliditySend :
id 				= SOLIDITY\_SEND
name 			= {\texttt{send} instead of \texttt{transfer}}
severity 		= 2
severityWord 	= medium
occur			= 3370
occurWord		= $3\,370$

\setparameter SolidityStyleGuideViolation :
id 				= SOLIDITY\_STYLE\_GUIDE\_VIOLATION
name 			= {Style guide violation}
severity 		= 1
severityWord 	= low
occur			= 1626
occurWord		= $1\,626$

\setparameter SolidityTimestampDependence :
id 				= SOLIDITY\_TIMESTAMP\_DEPENDENCE
name 			= {Timestamp dependence}
severity 		= 2
severityWord 	= medium
occur			= 7692
occurWord		= $7\,692$

\setparameter SolidityTxOrigin :
id 				= SOLIDITY\_TX\_ORIGIN
name 			= {Using \texttt{tx.origin}}
severity 		= 2
severityWord 	= medium
occur			= 197
occurWord		= $197$

\setparameter SolidityUncheckedCall :
id 				= SOLIDITY\_UNCHECKED\_CALL
name 			= {Unchecked external call}
severity 		= 3
severityWord 	= high
occur			= 986
occurWord		= $986$

\setparameter SolidityUncheckedMath :
id 				= SOLIDITY\_UNCHECKED\_MATH
name 			= {Unchecked math}
severity 		= 1
severityWord 	= low
occur			= 0
occurWord		= $0$

\setparameter SolidityVar :
id 				= SOLIDITY\_VAR
name 			= {Unsafe type inference}
severity 		= 2
severityWord 	= medium
occur			= 638
occurWord		= $638$

\setparameter SolidityVisibility :
id 				= SOLIDITY\_VISIBILITY
name 			= {Implicit visibility level}
severity 		= 1
severityWord 	= low
occur			= 81160
occurWord		= $81\,160$

\let\letcs\etoolboxletcs % go back to etoolbox definition


We present SmartCheck -- a static analysis tool for smart contracts in Solidity.

Ethereum is a major blockchain-based platform for smart contracts -- Turing complete programs that are executed in a decentralized network and usually manipulate digital units of value.
Solidity is the most mature high-level smart contract language.
Ethereum is a hostile execution environment, where anonymous attackers exploit bugs for immediate financial gain.
Developers have a very limited ability to patch deployed contracts.
Hackers steal up to tens of millions of dollars from flawed contracts, a well-known example being ``The DAO``, broken in June~2016.
Advice on secure Ethereum programming practices is spread out across blogs, 
papers, and tutorials.
Many sources are outdated due to a rapid pace of development in this field.
Automated vulnerability detection tools, which help detect potentially problematic language constructs, are still underdeveloped in this area.

We provide a comprehensive classification of code issues in Solidity and implement SmartCheck -- an extensible static analysis tool that detects them\footnote{The source code is available at \url{https://github.com/smartdec/smartcheck}.}.
SmartCheck translates Solidity source code into an XML-based intermediate representation and checks it against XPath patterns.
We evaluated our tool on a big dataset of real-world contracts and compared the results with manual audit on three contracts.
Our tool reflects the current state of knowledge on Solidity vulnerabilities and shows significant improvements over alternatives.
SmartCheck has its limitations, as detection of some bugs requires more sophisticated techniques such as taint analysis or even manual audit.
We believe though that a static analyzer should be an essential part of contract developers' toolbox, letting them fix simple bugs fast and allocate more effort to complex issues.

This Chapter is based on~\cite{Tikhomirov2018}.
\footnote{Contributions of the author of this thesis include: research on the types of vulnerabilities, formalize the vulnerability definitions in the SmartCheck's internal format, writing the paper.}
Solidity has evolved between the publication of this paper (2018) and the writing of this thesis (2020).
Comments entitled "2020 update" describe the changes in applicability of the vulnerability patterns.
The author thanks Evgeny Marchenko for the help in adding these updates.


%%%% SECTION 1: INTRODUCTION %%%%
\section{Introduction}

Ethereum was introduced in 2014 and launched in 2015~\cite{Buterin2014}.
Ethereum nodes store data, execute smart contracts, and maintain a shared view of the global state using a proof-of-work consensus mechanism similar to that in Bitcoin~\cite{Tikhomirov2017}.
Contrary to previous attempts at blockchain programming, e.g.,~Bitcoin scripting, Ethereum language is Turing complete and thus able to express arbitrarily complex logic.

Developers write contracts in high-level languages (the most popular and mature one is Solidity) and compile them to bytecode of the Ethereum virtual machine (EVM) -- a stack-based VM operating on 256-bit words\footnote{It is also possible to write contracts in bytecode directly.}.
Compared to general purpose VMs like the Java virtual machine, EVM is relatively simple, executes deterministically, and natively supports certain cryptographic primitives~\cite{Buterin2017}.
A contract is deployed by broadcasting a transaction containing its bytecode and initialization parameters.
Miners include it in a block, permanently storing the contract at a unique blockchain address.
Users interact with the contract by broadcasting transactions with its address, the function to be called, and its arguments.
Upon request, the contract can call other contracts and send units of \textit{ether} -- the Ethereum native cryptocurrency -- to users or other contracts.

To make spamming costly, the Ethereum protocol specifies a cost (denominated in \textit{gas} units) for each EVM operation~\cite{Wood2014}.
A user pays upfront for the expected amount of gas the computation will consume and gets a partial refund after a successful execution.
If an exception (including ``out of gas``) occurs, all state changes are reverted, but the gas may not be refunded\footnote{Gas refunds depend on the exception type: \texttt{assert} consumes all gas, \texttt{require} does not (starting from the Byzantium release in October~2017).}.
The ether price of a gas unit is determined by the market.

Ethereum allows people and companies globally to programmatically encode and trustlessly enforce complex financial agreements.
This opens up new business models and constitutes a dramatic change in the digital economy.
New software development tools are required to ensure correctness and security of smart contracts.


\subsection{Security challenges in Ethereum} \label{SecurityChallenges}

Security is a primary concern in Ethereum programming for multiple reasons:
\begin{itemize}
	\item \textbf{Unfamiliar execution environment}.
	Ethereum differs from centrally managed execution environments, be that mobile, desktop, or cloud.
	Developers are not used to their code being executed by a global network of anonymous, mutually distrusting, profit-driven nodes.
	\item \textbf{New software stack}.
	The Ethereum stack (the Solidity compiler, the EVM, the consensus layer, etc) is under development, with security vulnerabilities still being discovered~\cite{chriseth2017}.
	\item \textbf{Very limited ability to patch contracts}.
	A deployed contract can not be patched\footnote{Though workarounds exist, such as proxy contracts redirecting calls to an adaptable address of the latest version of the main contract.}.
	This makes a popular ``move fast and break things`` motto inapplicable: a contract must be correct before deployment.
	\item \textbf{Anonymous financially motivated attackers}.
	Compared to many cybercrimes, exploiting smart contracts offers higher gains (the prices of cryptocurrencies have been increasing rapidly), easier cashing out (ether and tokens are instantly tradable), and lower risk of punishment due to anonymity.
	\item \textbf{Rapid pace of development}.
	Blockchain companies strive to release their products fast, often at the expense of security.
	\item \textbf{Suboptimal high-level language}.
	Some argue that Solidity itself inclines programmers towards unsafe development practices~\cite{ydtm2016}.
\end{itemize}

A textbook example of an Ethereum contract exploit is the DAO hack.
The DAO was an Ethereum-based venture capital fund.
In May~2016, it collected around \$150 million in the largest crowdfunding campaign to date.
In June~2016, an unknown hacker exploited multiple vulnerabilities in the DAO code and gained control over ether worth around \$50 million at that time~\cite{Sirer2016}.
Though the Ethereum protocol executed correctly, the core developers proposed a hard fork to restore stakeholders' deposits, violating the premise of decentralized applications running ``exactly as programmed``\footnote{Concerns about Ethereum's governance lead to the creation of Ethereum~Classic~\cite{EthereumClassic} -- a continuation of the Ethereum blockchain without the DAO fork.}.
More recent examples of high-profile loss of ether due to software vulnerabilities include two incidents with the Parity multi-signature wallet in July and November~2017~\cite{Palladino2017}.
These and many similar events of a smaller scale illustrate the importance of security in Ethereum.

For the purposes of this paper, we assume correctness of the Ethereum core infrastructure and focus on security from a contract developer's viewpoint.
We classify issues in Solidity source code and develop a static analysis tool -- SmartCheck -- that detects them.
We test SmartCheck on a large set of real-world contracts and measure the relative prevalence of various code issues.
SmartCheck shows significant improvements over existing alternatives in terms of false discovery rate (FDR) and false negative rate (FNR).

%\input{smartcheck-code-correctness-approaches.tex}

%%%% SECTION 2: CLASSIFICATION %%%%
\section {Classification of issues in Solidity code}

We classify Solidity code issues as follows (based on~\cite{Henney2003}):
\begin{itemize}
	\item \textbf{Security} issues lead to exploits by a malicious user account or contract;
	\item \textbf{Functional} issues cause the violation of the intended functionality\footnote{Though without a specification we only assume what the intended functionality is.};
	\item \textbf{Operational} issues lead to run-time problems, e.g.,~bad performance;
	\item \textbf{Developmental} issues make code difficult to understand and improve.
\end{itemize}
We differentiate between functional and security issues: the former pose problems even without an adversary (though an external malicious actor can aggravate the situation), while the latter do not. Our primary sources are~\cite{Consensys2016, Solidity2017, Atzei2017, Delmolino2016, Chen2017, OpenZeppelin2017a}.
See Table~\ref{AllPatternsTable} for a summary of all issues (the second column denotes severity: 3 -- high, 2 -- medium, 1 -- low).

\subsection{Security issues}


%% SOLIDITY_BALANCE_EQUALITY
\subsubsection{\let\letcs\texapiletcs \usevalue SolidityBalanceEquality:name \let\letcs\etoolboxletcs}
\label{SolidityBalanceEquality}

Avoid checking for strict balance equality: an adversary can forcibly send ether to any account by mining or via \texttt{selfdestruct}.
\begin{lstlisting}[language=Solidity]
if (this.balance == 42 ether) { /* ... */} // bad
if (this.balance >= 42 ether) { /* ... */} // good
\end{lstlisting}

The pattern detects comparison expressions with \texttt{==} which contain \texttt{this.balance} as either left- or right-hand side.

%% SOLIDITY_UNCHECKED_CALL
\subsubsection{\let\letcs\texapiletcs \usevalue SolidityUncheckedCall:name \let\letcs\etoolboxletcs} \label{SolidityUncheckedCall}

Expect calls to external contract to fail.
When \texttt{send}ing ether, check for the return value and handle errors.
The recommended way of doing ether transfers is \texttt{transfer} (see Section~\ref{SoliditySend}).
\begin{lstlisting}[language=Solidity]
addr.send(42 ether); // bad
if (!addr.send(42 ether)) revert; // better
addr.transfer(42 ether); // good
\end{lstlisting}

The pattern detects an external function call (\texttt{call}, \texttt{delegatecall}, or \texttt{send}) which is not inside an \texttt{if}-statement.

%% SOLIDITY_DOS_WITH_THROW
\subsubsection{\let\letcs\texapiletcs \usevalue SolidityDosWithThrow:name \let\letcs\etoolboxletcs} \label{SolidityDosWithThrow}

A conditional statement (\texttt{if}, \texttt{for}, \texttt{while}) should not depend on an external call: the callee may permanently fail (\texttt{throw} or \texttt{revert}), preventing the caller from completing the execution.

In the following example, the caller expects the oracle to return an integer value (\texttt{badOracle.answer()}), but the actual oracle implementation may throw an exception in some or all cases.

\begin{lstlisting}[language=Solidity]
function dos(address oracleAddr) public {
	badOracle = Oracle(oracleAddr);
	if (badOracle.answer() < 42) { revert; }
	// ...
}
\end{lstlisting}

This rule contains multiple patterns:
\begin{itemize}
	\item an \texttt{if}-statement with an external function call in the condition and a \texttt{throw} or a \texttt{revert} in the body;
	\item a \texttt{for}- or an \texttt{if}-statement with an external function call in the condition.
\end{itemize}

%% SOLIDITY_SEND
\subsubsection{\let\letcs\texapiletcs \usevalue SoliditySend:name \let\letcs\etoolboxletcs} \label{SoliditySend}

The recommended way to perform ether payments is \texttt{addr.transfer(x)}, which automatically throws an exception if the transfer is unsuccessful, preventing the problem described in Section~\ref{SolidityUncheckedCall}.
The pattern detects the \texttt{send} keyword.

\paragraph{2020 update}
As of 2020, the best practice is that \texttt{send} is preferred to \texttt{call} and \texttt{transfer} (see comment in~\ref{SolidityCallValue}).

%% SOLIDITY_REENTRANCY_EXTERNAL_CALL
\subsubsection{\let\letcs\texapiletcs \usevalue SolidityReentrancyExternalCall:name \let\letcs\etoolboxletcs} \label{SolidityReentrancyExternalCall}

%A function call from contract \texttt{A} to contract \texttt{B} (including ether transfers) passes control from \texttt{A} to \texttt{B}, which can call \texttt{A} back before returning.
Consider the following code:

\begin{lstlisting}[language=Solidity]
pragma solidity 0.4.19;
contract Fund {
	mapping(address => uint) balances;
	function withdraw() public {
		if (msg.sender.call.value(balances[msg.sender])())
		balances[msg.sender] = 0;
	}
}
\end{lstlisting}

The contract at \texttt{msg.sender} can get multiple refunds and retrieve all \texttt{Fund}'s ether by recursively calling \texttt{withdraw} before its share is set to \texttt{0}.
Besides, it can modify the state of some third contract, which \texttt{Fund} depends on.
Use the ``checks -- effects -- interactions`` pattern: first check the invariants, then update the internal state, then communicate with external entities (see also Section~\ref{SoliditySend}):
\begin{lstlisting}[language=Solidity]
function withdraw() public {
	uint balance = balances[msg.sender];	
	balances[msg.sender] = 0;
	msg.sender.transfer(balance);
	// state reverted, balance restored if transfer fails
}
\end{lstlisting}

The pattern detects an external function call which is followed by an internal function call.

%% SOLIDITY_MALICIOUS_LIBRARIES
\subsubsection{\let\letcs\texapiletcs \usevalue SolidityMaliciousLibraries:name \let\letcs\etoolboxletcs} \label{SolidityMaliciousLibraries}

Third-party libraries can be malicious.
Avoid external dependencies or ensure that third-party code implements only the intended functionality.
The pattern simply detects the \texttt{library} keyword (and thus produces some false positives).

%% SOLIDITY_TX_ORIGIN
\subsubsection{\let\letcs\texapiletcs \usevalue SolidityTxOrigin:name \let\letcs\etoolboxletcs} \label{SolidityTxOrigin}

Contracts can call each others' public functions.
\texttt{tx.origin} is the first account in the call chain (always an externally owned one, i.e.,~not a contract); \texttt{msg.sender} is the immediate caller.
For instance, in a call chain \texttt{A} $\rightarrow$ \texttt{B} $\rightarrow$ \texttt{C}, from the \texttt{C}'s viewpoint, \texttt{tx.origin} is \texttt{A}, and \texttt{msg.sender} is \texttt{B}.

Use \texttt{msg.sender} instead of \texttt{tx.origin} for authentication.
Consider a wallet:

\begin{lstlisting}[language=Solidity]
pragma solidity 0.4.19;
contract TxWallet {
    address private owner;
    function TxWallet() { owner = msg.sender; }
    function transferTo(address dest, uint amount) public {
        require(tx.origin == owner);	// authentication
        dest.transfer(amount);
    }
}
\end{lstlisting}

User sends ether to the address of the \texttt{TxAttackerWallet}, which forwards the call to a \texttt{TxWallet} and obtains all funds, acting as the user (\texttt{tx.origin}):

\begin{lstlisting}[language=Solidity]
pragma solidity 0.4.19;
interface TxWallet {
    function transferTo(address dest, uint amount);
}
contract TxAttackerWallet {
    address private owner;
    function TxAttackerWallet() { owner = msg.sender; }
    function() payable {
        TxWallet(msg.sender).transferTo(owner, msg.sender.balance);
    }
}
\end{lstlisting}

The pattern detects the environmental variable \texttt{tx.origin}.

%% SOLIDITY_CALL_VALUE
\subsubsection{\let\letcs\texapiletcs \usevalue SolidityCallValue:name \let\letcs\etoolboxletcs} \label{SolidityCallValue}

Solidity provides many ways to transfer ether (see Section~\ref{SoliditySend}).
\texttt{addr.call.value(x)()} transfers \texttt{x} ether and forwards all gas to \texttt{addr}, potentially leading to vulnerabilities like re-entrancy (see Section~\ref{SolidityReentrancyExternalCall}).
The recommended way to transfer ether is \texttt{addr.transfer(x)}, which only provides the callee with a ``stipend`` of 2300 gas.
The pattern detects functions whose name is \texttt{call.value} and whose argument list is empty.

\paragraph{2020 update}
In 2019, Ethereum underwent the Istanbul upgrade.
Among other modifications, gas prices for some operations were increased~\cite{EIP1884}.
As a result, the call subsidy of $2300$ gas is often not sufficient to handle the call.
Therefore it is recommended to forward more than $2300$ gas (probably all available gas) when calling external contracts.
In the light of this pattern, \texttt{transfer} is no longer preferred over \texttt{send}.
Both these commands are considered undesirable.
The preferred way to interact with external contracts is \texttt{call}.


\subsection{Functional issues}

%% SOLIDITY_INTEGER_DIVISION
\subsubsection{\let\letcs\texapiletcs \usevalue SolidityIntegerDivision:name \let\letcs\etoolboxletcs} \label{SolidityIntegerDivision}

Solidity supports neither floating-point nor decimal types.
For integer division, the quotient is rounded down.
Account for it, especially when calculating ether or token amounts.
The pattern detects division (\texttt{/}) where the the numerator and the denominator are number literals.

%% SOLIDITY_LOCKED_MONEY
\subsubsection{\let\letcs\texapiletcs \usevalue SolidityLockedMoney:name \let\letcs\etoolboxletcs} \label{SolidityLockedMoney}

Contracts programmed to receive ether should implement a way to withdraw it, i.e.,~call \texttt{transfer} (recommended), \texttt{send}, or \texttt{call.value} at least once.
The patterns detects contracts that contain a \texttt{payable} function but contain neither of the withdraw-enabling functions mentioned above.

\paragraph{2020 update}
In Solidity 0.5.0, the address type has been split into \texttt{address} and \texttt{address payable}~\cite{Solidity050}.
Ether can only be sent to \texttt{address payable}.
A contract developer is thus forced to consider where money can be sent from their contract.
However, the issue captured in this pattern persists on the receiving side.
It is still possible to write a contract that receives money but cannot send it elsewhere.

%% SOLIDITY_UNCHECKED_MATH
\subsubsection{\let\letcs\texapiletcs \usevalue SolidityUncheckedMath:name \let\letcs\etoolboxletcs} \label{SolidityUncheckedMath}

Solidity is prone to integer over- and underflow\footnote{Referred to as simply overflow for brevity.}.
Overflow leads to unexpected effects and can lead to loss of funds if exploited by a malicious account.
Use the SafeMath library\footnote{See Section~\ref{SolidityMaliciousLibraries} for advice on library usage.} that checks for overflows (multiple implementations exist, e.g.~\cite{OpenZeppelin2017}).
The pattern detects arithmetic operations \texttt{+}, \texttt{-}, \texttt{*}, which are not inside a conditional statement.
This rule was temporarily muted for testing (Section~\ref{SectionResults}) due to a high false positive rate.


%% SOLIDITY_TIMESTAMP_DEPENDENCE
\subsubsection{\let\letcs\texapiletcs \usevalue SolidityTimestampDependence:name \let\letcs\etoolboxletcs} \label{SolidityTimestampDependence}

Miners can manipulate environmental variables and are likely to do so if they can profit from it.
Consider a lottery that distributes prizes depending on whether \texttt{now} (alias for \texttt{block.timestamp}) is odd or even:

\begin{lstlisting}[language=Solidity]
	if (now % 2 == 0) winner = pl1; else winner = pl2;
\end{lstlisting}

A miner can tweak the timestamp and gain unfair advantage.
%Use block numbers and average time between blocks to estimate the current time.
Use secure sources of randomness, such as RANDAO~\cite{RANDAO2017}.
The pattern detects the environmental variable \texttt{now}.

\paragraph{2020 update}
Implicit type inference from \texttt{var} was deprecated in Solidity 0.5.0~\cite{Solidity050}.
The pattern no longer applies.

%% SOLIDITY_VAR
\subsubsection{\let\letcs\texapiletcs \usevalue SolidityVar:name \let\letcs\etoolboxletcs} \label{SolidityVar}

Solidity supports type inference: the type of \texttt{i} in \texttt{var i = 42;} is the smallest integer type sufficient to store the right-hand side value (\texttt{uint8}).
Consider a \texttt{for}-loop:

\begin{lstlisting}[language=Solidity]
for (var i = 0; i < array.length; i++) { /*...*/ }
\end{lstlisting}

The type of \texttt{i} is inferred to \texttt{uint8}.
If \texttt{array.length} is bigger than \texttt{256}, an overflow will occur.
Explicitly define the type when declaring integer variables:

\begin{lstlisting}[language=Solidity]
for (uint256 i = 0; i < array.length; i++) { /*...*/ }
\end{lstlisting}

The pattern detects assignments where the left-hand side is a \texttt{var} and the right-hand side is an integer (matches \texttt{\^{}[0-9]+\$}).

\subsection{Operational issues}

%% SOLIDITY_BYTES_BYTE
\subsubsection{\usevalue SolidityBytesByte:name } \label{SolidityBytesByte}

Use \texttt{bytes} instead of \texttt{byte[]} for lower gas consumption.
The pattern detects the construction \texttt{byte[]}.

%% SOLIDITY_GAS_LIMIT_AND_LOOPS
\subsubsection{\usevalue SolidityGasLimitAndLoops:name } \label{SolidityGasLimitAndLoops}

Ethereum is a very resource-constrained environment.
Prices per computational step are orders of magnitude higher than with centralized cloud providers.
Moreover, Ethereum miners impose a limit on the total number of gas consumed in a block.
In the following example, if \texttt{array.length} is large enough, the function exceeds the block gas limit, and transactions calling it will never be confirmed:

\begin{lstlisting}[language=Solidity]
for (uint256 i = 0; i < array.length; i++) { costlyF(); }
\end{lstlisting}

This becomes a security issue, if an external actor influences \texttt{array.length}.
E.g.,~if \texttt{array} enumerates all registered addresses, and registration is open, an adversary can register many addresses, causing denial of service.
The rule includes two patterns:
\begin{itemize}
	\item a \texttt{for}-statement with a function call or an identifier inside the condition;
	\item a \texttt{while}-statement with a function call inside the condition.
\end{itemize}

\subsection{Developmental issues}

%% SOLIDITY_ERC20_API_VIOLATION
\subsubsection{\usevalue SolidityErc20ApiViolation:name } \label{SolidityErc20ApiViolation}

ERC20 is the de-facto standard API for implementing \textit{tokens} -- transferable units of value managed by a contract.
Exchanges and other third-party services may struggle to integrate a token that does not conform to it.
Certain ERC20 functions (\texttt{approve}, \texttt{transfer}, \texttt{transferFrom}) return a \texttt{bool} indicating whether the operation succeded.
It is not recommended to throw exceptions (\texttt{revert}, \texttt{throw}, \texttt{require}, \texttt{assert}) inside those functions.
Note that library functions may also throw exceptions (see Section~\ref{SolidityMaliciousLibraries}).

\begin{lstlisting}[language=Solidity]
function transferFrom(address _spender, uint _value)
returns (bool success) {
	require (_value < 20 wei);
	// ...
}
\end{lstlisting}

The pattern detects a contract inherited from a contract with a name including the word ``token``, which may throw exceptions from inside one of the functions mentioned above.

%% SOLIDITY_PRAGMAS_VERSION
\subsubsection{\usevalue SolidityPragmasVersion:name } \label{SolidityPragmasVersion}

Solidity source files indicate the versions of the compiler they can be compiled with:
\begin{lstlisting}[language=Solidity]
pragma solidity ^0.4.19;	// bad: 0.4.19 and above
pragma solidity 0.4.19;		// good: 0.4.19 only
\end{lstlisting}
It is recommended to follow the latter example, as future compiler versions may handle certain language constructions in a way the developer did not foresee.
The pattern detects the version operator~\texttt{\^{}} in the \texttt{pragma} directive.

%% SOLIDITY_PRIVATE_MODIFIER
\subsubsection{\usevalue SolidityPrivateModifier:name } \label{SolidityPrivateModifier}

Contrary to a popular misconception, the \texttt{private} modifier does not make a variable invisible.
Miners have access to all contracts' code and data.
Developers must account for the lack of privacy in Ethereum.
The pattern detects state variable declarations with a  \texttt{private} modifier.

\paragraph{2020 update}
Since Solidity 0.5.0, a visibility modifier is mandatory for all functions~\cite{Solidity050}.
The pattern no longer applies.

%% SOLIDITY_REDUNDANT_FALLBACK_REJECT
\subsubsection{\usevalue SolidityRedundantFallbackReject:name } \label{SolidityRedundantFallbackReject}

Contracts should reject unexpected payments (see Sections~\ref{SolidityBalanceEquality}, \ref{SolidityLockedMoney}).
Before Solidity 0.4.0, it was done manually:
\begin{lstlisting}[language=Solidity]
function () payable { throw; }
\end{lstlisting}
Starting from Solidity 0.4.0, contracts without a fallback function automatically revert payments, making the code above redundant.
The pattern detects the described construction (only if the \texttt{pragma} directive indicates the compiler version not lower than 0.4.0).


%% SOLIDITY_STYLE_GUIDE_VIOLATION
\subsubsection{\usevalue SolidityStyleGuideViolation:name } \label{SolidityStyleGuideViolation}

In Solidity, function\footnote{With the exception of constructors: they must share the name with the contract and thus usually start with an uppercase letter.} and event names usually start with a lower- and uppercase letter respectively:
\begin{lstlisting}[language=Solidity]
function Foo(); // bad
event logFoo(); // bad
function foo(); // good
event LogFoo(); // good
\end{lstlisting}
Violating the style guide decreases readability and leads to confusion.
The pattern detects the described constructions.

%% SOLIDITY_VISIBILITY
\subsubsection{\usevalue SolidityVisibility:name } \label{SolidityVisibility}

The default function visibility level in Solidity is \texttt{public}.
Explicitly define function visibility to prevent confusion.
\begin{lstlisting}[language=Solidity]
function foo() { /*...*/ } // bad
function foo() public { /*...*/ } // good
function bar() private { /*...*/ } // good
\end{lstlisting}

The pattern detects function and variable definitions without a visibility modifier.

\definecolor{lightlightgray}{gray}{0.90}

\begin{table}[]
	\centering
	\caption{Code issues detected by SmartCheck \\ (gray background -- false positives  possible)}
%	\resizebox{0.5\textwidth}{!}{
	\begin{tabular}{|p{0.25\linewidth}|p{0.03\linewidth}|p{0.64\linewidth}|}
	%\begin{tabular}{|c|c|c|}
		\hline
		\textbf{Name} & \textbf{S.} & \textbf{Description} \\
		\hline
		%% SOLIDITY_BALANCE_EQUALITY
		\usevalue SolidityBalanceEquality:name ~(\ref{SolidityBalanceEquality})
		&
		\usevalue SolidityBalanceEquality:severity 
		&
		Adversary can manipulate contract logic by forcibly sending it ether.
		Use non-strict inequality on balances
		\\
		\hline
		%% SOLIDITY_UNCHECKED_CALL
		\rowcolor{lightlightgray}
		\usevalue SolidityUncheckedCall:name ~(\ref{SolidityUncheckedCall})
		&
		\usevalue SolidityUncheckedCall:severity 
		&
		The return value is not checked.
		Always check return values of functions
		\\
		\hline
		%% SOLIDITY_DOS_WITH_THROW
		\rowcolor{lightlightgray}
		\usevalue SolidityDosWithThrow:name ~(\ref{SolidityDosWithThrow})
		&
		\usevalue SolidityDosWithThrow:severity 
		&
		Expect external calls to deliberately \texttt{throw}
		\\
		\hline
		%% SOLIDITY_SEND
		\rowcolor{lightlightgray}
		\usevalue SoliditySend:name ~(\ref{SoliditySend})
		&
		\usevalue SoliditySend:severity 
		&
		The return value of \texttt{send} should be checked.
		Use \texttt{transfer}, which is equivalent to \texttt{if~(!send())~throw;}
		\\
		\hline
		%% SOLIDITY_REENTRANCY_EXTERNAL_CALL
		\rowcolor{lightlightgray}
		\usevalue SolidityReentrancyExternalCall:name ~(\ref{SolidityReentrancyExternalCall})
		&
		\usevalue SolidityReentrancyExternalCall:severity 
		&
		External contracts should be called after all local state updates
		\\
		\hline
		%% SOLIDITY_MALICIOUS_LIBRARIES
		\rowcolor{lightlightgray}
		\usevalue SolidityMaliciousLibraries:name ~(\ref{SolidityMaliciousLibraries})
		&
		\usevalue SolidityMaliciousLibraries:severity 
		&
		Using external libraries may be dangerous.
		Avoid external code dependencies, audit all code that is part of the project
		\\
		\hline
		%% SOLIDITY_TX_ORIGIN
		\usevalue SolidityTxOrigin:name ~(\ref{SolidityTxOrigin})
		&
		\usevalue SolidityTxOrigin:severity 
		&
		A malicious contract can act on a user's behalf.
		Use \texttt{msg.sender} for authentication
		\\
		\hline
		%% SOLIDITY_CALL_VALUE
		\rowcolor{lightlightgray}
		\usevalue SolidityCallValue:name ~(\ref{SolidityCallValue})
		&
		\usevalue SolidityCallValue:severity 
		&
		\texttt{a.call.value()()} forwards all gas, allowing the callee to call back.
		Use \texttt{a.transfer()}: it only provides the callee with 2300~gas (insufficient for a callback)
		\\
		\hline
		%% SOLIDITY_INTEGER_DIVISION
		\rowcolor{lightlightgray}
		\usevalue SolidityIntegerDivision:name ~(\ref{SolidityIntegerDivision})
		&
		\usevalue SolidityIntegerDivision:severity 
		&
		The quotient is rounded down.
		Account for it, especially for ether and token amounts
		\\
		\hline
		%% SOLIDITY_LOCKED_MONEY
		\rowcolor{lightlightgray}
		\usevalue SolidityLockedMoney:name ~(\ref{SolidityLockedMoney})
		&
		\usevalue SolidityLockedMoney:severity 
		&
		The contract receives ether, but there is no way to withdraw it.
		Implement a withdraw function or reject payments
		\\
		\hline
		%% SOLIDITY_UNCHECKED_MATH
		\usevalue SolidityUncheckedMath:name ~(\ref{SolidityUncheckedMath})
		&
		\usevalue SolidityUncheckedMath:severity 
		&
		Without extra checks, integer over- and underflow is possible.
		Use \hbox{SafeMath}
		\\
		\hline
		%% SOLIDITY_TIMESTAMP_DEPENDENCE
		\rowcolor{lightlightgray}
		\usevalue SolidityTimestampDependence:name ~(\ref{SolidityTimestampDependence})
		&
		\usevalue SolidityTimestampDependence:severity 
		&
		Miners can alter timestamps.
		Make critical code independent of the environment
		\\
		\hline
		%% SOLIDITY_VAR
		\rowcolor{lightlightgray}
		\usevalue SolidityVar:name ~(\ref{SolidityVar})
		&
		\usevalue SolidityVar:severity 
		&
		Type inference choses the smallest integer type possible.
		Explicitly specify types
		\\
		\hline
		%% SOLIDITY_BYTES_BYTE
		\usevalue SolidityBytesByte:name ~(\ref{SolidityBytesByte})
		&
		\usevalue SolidityBytesByte:severity 
		&
		\texttt{byte[]} requires more than \texttt{bytes}
		\\
		\hline
		%% SOLIDITY_GAS_LIMIT_AND_LOOPS
		\rowcolor{lightlightgray}
		\usevalue SolidityGasLimitAndLoops:name ~(\ref{SolidityGasLimitAndLoops})
		&
		\usevalue SolidityGasLimitAndLoops:severity 
		&
		Expensive computation inside loops may exceed the block gas limit.
		Avoid loops with big or unknown number of steps
		\\
		\hline
		%% SOLIDITY_ERC20_API_VIOLATION
		\rowcolor{lightlightgray}
		\usevalue SolidityErc20ApiViolation:name ~(\ref{SolidityErc20ApiViolation})
		&
		\usevalue SolidityErc20ApiViolation:severity 
		&
		The contract \texttt{throw}s where the ERC20 standard expects a \texttt{bool}.
		Return \texttt{false} instead
		\\
		\hline
		%% SOLIDITY_PRAGMAS_VERSION
		\usevalue SolidityPragmasVersion:name ~(\ref{SolidityPragmasVersion})
		&
		\usevalue SolidityPragmasVersion:severity 
		&
		Contract compiles with future compiler versions.
		Specify the exact compiler version
		\\
		\hline
		%% SOLIDITY_PRIVATE_MODIFIER
		\rowcolor{lightlightgray}
		\usevalue SolidityPrivateModifier:name ~(\ref{SolidityPrivateModifier})
		&
		\usevalue SolidityPrivateModifier:severity 
		&
		The \texttt{private} modifier does not hide the variable's value, only prevents external contracts from editing it
		\\
		\hline
		%% SOLIDITY_REDUNDANT_FALLBACK_REJECT
		\usevalue SolidityRedundantFallbackReject:name ~(\ref{SolidityRedundantFallbackReject})
		&
		\usevalue SolidityRedundantFallbackReject:severity 
		&
		The payment rejection fallback is redundant.
		Remove the function to save space: payments are rejected automatically
		\\
		\hline
		%% SOLIDITY_STYLE_GUIDE_VIOLATION
		\usevalue SolidityStyleGuideViolation:name ~(\ref{SolidityStyleGuideViolation})
		&
		\usevalue SolidityStyleGuideViolation:severity 
		&
		Unfamiliar capitalization style causes confusion.
		Start function names with lowercase, events with uppercase
		\\
		\hline
		%% SOLIDITY_VISIBILITY
		\usevalue SolidityVisibility:name ~(\ref{SolidityVisibility})
		&
		\usevalue SolidityVisibility:severity 
		&
		Functions are \texttt{public} by default.
		Avoid ambiguity: explicitly declare visibility level
		\\
		\hline
	\end{tabular}
	\label{AllPatternsTable}
\end{table}


%%%% SECTION 3: SMARTCHECK %%%%
\section{SmartCheck architecture}

The two major approaches to code analysis are \textit{dynamic} analysis and \textit{static} analysis~\cite{Liu2012}.
Dynamic analysis runs the program, while static analysis considers the program code without running it.
Static analysis usually includes three stages:
\begin{enumerate}
	\item building an intermediate representation (IR), such as abstract syntax tree or three-address code, for a deeper analysis compared to analyzing text;
	\item enriching the IR with additional information~\cite{Woegerer2005} using algorithms such as control- and dataflow analysis (synonym, constant, and type propagation~\cite{Aho1986}), taint analysis~\cite{Tripp2009}, symbolic execution, abstract interpretation;
	\item vulnerability detection w.r.t.~a database of patterns, which define vulnerability criteria in IR terms.
\end{enumerate}
In this paper, we do not consider formal verification methods, as they require a rarely available formal specification of the contract's intended functionality.

SmartCheck is a static analysis tool implemented in Java.
It runs lexical and syntactical analysis on Solidity source code.
SmartCheck is implemented in Java.
It uses ANTLR~\cite{Parr} and a custom Solidity grammar to generate an XML parse tree~\cite{Aho1986} as an intermediate representation (IR).
The tool detects vulnerabilities by using XPath~\cite{Liu2009} queries on the IR\@.
Thus, SmartCheck provides full coverage: the analyzed code is fully translated to the IR, and all its elements can be reached with XPath matching.
Line numbers are stored as XML attributes and help localize findings in the source code.
IR attributes can be enriched with additional information as new analysis methods are implemented.
The tool can be extended to support other smart contact languages by adding the corresponding ANTLR grammar and a pattern database.
The IR-level algorithms remain unchanged.

As an example, consider the Balance equality issue~(\ref{SolidityBalanceEquality}).
We aim to detect constructions that test the contract balance for equality, for instance:

\begin{minipage}{\linewidth}
\begin{lstlisting}[language=Solidity]
if (this.balance == 42 ether){...}.
\end{lstlisting}
\end{minipage}

The parse tree of this construction is shown in Figure~\ref{ParseTree}, and the corresponding XPath pattern is shown in Listing~\ref{XPathListing}.

\begin{figure}
	\caption{Parse tree for the Balance equality code example.}
	\Tree [.ifStatement 
		{\texttt{if}}
		[.ifCondition
			[.expression 
				[.expression [.envVarDef {\texttt{this.balance}} ] ]
				{\texttt{==}}
				[.expression [.moneyExpr [.primaryExpr [.numberLiteral {\texttt{42}} ] ] {\texttt{ether}} ] ]
			]
		]
		[.block { \{ } {\ldots} { \} } ]
	]
	\label{ParseTree}
 \end{figure}

\begin{minipage}{\linewidth} % https://tex.stackexchange.com/a/73305/142924
\begin{lstlisting}[caption={XPath pattern for the Balance equality issue.},label={XPathListing},language=XML]
//expression[expression//envVarDef
[matches(text()[1],"^this.balance$")]]
[matches(text()[1],"^==|!=$")]
\end{lstlisting}
\end{minipage}

In this case we do not expect false positives, as we are able to precisely describe the target construction in XPath.\footnote{Assuming that ANTLR builds the AST correctly based on the Solidity grammar.}
More complex rules can not be precisely described with XPath, which leads to false positives.
Consider the Re-entrancy issue~(\ref{SolidityReentrancyExternalCall}).
SmartCheck reports violations of the Checks-Effects-Interactions (CEI) pattern, which does not always lead to re-entrancy (Listing~\ref{ReentrancyListing}).

\begin{lstlisting}[caption={Violation of CEI not leading to re-entrancy.},label={ReentrancyListing},language=Solidity]
pragma solidity 0.4.19;
contract Foo {
	bool inBar = false;
	function bar(address someAddress) {
		if (inBar) throw;
		inBar = true;
		someAddress.transfer(0);
		inBar = false;
	}
}
\end{lstlisting}



%%%% SECTION 4: RESULTS %%%%
\section{Experimental results} \label{SectionResults}

\subsection{Goals and definitions}

We compare SmartCheck with the results of manual audit (Section~\ref{ManualAudit}) and the three freely available vulnerability detection tools -- Oyente, Remix, and Securify (Section~\ref{MassiveTesting}).
\footnote{For Securify, we only consider partial results from the publicly available version of the tool.}

We define a true finding as an issue (detected by a tool with manual verification or manually) that is a bad practice and should be fixed from our viewpoint.
It may or may not be an exploitable vulnerability.
All issues found by the tools were manually labeled as either true positive (TP) or false positive (FP).
A false negative (FN) for each of the four tools (Oyente, Remix, Securify, and SmartCheck) is a true finding that was not detected by this tool.

For each tool, the false discovery rate (FDR) is the number of FPs for this tool divided by the number of all issues reported by this tool:

\[FDR = FP / (TP + FP)\]

False negative rate (FNR) is the number of FNs for this tool divided by the number of all true findings (found by any of the tools or manually):

\[FNR = FN / (TP + FN)\]


\subsection{Case studies} \label{ManualAudit}

We consider three contracts: Genesis ("the platform for the private trust management market"~\cite{Genesis}, source code ~\cite{GenesisGithub}, analyzed at commit \texttt{1ecf99d}), Hive ("the first crypto currency [sic] invoice financing platform"~\cite{Hive}, source code ~\cite{HiveGithub}, analyzed at commit \texttt{0d54699}), and Populous ("an online platform that matchmakes invoice sellers to invoice buyers hosted on the blockchain"~\cite{Populous}, source code ~\cite{PopulousGithub}, analyzed at commit \texttt{10de4ae}).
The FDR and FNR for each tool (in \% and in absolute numbers) are presented in Table~\ref{CaseStudyTable}.

\begin{table*}[t]
	\centering
	\caption{Tools results on the three projects and overall}
	\begin{tabular}{|l|l|r|r|r|r|}
		\hline
		\textbf{Project} &  & \textbf{Oyente} & \textbf{Remix} & \textbf{Securify} & \textbf{SmartCheck} \\
		\hline
		\multirow{3}{*}{Genesis Vision}
			& TP & $0$ & $0$ & $0$ & $7$ \\
			& FP & $6$ & $40$ & $19$ & $22$ \\
			& FN & $10$ & $10$ & $10$ & $3$ \\
			& FDR (\%) & $100$ & $100$ & $100$ & $75.86$ \\
			& FNR (\%) & $100$ & $100$ & $100$ & $30.00$ \\
		\hline
		\multirow{3}{*}{Hive}
			& TP & $0$ & $0$ & $0$ & $6$ \\
			& FP & $6$ & $11$ & $6$ & $7$ \\
			& FN & $22$ & $22$ & $22$ & $16$ \\
			& FDR (\%) & $100$ & $100$ & $100$ & $53.85$ \\
			& FNR (\%) & $100$ & $100$ & $100$ & $72.73$ \\
		\hline
		\multirow{3}{*}{Populous}
			& TP & $0$ & $4$ & $0$ & $14$ \\
			& FP & $7$ & $60$ & $45$ & $31$ \\
			& FN & $19$ & $15$ & $19$ & $5$ \\
			& FDR (\%) & $100$ & $93.75$ & $100$ & $68.89$ \\
			& FNR (\%) & $100$ & $78.95$ & $100$ & $26.32$ \\
		\hline
		\multirow{3}{*}{Overall}
			& TP & $0$ & $4$ & $0$ & $27$ \\
			& FP & $19$ & $111$ & $70$ & $60$ \\
			& FN & $51$ & $47$ & $51$ & $24$ \\
			& FDR (\%) & $100$ & $96.52$ & $100$ & $68.97$ \\
			& FNR (\%) & $100$ & $92.16$ & $100$ & $47.06$ \\
		\hline
	\end{tabular}
	\label{CaseStudyTable}
\end{table*}

Oyente and Securify did not show any TPs on these three contracts.
Remix detected TPs only in the Populous contract.
Remix and SmartCheck showed an overall FDR of $97\%$~and $69\%$~respectively, and an overall FNR of $92\%$~and $47\%$~respectively.
This means that SmartCheck showed better FDR and FNR compared to its closest competitor.
Overall, SmartCheck reported $87$~issues in the three contracts.

Requirements for code analysis tools differ across platforms and domains.
Due to a special security requirements in smart contract programming, low FN rate is crucial (a missed vulnerability can be disastrous), whereas a relatively high FP rate is tolerable.
Most contracts contain only a few hundreds of lines of code (see~Section~\ref{MassiveTesting}) and can be audited manually.

Though SmartCheck's FDR of $69\%$~may seem pretty high, it is not a serious issue in this domain.
$47\%$~is a reasonable level of FNR, since many vulnerabilities in smart contracts are related to business logic and can not be detected automatically.
Most of SmartCheck's FNs were found manually (not by other tools).

SmartCheck detected a critical issue in one of the contracts: an attacker could create an unlimited number of internal entities and block the normal operation of the contract.
A public function (i.e.,~such that any Ethereum user can call it) allowed to add an element to an internal array (Listing~\ref{AddingGroup}).
Several critical functions then iterated through this array (e.g.,~Listing~\ref{Iterating}).
An attacker could make those functions permanently fail, as the function call would require more gas than the block gas limit.

\begin{minipage}{\linewidth} 
\begin{lstlisting}[caption={Adding an element to the internal array},label={AddingGroup},language=Solidity]
function createGroup(string _name, uint _goal)
    onlyOpenAuction
    returns (uint8 err, uint groupIndex)
{
	if(checkDeadline() == false && _goal >= fundingGoal && _goal <= invoiceAmount) {
		groupIndex = groups.length++;
		groups[groupIndex].groupIndex = groupIndex;
		groups[groupIndex].name = _name;
		groups[groupIndex].goal = _goal;

		EventGroupCreated(groupIndex, _name, _goal);

		return (0, groupIndex);
	} else {
		return (1, 0);
	}
}
\end{lstlisting}
\end{minipage}

\begin{minipage}{\linewidth}
\begin{lstlisting}[caption={Iterating through the internal array},label={Iterating},language=Solidity]
function findBidder(bytes32 bidderId) constant returns (uint8 err, uint groupIndex, uint bidderIndex) {
	for(groupIndex = 0; groupIndex < groups.length; groupIndex++) {
		for(bidderIndex = 0; bidderIndex < groups[groupIndex].bidders.length; bidderIndex++) {
			if (Utils.equal(groups[groupIndex].bidders[bidderIndex].bidderId, bidderId) == true) {
				return (0, groupIndex, bidderIndex);
			}
		}
	}
	return (1, 0, 0);
}
\end{lstlisting}
\end{minipage}

\subsection{Testing on a massive sample} \label{MassiveTesting}

A \textit{blockchain explorer} is a website that displays information about blockchain transactions.
Etherscan~\cite{EtherscanVerified} is a popular Ethereum blockchain explorer.
Among other information, it offers \textit{contract verification} as a service.
A contract developer would upload the source code to Etherscan, which would then confirm that the deployed bytecode was indeed obtained from the provided source code.
We downloaded the source code of $4\,600$~verified contracts ($1\,537\,954$~lines of code) from Etherscan as of 4~October~2017 using a Java library JSoup~\cite{JSoup}.
We then ran SmartCheck on this dataset.

% https://tex.stackexchange.com/a/99834/142924
% $X$ causes errors
\begin{figure}
	\caption{Distribution of non-zero contract balances (ether)}
	\centering
	\begin{tikzpicture}
	\pgfplotstableread{
		Balance   			Contracts
		100\,000+			8
		10\,000-100\,000    14
		1\,000-10\,000      38
		100-1\,000      	46
		10-100     			64
		1-10				90
		0-1					356
		%0					3984
	}\datatable
	
	\begin{axis}[
	xbar stacked,   % Stacked horizontal bars
	xmin=0,         % Start x-axis at 0
	ytick=data,     % Use as many tick labels as y coordinates
	yticklabels from table={\datatable}{Balance}
	]
	\addplot [fill=black] table [x=Contracts, y expr=\coordindex] {\datatable};
	\end{axis}
	\end{tikzpicture}
	\label{BalancesFigure}
\end{figure}

The contract balances differ significantly (see Figure~\ref{BalancesFigure}).
The vast majority ($3\,984$,~or $86.6\%$)~of contracts have a zero balance.
One contract holds over one~million ether ($1\,500\,000$, or \$$440$~million at the time of testing), which accounts for~$38.4\%$~of the total balance of all contracts.
Contracts have from $1$~to $2\,525$~lines of code, with an average of $334$~lines and a median of $221$~lines.

\let\letcs\texapiletcs
\begin{figure}
	\caption{Findings on the big dataset \\ (excluding {\usevalue SolidityVisibility:name })}
	\centering
	\resizebox{0.8\textwidth}{!}{
	\begin{tikzpicture}[x={(.001,0)}, font=\large]
	\definecolor{darkgray}{RGB}{128,128,128}
	\definecolor{lightgray}{RGB}{232,232,232}
	\foreach  \l/\x/\c[count=\y] in {
		{\usevalue SolidityBytesByte:name }					/{\usevalue SolidityBytesByte:occur }				/lightgray,
		{\usevalue SolidityRedundantFallbackReject:name }	/{\usevalue SolidityRedundantFallbackReject:occur }	/lightgray, 
		{\usevalue SolidityLockedMoney:name }				/{\usevalue SolidityLockedMoney:occur }				/lightgray, 
		{\usevalue SolidityMaliciousLibraries:name }		/{\usevalue SolidityMaliciousLibraries:occur }		/lightgray, 
		{\usevalue SolidityErc20ApiViolation:name }			/{\usevalue SolidityErc20ApiViolation:occur }		/lightgray, 
		{\usevalue SolidityPrivateModifier:name }			/{\usevalue SolidityPrivateModifier:occur }			/lightgray, 
		{\usevalue SolidityStyleGuideViolation:name }		/{\usevalue SolidityStyleGuideViolation:occur }		/lightgray, 
		{\usevalue SolidityIntegerDivision:name }			/{\usevalue SolidityIntegerDivision:occur }			/lightgray, 
		{\usevalue SolidityPragmasVersion:name }			/{\usevalue SolidityPragmasVersion:occur }			/lightgray, 
		%{\usevalue SolidityUncheckedMath:name }			/{\usevalue SolidityUncheckedMath:occur }			/lightgray,
		%{\usevalue SolidityVisibility:name }				/{\usevalue SolidityVisibility:occur }				/lightgray, 
		{\usevalue SolidityBalanceEquality:name }			/{\usevalue SolidityBalanceEquality:occur }		/darkgray,
		{\usevalue SolidityTxOrigin:name }					/{\usevalue SolidityTxOrigin:occur }			/darkgray, 
		{\usevalue SolidityVar:name }						/{\usevalue SolidityVar:occur }					/darkgray,
		{\usevalue SolidityGasLimitAndLoops:name }			/{\usevalue SolidityGasLimitAndLoops:occur }	/darkgray, 
		{\usevalue SoliditySend:name }						/{\usevalue SoliditySend:occur }				/darkgray, 
		{\usevalue SolidityTimestampDependence:name }		/{\usevalue SolidityTimestampDependence:occur }	/darkgray, 
		{\usevalue SolidityDosWithThrow:name }				/{\usevalue SolidityDosWithThrow:occur }		/darkgray,
		{\usevalue SolidityCallValue:name }					/{\usevalue SolidityCallValue:occur }				/black,
		{\usevalue SolidityUncheckedCall:name }				/{\usevalue SolidityUncheckedCall:occur }			/black, 
		{\usevalue SolidityReentrancyExternalCall:name }	/{\usevalue SolidityReentrancyExternalCall:occur }	/black}
	{\node[left] at (0,\y) {\l};
		\fill[\c] (0,\y-.4) rectangle (\x,\y+.4);
		\node[right] at (\x, \y) {\x};}
	\draw (0,0) -- (8000,0);
	\foreach \x in {2000, 4000, ..., 8000}
	{\draw (\x,.3) -- (\x,0) node[below] {\x};}
	\draw (0,0) -- (0,19.4);
	\end{tikzpicture}}	
	\label{MassiveTestingFigure}
\end{figure}
\let\letcs\etoolboxletcs

SmartCheck analyzed the dataset in $7\,644$~seconds (approximately $2$~hours and $7$~minutes, or $437$~lines per second\footnote{Intel~Core~i5-4210M @ 2.60~GHz, 12~GB RAM, Windows~8.1 64~bit}).
As per SmartCheck, $99.9\%$~of contracts have issues, $63.2\%$~of contracts have critical vulnerabilities\footnote{The issues found by SmartCheck in the big dataset were not manually verified.}.
The findings are presented in Table~\ref{MassiveTestingTable} and Figure~\ref{MassiveTestingFigure} (colors denote severity levels: black -- high, dark gray -- medium, light gray -- low).
The most prevalent issue, \textbf{\let\letcs\texapiletcs \usevalue SolidityVisibility:name \let\letcs\etoolboxletcs} (detected~{\let\letcs\texapiletcs \usevalue SolidityVisibility:occur \let\letcs\etoolboxletcs} times, which accounts for $67.30\%$~of all findings), is excluded from the figure for clarity.

\let\letcs\texapiletcs
\begin{table}[t]
	\centering
	\caption{Code issues detected on a big dataset}
	\begin{tabular}{|c|l|r|r|}
		\hline
		\textbf{Severity} & \textbf{Pattern} & \textbf{Findings} & \textbf{\% of all} \\
		\hline
		\multirow{3}{*}{high} & {\usevalue SolidityReentrancyExternalCall:name } & {\usevalue SolidityReentrancyExternalCall:occur } & $3.329$ \\
		 & {\usevalue SolidityUncheckedCall:name } & {\usevalue SolidityUncheckedCall:occur } & $0.818$ \\
		 & {\usevalue SolidityCallValue:name } & {\usevalue SolidityCallValue:occur } & $0.228$ \\
		\hline
		\multirow{8}{*}{medium} & {\usevalue SolidityDosWithThrow:name } & {\usevalue SolidityDosWithThrow:occur } & $6.521$ \\
		& {\usevalue SolidityTimestampDependence:name } & {\usevalue SolidityTimestampDependence:occur } & $6.378$ \\
		& {\usevalue SoliditySend:name } & {\usevalue SoliditySend:occur } & $2.794$ \\
		& {\usevalue SolidityGasLimitAndLoops:name } & {\usevalue SolidityGasLimitAndLoops:occur } & $2.164$ \\
		& {\usevalue SolidityVar:name } & {\usevalue SolidityVar:occur } & $0.529$ \\
		& {\usevalue SolidityTxOrigin:name } & {\usevalue SolidityTxOrigin:occur } & $0.163$ \\
		& {\usevalue SolidityBalanceEquality:name } & {\usevalue SolidityBalanceEquality:occur } & $0.094$ \\
		\hline
		\multirow{10}{*}{low} & {\usevalue SolidityVisibility:name } & {\usevalue SolidityVisibility:occur } & $67.296$ \\& {\usevalue SolidityPragmasVersion:name } & {\usevalue SolidityPragmasVersion:occur } & $3.067$ \\
		& {\usevalue SolidityIntegerDivision:name } & {\usevalue SolidityIntegerDivision:occur } & $1.432$ \\
		& {\usevalue SolidityStyleGuideViolation:name } & {\usevalue SolidityStyleGuideViolation:occur } & $1.348$ \\
		& {\usevalue SolidityPrivateModifier:name } & {\usevalue SolidityPrivateModifier:occur } & $1.014$ \\
		& {\usevalue SolidityErc20ApiViolation:name } & {\usevalue SolidityErc20ApiViolation:occur } & $1.169$ \\
		& {\usevalue SolidityMaliciousLibraries:name } & {\usevalue SolidityMaliciousLibraries:occur } & $1.157$ \\
		& {\usevalue SolidityLockedMoney:name } & {\usevalue SolidityLockedMoney:occur } & $0.439$ \\
		& {\usevalue SolidityRedundantFallbackReject:name } & {\usevalue SolidityRedundantFallbackReject:occur } & $0.053$ \\
		& {\usevalue SolidityBytesByte:name } & {\usevalue SolidityBytesByte:occur } & $0.006$ \\
		\hline
	\end{tabular}
	\label{MassiveTestingTable}
\end{table}
\let\letcs\etoolboxletcs


%%%% SECTION 5: RELATED WORK %%%%
\section{Related work} \label{RelatedWork}

Multiple tools aim at improving the security and correctness of Ethereum smart contracts.
Static checks are built into the online Solidity compiler Remix~\cite{BrowserSolidity}.
Oyente~\cite{Luu2016} is a symbolic execution tool vulnerability detection in EVM bytecode.
Securify~\cite{Tsankov2018} analyzes Solidity source code as well as EVM bytecode. % not really sure what Securify does under the hood
\cite{Bhargavan2016} and \cite{Pettersson2016} propose writing Ethereum contracts in safer languages (F* and Idris respectively).
\cite{Hirai2017} describes existing attempts to formal verification of EVM bytecode as well as the EVM itself.
%\cite{DrYAnalyzer} and \cite{Luu16} use symbolic execution to analyze EVM bytecode.
\cite{Hildenbrandt2018} formally describes the full semantics of the EVM, providing the foundation for formal verification tools for EVM bytecode .


%%%% SECTION 6: CONCLUSION %%%%
\section{Conclusion and future work}
	
We provided a comprehensive overview and classification of code issues in Solidity -- the major high-level language for Ethereum smart contracts.
We implemented SmartCheck -- an efficient static analysis tool for Solidity, which offers significant improvements over existing alternatives.
We tested our tool on a massive set of real-world contracts and detected code issues in the vast majority of them.
The tool can be improved in multiple directions: improving the grammar\footnote{The currently used grammar failed to parse $0.16\%$~of lines in our dataset.}, making patterns more precise (e.g.,~the temporarily muted {\usevalue SolidityUncheckedMath:name }), adding new patterns, implementing more sophisticated static analysis methods, adding support for other languages.

Security is still an issue in blockchain development.
We hope that SmartCheck will help solve this major challenge by providing smart contract developers with fast and relevant feedback on potentially problematic source code patterns.
