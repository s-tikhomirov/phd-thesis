\chapter{Introduction to privacy in cryptocurrencies}

\label{Chapter02IntroP2P}

This part is dedicated to privacy in cryptocurrencies, with a focus on Bitcoin.
In this introductory Chapter, we provide an overview of the field.
We formulate the challenge of privacy in decentralized networks.
Additionally, we outline important differences in the designs of the networking layers for file-sharing networks, such as BitTorrent, and cryptocurrencies, such as Bitcoin.
This provides the necessary background for Chapter~\ref{Chapter03Clustering}, where we provide a novel networking-layer deanonymization technique for Bitcoin and a few alternative cryptocurrencies, and Chapter~\ref{Chapter04Wallets}, where we compare the privacy guarantees of mobile cryptocurrency wallets.


\section{Privacy in cryptocurrencies}

The central topic of this thesis is privacy.
Let us first explain why privacy should be considered a crucial property of digital money.

Privacy is important for a currency to provide fungibility.

\todo[inline]{Cashless society}

Bitcoin was, and still to some extent is, misleadingly referred to as an anonymous currency~\cite{Reid2011}.

Most early research on security and privacy of cryptocurrencies only considered Bitcoin as the dominant cryptocurrency at that time and was primarily focused on blockchain analysis~\cite{Meiklejohn2013, Ober2013, Ron2013}.
% 2011 - Reid - An analysis of anonymity in the bitcoin system 
% 2013 - Androulaki - Evaluating user privacy in bitcoin 
Reid et al.~\cite{Reid2011} and Androulaki et al.~\cite{Androulaki2013} provide an overview of privacy challenges in Bitcoin.
A popular mitigation, which does not require modifications to the Bitcoin protocol, is mixing.
A Bitcoin transaction spends a number of unspent transaction outputs (UTXO) as inputs and generates a number of new UTXOs.
Mixing allows users to create a joint transaction that combines all relevant inputs and outputs, making it harder for an adversary to track the flow of coins of a single user.
The major drawback of this scheme is that users must agree to co-sign the transaction using additional means of communication.
This process is unscalable without coordination by a trusted third party.
Bonneau et al.~\cite{Bonneau2014}~propose Mixcoin, a protocol to automate mixed payments in Bitcoin and similar cryptocurrencies which includes an accountability mechanism to expose theft.
Valenta et al.~\cite{Valenta2015} add a blind signature scheme to Mixcoin to prevent the operator from spying on users.
Alternative implementations of mixing protocols include CoinJoin~\cite{Maxwell2013} and CoinShuffle~\cite{Ruffing2014}.

Bitcoin provides a way to transact without any trusted intermediary, but its privacy guarantees are questionable.
Indeed, unlike traditional financial systems, Bitcoin addresses are not tied to any real-world identity at the protocol level, but this fact alone does not guarantee strong privacy.
Bitcoin transactions are broadcast through a peer-to-peer network in cleartext; after being verified by miners they are stored in a massively replicated shared database (the blockchain).
A common technique to improve privacy in Bitcoin is to use a fresh address for every transaction (generating addresses is only limited by the size of the 256-bit key space).
This piece of advice, often implemented in wallets, is no panacea, as the relationships between transactions can be inferred through blockchain analysis.

Most described attacks on the privacy of cryptocurrency transactions mostly employed some form of data analysis on the transaction graph.

\subsection{Network analysis}

% 2014 - Koshy - An Analysis of Anonymity in Bitcoin Using P2P Network Traffic 
Koshy et al.~\cite{Koshy2014} analyze Bitcoin's anonymity through the lens of P2P network properties.
They propose a technique for a global passive adversary to deanonymize users based on transaction propagation times.
The adversary aggregates network traffic into tuples containing the Bitcoin address, the first IP address to relay this transaction, and the transaction identifier.
For each transaction, the tuples are constructed for each input and output.
Each tuple is counted as a "vote" in favor of a hypothesis that a certain IP "owns" (i.e.,~possesses the private key of) a certain Bitcoin address.
While this paper provided valuable insights, it seems not to account for trickling / diffusion, which must have decreased the quality of the proposed deanonymization algorithm.

% 2014 - Biryukov - Deanonymisation of clients in Bitcoin P2P network
Biryukov et al.~\cite{Biryukov2014} describe the networking properties of Bitcoin and propose a multi-step attack for correlating Bitcoin clients' transactions with their IP addresses.
The attack proceeds as follows.
Firstly, the attacker prevents clients from using Tor by abusing the Bitcoin's anti-DoS mechanism: by sending invalid blocks or transactions through Tor it is possible to make Bitcoin servers temporarily ban all Tor exit nodes (see also~\cite{Biryukov2015}).
Next, the attacker establishes multiple connections to each of the servers and tracks which of them advertise an IP address of the victim client.
The intuition is that the client's \textit{entry nodes} will be the ones to advertise its IP address to the attacker (this is not guaranteed; the paper suggests ways to reduce noise in the resulting data).
After constructing a mapping of client IP addresses to sets of their entry nodes, the attacker listens to new transactions and correlates them with the victim client, if they are broadcast from that client's entry nodes.

% 2015 - Miller - Discovering bitcoin's public topology and influential nodes
Miller et al.~\cite{Miller2015} exploit some peculiarities in the update mechanism for a known address database (\texttt{addrMan}) in the Bitcoin reference implementation to infer the underlying graph structure.
Each Bitcoin node maintains a  database of IP addresses of peers it knows, along with corresponding timestamps intended to reflect the peer's "freshness".
Unintuitively, at the time of writing (2015), Bitcoin nodes only update timestamps for nodes they maintain outgoing connections with (at each message received).
For incoming connections, the peer preserves the first timestamp relayed along with the address.
The authors implement a tool that takes advantage of such rules to make quite an accurate guess of the topology of the Bitcoin network.
After an update of Bitcoin Core in March~2015, this technique is no longer feasible.

% 2016 - Neudecker - Timing Analysis for Inferring the Topology of the Bitcoin Peer-to-Peer Network
Neudecker et al.~\cite{Neudecker2016} propose a timing analysis attack to infer the network topology.
Their approach is different from the previous work (and similar to ours) in that it does not use any side-channels, but only the timing of transaction propagation.
The real-world validation in the Bitcoin network inferred network links at a substantial recall and precision.
The authors showed that an inappropriately parameterized trickling mechanism can actually reduce the resistance to traffic analysis compared to na{\"i}ve gossip (for the goal of learning the network topology).

% 2017 - Wang - Towards better understanding of Bitcoin unreachable nodes
Wang and Pustogarov~\cite{Wang2017} conduct a measurement study of Bitcoin to analyze the unreachable nodes (i.e.,~those behind NATs and firewalls) and report, among other findings, that a large share of Bitcoin transactions originate from only two mobile applications.

% 2017 - Fanti - Anonymity Properties of the Bitcoin P2P Network
Fanti et al.~\cite{Fanti2017} study the anonymity properties of trickling and diffusion.
Despite the motivation to change the Bitcoin's propagation mechanism from trickling to diffusion, as the study shows, this provided only a marginal privacy improvement.
The authors conclude that the key feature that enables deanonymization in both trickling and diffusion is an inherent symmetry: as messages spread through the network in a circular fashion, a global adversary can estimate where the center (i.e.,~the message source) is.

% 2017 - Venkatakrishnan - Dandelion: Redesigning the Bitcoin Network for Anonymity
% 2018 - Fanti - Dandelion++: Lightweight Cryptocurrency Networking with Formal Anonymity Guarantees
Dandelion~\cite{Venkatakrishnan2017} and its improvement Dandelion++~\cite{Fanti2018} are message propagation protocols for P2P networks designed to prevent deanonymization attacks.
Its key idea is introducing asymmetry: a message is first sent along a random path, and only then broadcast gossip-style.
Message propagation in Dandelion++\footnote{We focus on the latest, improved version of the protocol.} proceeds in two stages: the "stem" phase and the "fluff" stage.
In the stem phase, a new message is broadcast along a random path in the anonymity graph: an approximately regular random graph based on the same set of nodes as the regular P2P network.
In the fluff phase, the latest node to receive the message disperses it using the regular gossip-style broadcast.
The authors show that the protocol achieves much stronger anonymity than Bitcoin's current propagation mechanism, though at the cost of a several second propagation delay and additional sensitivity to DoS attacks at stem phase. 

Though the authors mention (Section~4.2) that some configurations of the protocol may be prone to transaction  correlation attacks, our approach is not suitable against Dandelion++.
The key feature that allows our well-connected listening node to gather useful information is that nodes choose neighbors to propagate messages at random, without distinguishing incoming and outgoing connections.
This means that by saturating 50\% of a node's connection slots we have a 50\% chance to be the first to receive a new transaction from it.
In Dandelion++, nodes choose neighbors for the stem phase propagation only from outgoing connections.
There is no obvious way to force a remote peer to initiate a connection to us, therefore a malicious node with many outgoing connections will not have any advantage in the stem phase (it can only aggregate incoming information while acting as a regular relay, which may gain some but not much insight into possible transaction clusters).

% 2017 - Neudecker, Hartenstein - Could Network Information Facilitate Address Clustering in Bitcoin?
Neudecker and Hartenstein~\cite{Neudecker2017} combine blockchain and network analysis to cluster Bitcoin addresses and associate them with IP addresses.
They determine the originator of a transaction as the first originator, using two independent listening nodes and some heuristics to make the estimation more precise.
The authors conclude that for the majority of users network-based deanonymization is not a concern, though a small percentage of users might be susceptible to attacks of this type.

Gervais et al.~\cite{Gervais2014} analyze the privacy implications of Bloom filters in SPV wallets.






\section{Design goals of a cryptocurrency P2P network}

BitTorrent remains the primary example of a computer network which no one was able to fully stop.
Despite multiple lawsuits against large torrent trackers, such as the Pirate Bay, as well as against regular users, law enforcers were unable to stop BitTorrent.
One may argue that this is impossible in principle: BitTorrent is just a protocol, in other words -- a language.
As long as the specifications are widely available, and there are two computers in the world willing to communicate according to its rules, BitTorrent is not dead.

This resiliency makes it similar to Bitcoin.
However, Bitcoin is not \textit{only} a set of rules -- it is a protocol that maintains specific \textit{state}.
While BitTorrent is agnostic to the format and semantics of the date being shared, Bitcoin's P2P layer is intended to share a concrete dataset -- the blockchain -- specific to the system at large.
This in the reason behind important distinctions between the design goals of a file-sharing networks (exemplified by BitTorrent for concreteness) and a cryptocurrency, exemplified by Bitcoin.


\subsection{Blockchain as a common dataset}

BitTorrent has no global state.
Users wishing to download a file join a \textit{swarm} of users who have (at least parts of) this file and have no interest in how other files are shared.

The goal of Bitcoin's P2P network is to share the single global state of the system.
The series of confirmed blocks contains the single source of truth on which key controls how many bitcoin.
Each user is interested in obtaining this dataset.
If Bob is expecting Alice to send him a payment for goods or services, he wants to make sure that she is not cheating.
To ensure that the received coins are genuine, Bob must check that they originate from a transaction in a confirmed block, and that transaction links back to another confirmed transactions, and so on, until the coinbase transactions that brought the coin to life by rewarding a miner.
Bob can only do so if he has the full dataset of all transactions that happened since the launch of the system.
(It is possible to get weaker assurances by performing simplified verification, but we omit these details for clarity.)

Sharing one global state means that content addressing is not a problem in Bitcoin.
There is no question about who has the required file, because every (full) node has it.

BitTorrent file-sharing is bounded in time: a user downloads a file, and leaves the network.
In contrast, cryptocurrency P2P networks perform constant synchronization, as new transactions and block are being generated.
Moreover, mining and non-mining nodes have different requirements regarding synchronization.

\subsection{Logical properties of blockchain data}

BitTorrent is agnostic to the semantics or format of the files being shared.
In Bitcoin, the data has a logical order: each block depends on the previous block.
A Bitcoin node (assuming the most conservative threat model) should download the blockchain sequentially, verifying every block.

The task of blockchain synchronization can be divided into two tasks: the initial blockchain download (IBD) and the ongoing synchronization.
These tasks have different requirements for two types of users: miners and non-mining nodes.

\subsubsection*{Miners and non-miners}
Satoshi Nakamoto initially envisioned a network where every node could participate in mining, or coin generation.
Early versions of Bitcoin reference implementation included an option to generate coins.
But the economic forces quickly lead to mining specialization, and since 2013 mining is a highly specialized business requiring large capital investment~\cite{Kroll2013}.
This lead to two informal protocol roles emerging.
Despite not having any special privileges encoded in the protocol, mining nodes have other requirements to the P2P protocol.
We will outline the differences in the next paragraph with respect to blocks and other message types.
 
\subsubsection*{Syncing blocks}
A node performs IBD when it reconnects to the network after a period of absence, or when connecting for the first time.
This task is relatively similar to the BitTorrent setting, in that there is a finite dataset to pull from a set of peers.
In BitTorrent, files are split into chunks, which are downloaded in parallel from different peers.
The order in which the chunks are downloaded is not fixed (though in some clients a user can prioritize first chunks of a video file to start watching it while later chunks are still being fetched).
In Bitcoin, on the other hand, blocks have a logical structure.
Downloading later blocks without ensuring the validity of earlier blocks is a risk: if an early blocks turns out to be invalid, all blocks dependent on it are invalid as well.
Therefore, in the most basic implementation, a Bitcoin node downloads and validates blocks sequentially.
This process has been optimized in 2015 (Bitcoin~Core~10.0.0) with \textit{headers-first} download~\cite{Core2015}.

Mining and non-mining nodes have different requirements on the speed of block synchronization.
Miners want to ensure that they get new blocks as fast as possible, and that the block they mine propagate to other miners quickly.
Mining on top of an outdated block, or letting one's block get orphaned by another simultaneously mined block, leads to direct financial losses\footnote{There is a counter-intuitive strategy of not immediately broadcasting a newly found block (selfish mining~\cite{Eyal2018}), put even selfish miners presumably want be technically able to broadcast and fetch blocks as quickly as possible to keep their options open.}.
This lead to optimizations of block propagation (such as \textit{compact blocks}~\cite{Core2016}) and the emergence of dedicated physical lines for very fast block dissemination among miners~\cite{FALCON, FIBRE}.
Non-mining nodes generally can tolerate a latency of a few seconds (much smaller than an average block interval of ten~minutes), so they use the regular Bitcoin P2P network.
	
\subsubsection*{Exchanging transactions and other messages}
Apart from blocks, the Bitcoin P2P protocol disseminates other types of messages, such as unconfirmed transactions and address messages.
In contrast to confirmed blocks, there is no canonical set of unconfirmed transactions.
Each node has its own version of this set (the \textit{mempool}).
Other protocol messages also do not need to be disseminated among all nodes very quickly.

Despite the fact that most wallets show an incoming transaction as soon as it is broadcast in the network, from the protocol point of view unconfirmed transactions are not secure.
Therefore, strictly speaking, non-mining nodes are only interested in receiving blocks, but not individual unconfirmed transactions.
They are, however, interested in broadcasting their transactions to miners quickly to get them mined.
Miners, on the other hand, need to receive unconfirmed transactions quickly to be able to include them in blocks and earn the associated fees.



\subsection{Incentives and privacy}

While many BitTorrent seeders are profit-driven~\cite{Rumin2010}, their rewards do not come from the protocol directly.
While the P2P protocol accounts for the actual bandwidth and assigns a higher internal weight to peers with fast and reliable connections, such in-protocol points are hard to convert into more conventional value systems.
Some torrent trackers employ reputation systems, where the tracker gives additional privileges to active seeders, but torrent seeders have been largely altruistic~\cite{Rehn2004}.
In file-sharing, nodes seldom have incentives to lie.
As a user knows the hash of the content they search for in advance, it is impossible to violate the integrity of the file\footnote{It is possible to distribute malicious content (i.e.,~via \textit{poisoned} torrents~\cite{Lou2006}) but its harmful properties manifest itself outside of the file-sharing protocol.}.

Bitcoin is powered by economic incentives.
Nevertheless, sharing blockchain data is not incentivized directly.
Nodes get no reward for propagating data to their peers.

We distinguish three requirements for a blockchain P2P layer based on tasks a user may wish to perform.
Namely, a user may wish to read the blockchain, write to the blockchain (i.e.,~send transactions), and prevent others from observing these interactions (privacy).

\subsubsection*{Reading the blockchain}
In cryptocurrencies, propagating false data may be a part of an attack.
An adversary eclipsing a victim with nodes under their control may make them believe that a different state of the blockchain is valid, compared to what the larger world is agreeing upon.

Bitcoin's proof-of-work makes it difficult for an adversary to feed a fake blockchain to a user: malicious blocks must also contain a valid proof-of-work that is expensive to generate.
Eclipsing can facilitate other attacks, including ones on privacy.
To prevent this, Bitcoin and other cryptocurrencies deliberately choose neighbors randomly from a large set of possibly live nodes, and additional checks ensure network diversity among one's neighbors, ensuring that not too many neighbors are from the same subnet, as evident from their IP addresses.
Compare this to file-sharing, which optimizes for an opposite outcome: for instance re-trackers allow users to connect to peers in their immediate physical proximity, achieving faster downloads by using their ISP's internal infrastructure instead of the global Internet~\cite{Yoshida2012,Wang2012}.

\subsubsection*{Writing to the blockchain}
Arguably the most important quality that Bitcoin and other decentralized cryptocurrencies provide is censorship resistance.
In contrast to centralized banking, where clients are often subject to arbitrary account freezes or are not allowed to open an account in the first place, it is (or at least should be) very hard to prevent a cryptocurrency user from spending their coins.
In the most extreme scenario, censorship becomes a form of theft: despite having the private keys, a user would essentially lose coins.
Luckily, complete censorship is hard to perform\footnote{Assuming a non-custodial wallet, i.e.,~the user holding their own private keys.}.
Even if a user's node is eclipsed, they can sign a transaction and broadcast it without directly talking to the Bitcoin P2P network, using a third-party web service such as~\cite{Blockstream} (which can be accessed through Tor).

\subsubsection*{Privacy}
The lack of privacy in BitTorrent enabled tracking its users.
Though mass surveillance and prosecution of users engaged in file-sharing is technically possible, there are not too much incentives to do so.
The major reason why the share of P2P file-sharing traffic on the Internet has fallen sharply since early 2000s was the emergence of centralized streaming services that ultimately provided a better user experience legally and for a reasonable price\footnote{However, the proliferation of streaming services and the fragmentation of content between them has lead to a rise in BitTorrent usage in late 2010s~\cite{Bode2018}.} rather than law enforcement.
This trend suggests that privacy is not of crucial importance for P2P file-sharing.

For money, however, privacy is a crucial property.
In addition to ethical opposition to surveillance enabled by centralized digital payment systems, the requirement for privacy is motivated by the fact that a good currency has fungibility.
Each monetary unit should be valued the same as any other.
If this is not the case, a currency fails to implement its major function as a unit of account, as economic agents waste resources applying discounts for their incoming payments based the origin of concrete coins involved.
The ultimate way to enable fungibility is to make coins technically indistinguishable.
Bitcoin does a relatively poor job at that, as the transaction graph is public and saved in the blockchain.
The fact that Bitcoin addresses are not linked to real-world identities and can be generated anew for every transaction provides only a weak protection.
This design decision was motivated by a simple observation that, in the absence of a trusted third party, "[t]he only way to confirm the absence of a transaction is to be aware of all transactions"~\cite{Nakamoto2008}.
Novel cryptographic algorithms (zero-knowledge proofs) have been developed since that might allow to alleviate this limitation, but they remain experimental and unlikely to be incorporated in Bitcoin in the foreseeable future (while other cryptocurrencies, such as Zcash~\cite{BenSasson2014}, take advantage of them).

Keeping all these considerations in mind, in the next Section we outline the design of Bitcoin's P2P networking protocol.



\section{Our contributions}
\label{sec:C02S4OurContributions}

In the next Chapter, we present our contributions towards better understanding of privacy of Bitcoin and other cryptocurrencies on the networking layer.
\todo{add Refs to chapters}

First, we explore the scenario of a global passive adversary.
We tackle the following question: can an adversary infer relationships between transactions based solely on their propagation pattern in the network?
We review the Bitcoin's P2P protocol and propose an algorithm that allows us to distinguish transactions issued from the same node from transactions issued from different nodes (with s certain degree of accuracy).
We implement and test our method on Bitcoin and three other cryptocurrencies focused on privacy: Dash, Monero, and Zcash.
We also test the same technique on transactions issued from mobile wallets, which are studied in more detail in the subsequent chapter.

Then, we study mobile wallets.
With smartphones vastly outnumbering desktops and laptops and being the primary or only computing device for hundreds of millions of people.\todo{add cite}
The security of cryptocurrency wallets is therefore of high importance.
We systematize various privacy-related characteristics of mobile applications that are relevant for cryptocurrency wallets, and compare a selection of \missing{XX} wallets for Android -- the most prevalent mobile OS.
Our results show that the majority of wallets do not follow all security and privacy guidelines.

