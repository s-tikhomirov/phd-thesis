\chapter{P2P protocols in cryptocurrencies}

\label{Chapter02IntroP2P}

Cryptocurrencies use P2P networks to disseminate data.
peers exchange transactions and other protocol messages.

The design goals of a cryptocurrency P2P protocol differ from that of a file-sharing network.
First, a cryptocurrency network is optimized for the specific dataset (the blockchain).
Second, content addressing becomes less important: all peers host the same data.
Third, unlike sharing a particular file, blockchain synchronization is ongoing forever.
Finally, resilience, privacy, and censorship resistance play a more significant role.
This leads to other optimization trade-offs.
File-sharing networks are optimized to prioritize local traffic~\cite{Yoshida2012,Wang2012}.
Cryptocurrencies, on the contrary, try to guarantee diversity of connections to prevent denial-of-service.

In this introductory Chapter, we briefly describe the P2P protocol of Bitcoin and a few selected cryptocurrencies.
We refer the reader to~\cite{BitcoinWiki, Garay2015} for a more comprehensive documentation.
We also outline the relevant privacy attacks that our contributions are based upon.


\section{Bitcoin P2P protocol}

\subsection{Connections}

Bitcoin, as any P2P network, faces the bootstrapping problem.
A new peer does not know the IP addresses of other peers.
Bitcoin solves this issue with \textit{DNS bootstrapping}.
Bitcoin~Core contains eight hard-coded DNS records.
Well-known Bitcoin developers maintain them.
The records resolve into IP addresses of stable Bitcoin peers.
A new peer connects to a random subset of those.

Bitcoin peers establish unencrypted TCP connections.
\footnote{Different networks use different default ports: 8333 for Bitcoin, 18333 for Bitcoin testnet, 8233 for Zcash, 18080 for Monero, 9999 for Dash.}
A peer tries to maintain $8$~outgoing connections and may also allow up to $117$~incoming connections.
\footnote{The parameters can be changed in the configuration file of using command line arguments.}


\subsection{Peer discovery}

Peers exchange information about other peers.
A new peer advertises its IP address to its neighbors in an \texttt{ADDR} message.
Upon receiving an \texttt{ADDR} message, a peer may decide to relay it to some of its neighbors.
Peers advertise their IP addresses approximately every $24$~hours.

A peer can ask its neighbor which peers it is aware of.
This is done with a \texttt{GETADDR} message.
The neighbor responds with an \texttt{ADDR} message containing up to $1000$~addresses of recently seen peers.

After DNS bootstrapping, a new peer asks the initial peers about other peers and connects to those.
It then disconnects from the bootstrapping peers to keep them available for new joining peers.
Each peer maintains a persistent databased of IP addresses of peers it knows.
Ideally, this should suffice for all subsequent connections to the network.
DNS bootstrapping remains available as a fallback mechanism.

\subsection{Initial block download}

After connecting to the network, a new peer downloads and validates all previous blocks.
This process is known as the \textit{initial block download} (IBD).
A peer may decide to delete most of the block data from disk after validation (\textit{pruning}).
It may also perform \textit{indexing}, allowing fast querying of transactions in the local database.


\subsection{Propagation of transactions and blocks}

Bitcoin peer exchange data about objects (blocks or transactions) in a three-step protocol.
A sending peer first announces that it knows about a new object with an inventory (\texttt{INV}) message.
\texttt{INV}s uniquely identify objects by their hash.
They can be sent unsolicited or as a reply to a request.

Upon receiving an \texttt{INV}, a peer may reply with \texttt{GETDATA} to receive the full data.
The data is sent in a \texttt{BLOCK} or \texttt{TX} message for blocks and transactions, respectively.

Miners have special requirements for the P2P protocol.
They need to quickly receive new transactions to include them in a block, and propagate new blocks to other miners.
Since 2016, Bitcoin uses an optimized block propagation protocol (\textit{compact blocks}~\cite{Core2016}).
Miners use dedicated networks~\cite{FALCON, FIBRE} for fast block dissemination.


\subsection{Broadcast randomization}

Privacy is important for a cryptocurrency P2P network.
A flood-based message propagation can harm privacy.
An adversary listening to the network can infer which peers was the first to send a new transaction.
To make such attacks more difficult, Bitcoin peers randomize transaction propagation.

A peer announces a transaction after a random delay unique for each neighbor.
This mechanism is called \textit{diffusion}.
It replaced another randomization technique known as \textit{trickling}.
With trickling, a peer announces a transaction to a random subset of neighbors.
Such subsets are chosen once in a fixed-length time period.
Diffusion replaced trickling in 2015~\cite{Wuille}.
\cite{Fanti2017} showed that both randomization techniques provide only a marginal privacy improvement.
%The authors conclude that the key feature that enables deanonymization is an inherent symmetry of message propagation.
%This enables a global adversary to estimate the "rumor source".

\subsection{P2P protocols in alternative cryptocurrencies}

Alternative cryptocurrencies use various P2P networking protocols.
Some of them are based on a modified Bitcoin~Core database.
Examples include Dash and Zcash.
They inherit and possibly modify Bitcoin's P2P protocol.
Others, such as Ethereum and Monero, are implemented from scratch.


\section{Network-level privacy in cryptocurrencies}

An adversary can perform various attacks on the P2P level.

In an eclipse attack, an adversary fully controls the information flow between the victim and the network.
This allows the adversary to censor transactions and facilitates double-spending.

The adversary can also try to learn the \textit{network topology} -- which pairs of nodes are connected.
As P2P networks are formed in an ad hoc manner, the topology is fluid.
The adversary can estimate the topology from traffic patterns and use this information in further attacks.
Multiple topology estimation algorithms have been described~\cite{Miller2015, Neudecker2016, Wang2017}.

Finally, network-level data can leak sensitive information about transactions.
Different nodes receive the announcement of the same transaction at slightly different time.
This may allow an adversary to infer which node has initially sent it.

The first deanonymization attack based on transaction propagation timing was introduced in~\cite{Koshy2014}.
A global passive adversary analyzes the tuples containing the Bitcoin address, the first IP address to relay a transaction, and the transaction identifier.
Each tuple would be counted as a "vote" in favor of a hypothesis that a certain IP "owns" (i.e.,~possesses the private key of) a certain Bitcoin address.

A similar attack was described in ~\cite{Biryukov2014}.
An attacker established multiple \textit{parallel} connections to other peers.
This improves the amount if information it can capture.
The attacker can also prevent the victim from using Tor -- an anonymization overlay network~\cite{Biryukov2015}.
Our contribution in Chapter~\ref{Chapter03Clustering} builds upon this work.

\cite{Neudecker2017} combines transaction graph analysis and network analysis to cluster Bitcoin addresses and associate them with IP addresses.
The attacker determines the transaction originator using the first originator heuristic with additional heuristics to improve the estimation.
%The authors conclude that for the majority of users network-based deanonymization is not a concern, though a small percentage of users might be susceptible to attacks of this type.

Network-level attacks on privacy-focused cryptocurrencies have also been described~\cite{Quesnelle2017, Biryukov2019d, Bogatyy2019, Tramer2020}.
