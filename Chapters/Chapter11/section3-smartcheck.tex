\section{Automated analysis of smart contracts}

\subsection{Approaches to code analysis}

Dynamic code analysis runs the program and considers only a subset of all execution paths on some input data~\cite{Liu2012}.
Static code analysis may guarantee full coverage without executing the program and may run fast enough on code of reasonable size.
Static analysis usually includes three stages:
\begin{enumerate}
	\item building an intermediate representation (IR), such as abstract syntax tree or three-address code, for a deeper analysis compared to analyzing text;
	\item enriching the IR with additional information~\cite{Woegerer2005} using algorithms such as control- and dataflow analysis (synonym, constant, and type propagation~\cite{Aho1986}), taint analysis~\cite{Tripp2009}, symbolic execution, abstract interpretation;
	\item vulnerability detection w.r.t. a database of patterns, which define vulnerability criteria in IR terms.
\end{enumerate}
In this paper, we do not consider formal verification methods, as they require a rarely available formal specification of the contract's intended functionality.

	
\subsection{SmartCheck}

We propose SmartCheck -- a static analysis tool for Ethereum smart contracts implemented in Java.
SmartCheck runs lexical and syntactical analysis on Solidity source code.
It uses ANTLR~\cite{Parr} and a custom Solidity grammar to generate an XML parse tree~\cite{Aho1986} as an intermediate representation (IR).
We detected vulnerability patterns by using XPath~\cite{Liu2009} queries on the IR.
Thus SmartCheck provides full coverage: the analyzed code is fully translated to the IR, and all its elements can be reached with XPath matching.
Line numbers are stored as XML attributes and help localize findings in source code.
IR attributes can be enriched with additional information when new analysis methods are implemented.
The tool can be extended to support other smart contact languages by adding an ANTLR grammar and a pattern database (IR-level algorithms remain unchanged).

As an example, consider the Balance equality issue~(\ref{SolidityBalanceEquality}).
We aim to detect constructions that test the contract's balance for equality, for instance:

\begin{lstlisting}[language=Solidity]
if (this.balance == 42 ether){...}.
\end{lstlisting}

The parse tree of this construction is shown in Figure~\ref{ParseTree}.

\begin{figure}
	\caption{Parse tree for the Balance equality code example}
	\Tree [.ifStatement 
		{\texttt{if}}
		[.ifCondition
			[.expression 
				[.expression [.envVarDef {\texttt{this.balance}} ] ]
				{\texttt{==}}
				[.expression [.moneyExpr [.primaryExpr [.numberLiteral {\texttt{42}} ] ] {\texttt{ether}} ] ]
			]
		]
		[.block { \{ } {\ldots} { \} } ]
	]
	\label{ParseTree}
 \end{figure}

The corresponding XPath pattern is shown in Listing~\ref{XPathListing}.

\begin{minipage}{\linewidth} % https://tex.stackexchange.com/a/73305/142924
\begin{lstlisting}[caption={XPath pattern for the Balance equality issue},label={XPathListing},language=XML]
//expression[expression//envVarDef
[matches(text()[1],"^this.balance$")]]
[matches(text()[1],"^==|!=$")]
\end{lstlisting}
\end{minipage}

In this case we do not expect false positives, as we are able to precisely describe the target construction in XPath\footnote{Assuming that ANTLR builds the AST correctly based on the Solidity grammar.}.

More complex rules can not be precisely described with XPath, which leads to false positives.
Consider the Re-entrancy issue~(\ref{SolidityReentrancyExternalCall}).
SmartCheck reports violations of the Checks-Effects-Interactions (CEI) pattern, which does not always lead to re-entrancy (Listing~\ref{ReentrancyListing}).

\begin{lstlisting}[caption={Violation of CEI not leading to re-entrancy},label={ReentrancyListing},language=Solidity]
pragma solidity 0.4.19;
contract Foo {
	bool inBar = false;
	function bar(address someAddress) {
		if (inBar) throw;
		inBar = true;
		someAddress.transfer(0);
		inBar = false;
	}
}
\end{lstlisting}

%TODO: add this info info in the table
%Table~\ref{AllPatternsTable} shows which of the issue under consideration lead to false positives.

