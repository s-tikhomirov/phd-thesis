\section{Introduction}

Ethereum was introduced in 2014 and launched in 2015~\cite{Buterin2014}.
Ethereum nodes store data, execute smart contracts, and maintain a shared view of the global state using a proof-of-work consensus mechanism similar to that in Bitcoin~\cite{Tikhomirov2017}.
Contrary to previous attempts at blockchain programming, e.g.,~Bitcoin scripting, Ethereum language is Turing complete and thus able to express arbitrarily complex logic.

Developers write contracts in high-level languages (the most popular and mature one is Solidity) and compile them to bytecode of the Ethereum virtual machine (EVM) -- a stack-based VM operating on 256-bit words\footnote{It is also possible to write contracts in bytecode directly.}.
Compared to general purpose VMs like the Java virtual machine, EVM is relatively simple, executes deterministically, and natively supports certain cryptographic primitives~\cite{Buterin2017}.
A contract is deployed by broadcasting a transaction containing its bytecode and initialization parameters.
Miners include it in a block, permanently storing the contract at a unique blockchain address.
Users interact with the contract by broadcasting transactions with its address, the function to be called, and its arguments.
Upon request, the contract can call other contracts and send units of \textit{ether} -- the Ethereum native cryptocurrency -- to users or other contracts.

To make spamming costly, the Ethereum protocol specifies a cost (denominated in \textit{gas} units) for each EVM operation~\cite{Wood2014}.
A user pays upfront for the expected amount of gas the computation will consume and gets a partial refund after a successful execution.
If an exception (including ``out of gas``) occurs, all state changes are reverted, but the gas may not be refunded\footnote{Gas refunds depend on the exception type: \texttt{assert} consumes all gas, \texttt{require} does not (starting from the Byzantium release in October~2017).}.
The ether price of a gas unit is determined by the market.

Ethereum allows people and companies globally to programmatically encode and trustlessly enforce complex financial agreements.
This opens up new business models and constitutes a dramatic change in the digital economy.
New software development tools are required to ensure correctness and security of smart contracts.


\subsection{Security challenges in Ethereum} \label{SecurityChallenges}

Security is a primary concern in Ethereum programming for multiple reasons:
\begin{itemize}
	\item \textbf{Unfamiliar execution environment}.
	Ethereum differs from centrally managed execution environments, be that mobile, desktop, or cloud.
	Developers are not used to their code being executed by a global network of anonymous, mutually distrusting, profit-driven nodes.
	\item \textbf{New software stack}.
	The Ethereum stack (the Solidity compiler, the EVM, the consensus layer, etc) is under development, with security vulnerabilities still being discovered~\cite{chriseth2017}.
	\item \textbf{Very limited ability to patch contracts}.
	A deployed contract can not be patched\footnote{Though workarounds exist, such as proxy contracts redirecting calls to an adaptable address of the latest version of the main contract.}.
	This makes a popular ``move fast and break things`` motto inapplicable: a contract must be correct before deployment.
	\item \textbf{Anonymous financially motivated attackers}.
	Compared to many cybercrimes, exploiting smart contracts offers higher gains (the prices of cryptocurrencies have been increasing rapidly), easier cashing out (ether and tokens are instantly tradable), and lower risk of punishment due to anonymity.
	\item \textbf{Rapid pace of development}.
	Blockchain companies strive to release their products fast, often at the expense of security.
	\item \textbf{Suboptimal high-level language}.
	Some argue that Solidity itself inclines programmers towards unsafe development practices~\cite{ydtm2016}.
\end{itemize}

A textbook example of an Ethereum contract exploit is the DAO hack.
The DAO was an Ethereum-based venture capital fund.
In May~2016, it collected around \$150 million in the largest crowdfunding campaign to date.
In June~2016, an unknown hacker exploited multiple vulnerabilities in the DAO code and gained control over ether worth around \$50 million at that time~\cite{Sirer2016}.
Though the Ethereum protocol executed correctly, the core developers proposed a hard fork to restore stakeholders' deposits, violating the premise of decentralized applications running ``exactly as programmed``\footnote{Concerns about Ethereum's governance lead to the creation of Ethereum~Classic~\cite{EthereumClassic} -- a continuation of the Ethereum blockchain without the DAO fork.}.
More recent examples of high-profile loss of ether due to software vulnerabilities include two incidents with the Parity multi-signature wallet in July and November~2017~\cite{Palladino2017}.
These and many similar events of a smaller scale illustrate the importance of security in Ethereum.

For the purposes of this paper, we assume correctness of the Ethereum core infrastructure and focus on security from a contract developer's viewpoint.
We classify issues in Solidity source code and develop a static analysis tool -- SmartCheck -- that detects them.
We test SmartCheck on a large set of real-world contracts and measure the relative prevalence of various code issues.
SmartCheck shows significant improvements over existing alternatives in terms of false discovery rate (FDR) and false negative rate (FNR).