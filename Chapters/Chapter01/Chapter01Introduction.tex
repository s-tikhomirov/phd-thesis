\chapter{Introduction}

\label{Chapter01Introduction}

\section{Foreword}

This thesis explores the properties of cryptocurrencies -- a new type of digital money.


Money is a social rather than technological phenomenon.
Economists give various definitions of money and have different theories of its evolution. 
However, for the purposes of this work, which is mostly technology-focused, let is define money as a \textit{means to represent value}.

An important feature of a civilization is the ability to \textit{transfer value} across space and time.
One example could be a merchant buying and selling goods in different geographical locations.
Some goods have a limited shelf life, other require special conditions to store.
It is thus desirable to the merchant to store some \textit{representation} of value that would be recognized by fellow merchants in another town.
Another example is \textit{savings}.
A worker who owns more than they are willing to immediately spend wishes to transfer the value across time into the future.
It should be expected that the mechanism used for such transfer would not deteriorate.

Value is subjective.
The willingness of someone else to accept my payment in the future for goods or services that I would need depends not only on the physical properties of money, but also on the unpredictable future circumstances.
After all, the decision to accept or reject a trade is made by an individual, whose thinking is influences by a myriad factors, not all of them rational.
Nevertheless, some types of physical objects serve better as money than others.
Gold is arguable the longest living type of good acknowledged as money for millennia.
Its key properties -- divisibility, recognizability, and most importantly \textit{scarcity} -- has made it relatively reliable \textit{store of value}.

However, its physical properties make handling it directly a burden.
It is heavy; its storage must be protected against theft; and probably most importantly for the modern economy -- it is not transmittable over distance.

Taking advantage of the rapid development of communication technologies, most commerce today happens in the form of electronic signals.
This very efficient: liquid markets allow even small inefficiencies to be immediately arbitraged away.
However, the fundamental question is: what does actually these electronic signals \textit{represent}?

As the world has moved from precious metals to \textit{fiat} (i.e.,~government issued) money, one key property has been lost.
Gold is a \textit{bearer asset}: it represents value \textit{itself} and is not anyone's liability.
Paper banknotes and coins are bearer assets as well, though their value can be diluted with money printing.
Contrary to this, securities represented as records in a database usually encode a \textit{promise} that someone \textit{will} provide something of value on request in the future.
Thus modern commerce, exchanging mostly electronically, is relying on \textit{trust} to a higher and higher extent.

Inherently, this is not a bad thing.
Societies with high levels of trust are more prosperous due to lower transaction costs.
Instead of hiring an army of lawyers and security guards, people trust each other, and it in most cases it works, the benefits outweigh the losses in rare cases it doesn't.
However, if the \textit{money} itself is relying upon trust, this puts very high power in the hands of the administrators of the system -- central banks.
And while in democracies people can to some extent rely on the election process and political checks and balances, authoritarian governments often abuse their power over money to re-distribute wealth towards their goals at the expense of the citizens.

Would not it be nice to invent a \textit{digital} system of money which would not be under control of any single entity, like a central bank?
Up until late 2000-s, this problem seemed unsolvable.
Electronic information is very easy to copy.
Therefore, whatever string of bits Alice gives to Bob, how can Bob be sure that Alice did not keep a copy and did not spend it elsewhere?
A natural solution is a trusted central parties that keep track of who owns how much and \textit{promises} to update users' balances honestly.

\todo[inline]{Merge}
Cryptocurrencies have emerged at the intersection of two multi-decade historical trends.

First, the world has moved off the gold standard into the fiat monetary system.
Starting form 1970-s, the US dollar, the global reserve currency, is no longer "backed" by anything.
The monetary policy is determined by central banks.
A few people regulate interest rates and decide whether or not to bail out companies during crises.

Second, the world has gone digital.
From the first mainframes and pre-Internet networks in 1950-s and 1960-s, to personal computers in 1970-s and 1980-s, to the World Wide Web in 1990-s, file-sharing networks in 2000-s, and, finally, smartphones and social networks in 2010-s, a large part of information exchange and, more importantly in the contexts of this work, financial transactions is conducted digitally.
Despite the seemingly borderless and apolitical nature of the Internet, financial transactions are an object of heavy regulations and scrutiny.
This fact raises concern in terms of availability and privacy of financial services, starting from the most basic one -- simply saving value.

Starting from 1980-s, shortly after public key cryptography was invented, a loose group of people known as \textit{cypherpunks} has been working on digital money designs.
The key goal was to avoid having to trust a third party.
While it proved possible to protect user's privacy to a certain extent (Chaumian e-cash), the hardest nut to crack was \textit{emission}.
Who decides how much money goes into circulation?
And as it does enter, who does it go to?
This fundamental problem prevented early e-cash systems from being widely deployed.

Note that the third party must be trusted to satisfy all three pillars of information security:

\begin{itemize}
	\item \textit{availability} -- that Alice can always pay Bob if she has enough money;
	\item \textit{integrity} -- all balances are changes only through legitimate transactions, i.e.,~money are never arbitrarily printed "out of thin air";
	\item \textit{confidentiality} -- the details of a transaction are not revealed to other parties except for the counterparties.
\end{itemize}

Modern centralized payment systems not only violate these requirements \textit{often} -- it is at the core of their operations.
Digital payment providers are forced, among other things, to freeze accounts in case of "suspicious" transactions and implement international sanctions, making the system unavailable for citizens of certain countries.
The system's integrity is violated by the central banks themselves in selective bailouts and quantitative easing.
Finally, the business model of payment providers often involves analyzing transaction patterns of the users and selling this information to credit scoring agencies and advertisers, violating their privacy.

Bitcoin, announced in 2008 and launched in 2009, is the culmination of these multi-decade efforts.
The solution was proposed by a pseudonymous person (or a group of people) known as Satoshi Nakamoto.
Nakamoto ingeniously combined some of the well-known cryptographic primitives (signatures, hash functions, Merkle trees) in the world-first \textit{cryptoeconomic} system, i.e., a cryptographic system whose security guarantees are provided not only by mathematical properties but also by the \textit{economic incentives} of the involved parties.
Bitcoin has slowly gained traction first in the cypherpunk circles, then in a broader technology community.

As of 2020, Bitcoin has entered the public consciousness, but it can yet hardly be called a widely accepted means of payment.
All bitcoins in circulation are worth \missing{XXX} USD.

What is more relevant for this work, Bitcoin has introduces a host of novel problems at the intersection of computer science and economics.
This thesis is an attempt to tackle some of those problems, focusing on privacy and security.

In the remainder of this introductory Chapter, we give a more elaborate introduction to the field.
We outline the history of digital moneys, discuss filesharing networks (an important precursor to Bitcoin), and list some of the challenges Bitcoin faces and some of the alternative cryptocurrencies proposed to address them.
Finally, we list our original contributions and put them into context.


\section{Historical overview}

Bitcoin, despite having appeared outside academia, is deeply rooted in academic research in cryptography and distributed systems.
Narayanan and Clark~\cite{Narayanan2017} track the "Bitcoin's academic pedigree" and outline the ideas that Nakamoto cleverly combined, which include verifiable logs, digital cash, proof of work, and Byzantine fault tolerance.
The innovation of Bitcoin was not in inventing a novel technology, but rather in combining an array of technologies from different fields of computer science and merging them into a single system.

To put our work into historical context, we overview two important classes of technological precursors to Bitcoin.

The first one, naturally, is digital currencies.
Multiple researchers and entrepreneurs tried developing such systems.
Centralized systems, dependent on a known entity to operate, eventually got either shut down (the best-known examples being Liberty~Reserve, Liberty~Dollar and e-gold~\cite{White2014, Trautman2014}) or incorporated into the current system (PayPal) as soon as they become big enough to be reckoned with.


The second one are peer-to-peer file-sharing networks.
They served as an example of an Internet protocol which governments and large commercial entities were unable to shut down.
This feature is crucial for cryptocurrencies, but the design goals for these systems differ substantially.


\subsection{Early digital currencies}

We now briefly discuss notable attempts at creating \textit{independent} digital currencies or payment systems.
\footnote{Centralized digital payments, such as internet banking apps, are a part of the traditional banking system and are thus outside the scope of this work.}

\subsubsection*{Chaumian ecash}

David Chaum introduced an anonymous digital cash protocol built with blind signatures in 1982~\cite{Chaum1982} and enhanced in 1988~\cite{Chaum1988}.
The protocol allowed a central party -- a bank -- to issue digital coins not tied to users' identities.
Such notes could be exchanges among users anonymously.
Compared to Bitcoin, Chaumian ecash lacks the mechanism of preventing double-spending (in Bitcoin this problem is solved with proof-of-work).
Instead, upon receiving a transaction from Alice, Bob would consult the bank to check that the Alice's coin is genuine.
If Alice is trying to spend the coin twice, the bank detects it with a high probability, but Alice's identity remains hidden.
In 1989, Chaum founded Digicash, a company aimed at developing and commercializing his \textit{ecash} protocol.
The company has started to get traction in mid-1990s\footnote{Interestingly, the authorities of the Netherlands considered contracting Digicash to implement an ecash-based payment system for road tolls, but the project did not go through~\cite{Chaum2019}.}, but eventually declared bankruptcy in 1998.

\subsubsection*{B-money and Bitgold}

"More coherent approaches to treating puzzle solutions as cash are found in two essays that preceded bitcoin, describing ideas called b-money13 and bit gold42 respectively. These proposals offer"

B-money is an essay published by Wei Dai in 1998~\cite{Dai1998}.
It proposes a system of digital money to help implement the vision of crypto-anarchy described by Tim May~\cite{May1988}, where "violence is impossible because its participants cannot be linked to their true names or physical locations".
The proposal described a system very similar to Bitcoin: the participants are identified with public keys, every participant maintains the current account balance of all participants, transactions are broadcast as signed messages.
The outlined scheme for money emission also resembles Bitcoin mining: "Anyone can create money by broadcasting the solution to a previously unsolved computational problem. The only conditions are that it must be easy to determine how much computing effort it took to solve the problem and the solution must otherwise have no value, either practical or intellectual".
\footnote{Amazing how no one made the remaining small step to Bitcoin's eventual design for the next ten years after this document was written.}

Another similar proposal, Bitgold, was described by Nick Szabo in 2005~\cite{Szabo2005}.
Bitgold proposed digital coins represented with "a string of bits from a string of challenge bits" computed using a proof-of-work function.
The solutions to such puzzles were to be linked in a chain and timestamp using multiple timestamping servers.

\subsubsection*{Proof-of-work and Hashcash}

Proof-of-work was first proposed by Dwork and Naor in 1992 as an anti-spam mechanism~\cite{Dwork1992}.
The sender of an electronic message was expected to perform some computational work.
An electronic message would only be received if the sender attached a proof that some amount of computational work had been performed.
This would incur a negligible burden on regular users, but would become costly for spammers trying to send many messages quickly.
Crucially, this work depended on the content and the receiver of the message.
Otherwise, a spammer could re-use one solution for different recipients and messages.

Hashcash, a similar proposal, was described in 1997 by Adam Back~\cite{Back1997}.
It suggested using cryptographic hash functions as a building block for proof-of-work, utilizing the assumption that if a hash function behaves like a random oracle, the only way to discover partial collisions (i.e.,~find a value s.t. its hash is smaller than a given value) is by brute force search.
Hashcash was positioned as an alternative to Chaum's Digicash at the time, but for practical deployment it lacked a critical feature -- protection against double-spending.



\subsection{P2P filesharing networks}

An important milestone in the development of the Internet was the proliferation in the early 2000-s of peer-to-peer filesharing network, with BitTorrent being the primary example.
Satoshi Nakamoto, the creator of Bitcoin, wrote: "Governments are good at cutting off the heads of a centrally controlled networks like Napster, but pure P2P networks like Gnutella and Tor seem to be holding their own."~\cite{Nakamoto2008}.
BitTorrent resiliency remains an inspiration for cryptocurrency designers.
We now briefly review the major examples of noteworthy pre-Bitcoin P2P networks.


\subsubsection*{Napster and Gnutella}

The Internet has gained momentum in the developed world, with regular people and businesses exposed for the first time to the possibility of sharing data across the world with little to no pre-moderation.
Not surprisingly, the most common type of data people wanted to share media content such as movies and music.

File-sharing networks emerged in late 1990s to satisfy this demand, often by illegal means.
The two important early P2P networks are Napster and Gnutella~\cite{Saroiu2003}.
Napster -- the first successful music-centered file-sharing network -- was launched in 1999.
Having quickly gained millions of users, it drew attention from law enforcement, and was shut down in 2001.

An important point in the context of this thesis is that it was \textit{possible} to shut the network down.
Napster users were downloading content from each others' computers, but a central server was responsible for content search.
Without it, the network could not operate.

We can divide the task of file-sharing into two sub-tasks: locating a file and downloading it.
One may argue that as long as a user knows where the required file is located, it is only a technical matter to download it efficiently.
A more crucial task, which Napster failed to implement in a resilient fashion, is content discovery.

Using a server for this task introduces a central point of failure and renders the network vulnerable.
Gnutella, on the other hand, went to the other extreme and used a flooding technique.
A user would forward the query to its neighbors, each of which would either reply with the requested content, or forward the query further.
While this approach has no single point of failure, it is significantly less efficient.


\subsubsection*{BitTorrent}

BitTorrent~\cite{Pouwelse2005}, developed and launched in 2001, learned the lessons from both Napster and Gnutella.
Its design strikes a good balance between efficiency and resilience, without making user experience too unwelcoming.

As in other file-sharing P2P networks, users in BitTorrent download files from each other.
There are, however, multiple ways to locate the files.
One common way is \textit{torrent files}, which are distributed by specialized web servers -- \textit{trackers}.
A torrent file contains information about the file, its checksum, and addresses of some peers that likely host it.
This method does not make the whole network dependent on a single server, but a certain degree of centralization remains.
Alternatively, BitTorrent users may locate files with \textit{magnet links}, which utilize a modification of Kademlia~\cite{Maymounkov2002} -- a distributed hash table.

A distributed hash table (DHT) is a method of addressing content in a P2P network.
It randomly distributes content among nodes and allows for efficient querying.
It also requires only a minimal network restructuring when nodes leave or join.


\subsubsection*{On incentives and Sybil protection}

P2P networks face a dilemma.

In a file-sharing network, someone has to host the files.
If there is no central server, regular users must to this.
But regular users simply want to download a movie and watch it.
They do not want to take deliberate action "for the good of the network".
And as P2P network's value grows with the network effect, it is very important to make it simple for end users to use, but still make them contribute, maybe even without realizing it.

P2P file-sharing networks depended to a large extent on the users' goodwill.
They provided little monetary incentive for users to distribute files, let alone upload new content onto the network.
However, in contradiction to naive economical models, file-sharing networks gave birth to the \textit{warez scene} -- a community of people sharing content, mostly illegally, but with altruistic rather than profit-seeking motivations~\cite{Rehn2004}.
Regular users of file-sharing users were encouraged to share by trackers (in case of BitTorrent): for instance, by enforcing a ratio between what a user downloaded and uploaded.
However, such metrics are inherently vulnerable, as file-sharing network lack a strong identity system.
A tracker can force upload ratios for each account, but preventing users from signing up multiple times (a Sybil attack) requires either a centralized identity management, or a proof-of-work based solution.














\section{Bitcoin}

Bitcoin, introduced in 2008 and launched in 2009, is the first digital currency to solve the double spending problem without relying on a trusted third party.

\todo[inline]{Describe Bitcoin 101}


\section{Cryptocurrency dilemmas and challenges}

Satoshi Nakamoto had to decide on lots of parameters for Bitcoin.
Amazingly enough, this parameter combination proved to be good enough for Bitcoin to survive.
Still, many underline inefficiencies or drawbacks in the Bitcoin design any propose alternative systems occupying other points in this multi-dimensional design space.

Broadly speaking, a cryptocurrency enables two things: to \textit{validate} that the state of the system adheres to the \textit{rules}, and to come to consensus on a \textit{single} version of history out of multiple valid versions.
Let us refer to the first step as \textit{validation}, and to the second step as \textit{consensus}.
This frameworks leads to two questions to answer:
\begin{enumerate}
	\item How to ensure that all nodes can perform validation?
	\item How to ensure that the network as a whole comes to consensus? In Bitcoin and related cryptocurrencies, this is enabled by mining.
	\item How to decide whether or not and how the rules should be changed? This question is often referred to as \textit{governance}.
\end{enumerate}

Let us now outline some of the inherent trade-offs Bitcoin makes and how alternative cryptocurrencies make them differently.

\subsection{Mining objectivity vs "usefulness"}

Mining is probably the most counter-intuitive part of Bitcoin's design.
A concern that is often raised is that mining is "wasteful".
Indeed, arrays of specialized computers burning \missing{terawatts} of energy to solve a seemingly arbitrary mathematical equation which does not serve any purpose by itself looks wasteful.
However, the notion of "wasteful" is inherently subjective.
Clearly, Bitcoin provides value to some people (and, therefore, has a price).
This utility is achieved with mining as it currently exists.
The question of whether such expense is justified boils down to the question: does this expense justify the benefits?

\subsubsection*{Alternative PoW}
In any case, the question of increasing the "usefulness" of proof-of-work have attracted the attention of researchers and developer from the early days of cryptocurrency.
One of the first examples of a moderately popular (by 2013 standards) cryptocurrency was Primecoin.
It replaces the hash-based PoW problem with the problem of finding prime numbers (Cunningham and bi-twin chains).
While these numbers also have no immediate economical purpose, the rationale was that it helps advance mathematics as a science.
Indeed,the longest sequences of such numbers have been found by Primecoin miners.

However, there are two main objectives against using computational problems with "real-world value" for PoW.
First, the lack of any value outside the protocol makes economical analysis simpler.
If the puzzles for the miners to solve were to have some extrinsic value, they would have to make more difficult calculations to decide whether it is worth mining.
In an extreme case, if the price of the solution rises too much, they could forget about securing the network altogether and simply produce "solutions" for whoever is willing to buy them, hence destroying the initial purpose of securing the network through computational work.

Second, most real-world problems are not file-tuneable, which is a crucial feature of PoW.
\footnote{The authors of~\cite{Narayanan2016} refer to a similar property as \textit{puzzle-friendliness}.}
A PoW puzzle solves as a proxy for the amount of the amount of computation.
The system cannot directly measure the amount of energy a miner has committed, therefore the resulting puzzle solution must act as a good approximation.
For hash-based PoW puzzles we know in advance what is the probability of finding a solution for every attempt.
We assume (if the cryptographic assumptions hold) that solutions are randomly distributed in the search space.
This allows us to fine-tune the difficulty.
In contrast, in the real-world problems this may not be the case.
Consider the problem of protein folding, which had been used in volunteer computation projects (Folding@Home~\cite{Beberg2009}).
Despite a seeming similarity (looking for solutions in a large space by trial and error), it is not a good PoW function: we do not know how the probability of finding a solution per unit time changes with changes in computational power, therefore it is not clear how to parameterize the problem to make it $X$ times harder, for an arbitrary coefficient $X$.


\subsubsection*{Proof-of-stake and BFT-like consensus}

A more radical approach is to replace proof-of-work completely.
The general idea behind a family of designs known as \textit{proof-of-stake}, or PoS, is the following.
Let Alice have a certain amount of capital to invest into minting a cryptocurrency.
In a PoW system, Alice would buy a miner, and run it, burning electricity.
In a PoS system, Alice buys a "virtual miner" by buying the units of cryptocurrency and temporarily locking (\textit{staking}).
The protocol would then assign the right to generate a block (and to assign oneself a reward) to stakers proportionally to the amount of coins they stake.
Misbehaving miners can additionally be punished (slashed) by burning or re-distributing their stake.

Research in PoW alternatives also borrows ideas from decades of distributed systems research in Byzantine fault tolerant (BFT) systems, whereby a group of computers come to consensus under an assumption that under a third of them are malicious, and all participants are known beforehand.

Alternative consensus algorithms, including modifications and combinations of various flavors of BFT and proof-of-stake, is an active are of research (see~\cite{Bano2019} for an overview).
Modifying a proof-of-work protocol into proof-of-stake is not a clear task.
Novel attacks on PoS are being discovered~\cite{Fanti2019,Gazi2018,BrownCohen2019}, including potential game-theoretic incompatibilities with on-chain financial instruments~\cite{Chitra2020}.


\subsection{Mining efficiency vs decentralization}

Another angle of PoW critique is its proneness to centralization.
The main objective of cryptocurrencies is to build a system without a central point of failure.
However, the key security assumption in Bitcoin and similar systems is that the majority of miners are not colluding.
A coalition of more than half of the miners (weighted by hashing power) can re-write the blockchain and double-spend coins (the so-called \textit{51\% attack}).
In real circumstances, due to network latency, this threshold is even smaller.
Moreover, large miner coalitions, even smaller than 50\%, can enable game-theoretic attacks such as selfish mining~\cite{Eyal2018}.
Therefore, it is desirable for a cryptocurrency to keep the mining power under the control of a diverse group of participants.

Bitcoin uses a general-purpose cryptographic function SHA-256 for PoW.
\footnote{More precisely, double SHA-256.}
Evaluation of this function can be massively parallelized.
Note that the brute-force search that the miners perform is nearly perfectly parallelizable task: hashing of each candidate block header does not depend on other such calculations.
This has leas to a massive specialization, from CPUs to GPUs to FPGAs to ASICs.
This, in turn, has risen the importance of the economies of scale for mining operations, because it is cheaper to buy specialized devices in bulk.
Allocating a large amount of capital when starting a mining business helps negotiate electricity and rent prices.
This has lead to concentration of Bitcoin mining -- geographically and in terms of pools.

Some say that this is a natural outcome of a free market which cryptocurrency developers should not attempt to manipulate.
An alternative line of reasoning goes that a cryptocurrency can be made less prone to such effects by using a better suited hash function for PoW.
Such goal is commonly referred to as \textit{ASIC resistance}.
The key observation is that data processing is parallelized much more effectively than memory.
Hence a separate branch of cryptography develops purposefully \textit{memory-hard} hash functions.
\footnote{They are useful, among other things, for password hashing.}

Arguably the first alternative cryptocurrency to use a memory-hard hash function for ASIC resistance was Litecoin.
It uses scrypt instead of SHA-256 and is mineable on GPUs.
Most modern PoW cryptocurrencies use some form of a memory-hard hash function as a PoW (see Table~\ref{tab:pow-coins-hash-functions}).

\begin{table}[]
	\begin{tabular}{|l|l|}
		\hline
		\textbf{Cryptocurrency} & \textbf{Hash function} \\ \hline
		Bitcoin & SHA-256 \\ \hline
		Bitcoin Cash & SHA-256 \\ \hline
		Ethereum & Ethash \\ \hline
		Litecoin & scrypt \\ \hline
		Monero & CryptoNight \\ \hline
		Zcash & Equihash \\ \hline
	\end{tabular}
	\caption{Prominent PoW-based cryptocurrencies and their hash functions}
	\label{tab:pow-coins-hash-functions}
\end{table}

Using an ASIC resistant hash function introduces new challenges.
Bitcoin security in in part enabled by the fact that miners incur large capital investment into hardware which is only suited for one purpose -- calculating SHA-256 hashes.
This disincentivizes attacks: in case of a successful attacks the price of bitcoin is likely to drop, which decreases the price of the ASICs and makes the attack cost non-refundable.
ASIC-resistant hash functions, on the other hand, are usually calculated using GPUs.
GPUs are widely available and suitable for other purposes, mainly gaming and scientific computing.
This leads to two factors potentially decreasing the security of ASIC-resistant cryptocurrencies:
\begin{itemize}
	\item An attacker can sell the GPUs on the free market after the attack, at least partially recouping the initial investment;
	\item The attacker can \textit{rent} hash power only for the duration of the attack using hashrate markets such as Nicehash. \footnote{Some ASIC-resistant PoW-based cryptocurrencies have been 51\%-attackes in practice: Ethereum~Classic, Bitcoin~Gold.}
\end{itemize}
We should also note that even if we assume that ASIC-resistance is desirable, it may not be achievable in practice.
As the line of reasoning goes, no function is ASIC-resistant given enough economic incentives.
If a cryptocurrency gains massive traction and its price raises, hardware manufacturers have a strong incentive to invest in ASIC development.
Accounting for the fact that specialized devices can be made more efficient at calculating even a memory-hard hash function compared to consumer hardware (for example, by throwing away unnecessary peripherals, utilizing massive-scale cooling systems, etc), ASICs will be developed given enough incentives.
Indeed, this happened in practice with Ethereum~\cite{OLeary2018} and Zcash~\cite{Floyd2018}, despite them using memory-hard hash functions (however, their ASICs are not as much more efficient compared to regular GPUs than ASICs for Bitcoin's SHA-256).
Following from this, the only way to prevent ASIC development is to change the hash function unpredictably and more often than it takes to develop and manufacture an ASIC.
Such measure is implemented in Monero~\cite{Kim2019} (albeit with limited success) and is being considered in Ethereum (under the term Programmable PoW, or ProgPoW)~\cite{OLeary2019}.
However, such update requires coordination and a degree of trust in the core developers, manifesting a central point of failure.

Analyzing the feasibility and the security implications of using ASIC-resistant hashing in PoW-based cryptocurrencies remains an open research question.


\subsection{Validation vs scaling}

One of the key conditions of a cryptocurrency security model is that anyone can independently validate all transactions.
If it is not the case, a user would have to trust a third party to report transactions, which is what the whole design trying to avoid.

This makes scaling difficult.
If all transactions are validated by all nodes, the system as a whole can only process as many transactions as the slowest node.
There are three approaches to this problem.

Another approach is to extract as much efficiency as possible from the current architecture.
This may involve raising parameters such as block size limit and the frequency of blocks, which would mechanically raise the number of transactions per second a system processes.
This results in cutting off less powerful nodes from the ability to validate, with the assumption that decentralization would sufficient still to avoid censorship or other attacks.

One approach to address this is sharding.
The term is borrows from database design, where a database is split into parts (shards) where each transaction is only handled by one shard.
However the problem is not the same for a cryptocurrency: we still want all nodes, no matter which shard they belong to, to be able to independently validate all transactions.
Therefore the key sub-problem is \textit{cross-shard communication}: how do nodes in one shard be reasonably strongly convinced that all transactions in other shards are valid, without fully validating them?

Finally, the third approach is \textit{off-chain protocols}, also referred to as layer-two (L2).
Such protocols remove transactions off the blockchain completely, while preserving the possibility for the involved parties to leverage layer-one security to resolve dispute.
Bitcoin's Lightning Network is a primary example of an L2 protocol.
\footnote{Part~\ref{Part2Lightning} of this thesis is dedicated to privacy aspects of the Lightning Network.}


\subsection{Validation vs privacy}

For a node to be able to validate the blockchain state, it must \textit{obtain} this state.
While this statement is not entirely true accounting for zero-knowledge cryptography, at the time Bitcoin was developed it was true for all practical purposes.
All Bitcoin nodes maintain a copy of all transactions that have ever taken place.
This allows a node to independently verify the validity of an incoming transaction (or of any transaction).
However, this characteristic badly affects privacy.
As Alice pays to Bob, the fact of this transaction, including the amount, is stored forever in a massively replicated database.
This information can be extracted and analyzed years later.
Even if the users adhere to the recommendation to generate a new address for every transaction, linking transactions that belong to the same user is not hard using a few heuristics.

There exist multiple approaches at addressing this problem.
A few specialized cryptocurrencies have been developed with an emphasis on protecting privacy.
We will refer to them as \textit{privacy-preserving cryptocurrencies}.

\subsubsection*{Enhancing privacy in Bitcoin}

Multiple cryptographic techniques have been proposed to address the Bitcoin privacy problem.


\subsubsection*{Privacy-preserving cryptocurrencies}

Alternative cryptocurrencies aim to provide stronger privacy by using sophisticated cryptographic techniques to obfuscate transaction data.
\todo[inline]{Add more details + MW coins}
Dash relies on built-in background mixing powered by the so called masternode network.
Monero implements ring signatures and confidential transactions.
Zcash uses zero-knowledge proofs, namely, zk-SNARKs (though the majority of transactions do not take advantage of them due to heavy performance cost).
Zcash and Dash are based on a fork of the Bitcoin~Core codebase, while Monero is not.

Quesnelle~\cite{Quesnelle2017} proposes a method to link Zcash transaction based on a heuristic extracted from real-world usage of transparent and shielded addresses.
\todo[inline]{Add links to other attacks, incl MW networking}

The key question remain regarding privacy-preserving cryptocurrencies: is cryptography enough?
Anonymity can only be defined w.r.t. a certain \textit{anonymity set}.
That is, if only a few people use a privacy-preserving technology, the very fact of using it can endanger its users.
This is a common problem for anonymization software such a Tor.
Similarly, the final "degree of privacy" a cryptocurrency provides depends not only on the cryptographic mechanisms it uses, but also on its popularity.
Privacy-preserving cryptography such as zero-knowledge proofs can be computationally heavy, which discourages users, especially ones using mobile devices.
Integrating PPCs into an exchange may also be a challenge both technically and legally, while smaller trading volumes can make it not worth it economically.
At the same time, Bitcoin is being made more private, albeit its privacy limited by the fundamentals of the protocol.
The question remains: will more advanced cryptography in PPCs outweigh the handicap of a smaller anonymity set?
\todo[inline]{Introduce "open questions"?}


\subsection{Immutability vs security}

Once confirmed\footnote{The definition of "confirmed" in PoW-based cryptocurrencies is probabilistic.}, a cryptocurrency transaction can not be modified.
This makes canceling "incorrect" transactions impossible.
One line of reasoning states that this precisely \textit{is} the selling point: in cryptocurrency, a merchant can be sure that the payment will not be rolled back or frozen week and months after it is received, as it happens with credit card payments.
However, this presents a challenge in a programmable blockchains such as Ethereum, where the code of a smart contract, when deployed, would act as programmed, including the bugs.
\todo[inline]{Check numbers and facts}
The most well-known example of a smart contrast disaster cause by faulty code was \textit{The DAO} -- an investment fund implemented as a set ot Ethereum smart contracts.
It was launched in April~2016, and, surprisingly for its own creators, has raised 150~M USD in ether.
An unknown attacker exploited a subtle bug in the code and withdrew 60~M USD, causing controversy among the Ethereum community.
Eventually, the core developers implemented a hard fork -- a non backwards-compatible protocol modification -- to return the "stolen" funds.
The dissident minority continued to run the old chain, stating that manually interfering in smart contract code defeats its main purpose of executing "exactly as programmed".

This raised a question regarding best practices of contract development.
Should developers retain the right to modify the contract code (for example, by encoding a special "admin key" into the code)?
If so, what changes are they allowed to make?
How to ensure that the admin key is not stolen or used maliciously?
And if the developers relinquish their rights to modify the contract, how to make sure that the code contains no bugs?

Real-world systems usually combine the two approaches.
On the one hand, most popular Ethereum contracts contain some form of administrator access.
\todo[inline]{Add reference}
On the other hand, multiple tools for code analysis have emerged, leveraging insights from decades of secure software engineering and applying them to this new type of software.
\footnote{We contribute to this field in Chapter~\ref{Chapter10Findel} and Chapter~\ref{Chapter11SmartCheck}.}


\subsection{Consensus vs governance}

Finally, a separate question arises which is more of a political rather than technical nature.
Who decides how to modify the cryptocurrency itself?
Cryptocurrency nodes can validate the rules and choose one valid version of history from potentially multiple valid ones, but the rules are encoded in the software by (core) developers.
How do they decide what to include?
Moreover, as the developers cannot force node operators to run a certain version of software, what happens if the rules are modifies, but the majority of nodes does not follow them?

This question is more of a political rather than of a technical nature, therefore it is largely out of scope of this thesis.
However, we note that this is a crucial aspect of cryptocurrencies as cryptoeconomic, rather than purely cryptographic systems.
The question of who decides the rules can be divided into two: what are the goals of the system, and what means are best suited to achieve these goals?
Bitcoin is arguably in a unique position.
Its strategic goal -- to implement a digital currency -- was set at the beginning by Satoshi Nakamoto.
The discussion on the means to achieve this goals is largely limited by what is perceived as the most important "meta-rule": that rules must not change.
Bitcoin's development is mostly conservative.

Ethereum's development is more dynamic.
The network regularly goes through hard forks, which, despite formally being opt-in, are strongly motivated by the so-called \textit{difficulty bomb} -- a mechanism that slows down mining in the non-updated nodes after a set deadline.

Some alternative blockchain projects aim to implement \textit{on-chain governance} -- a mechanism by which token holders could vote on the future changed of the protocol, the primary example being Tezos.


\section{Outline of our contributions}

The rest of this thesis is structured as follows.

\begin{itemize}
	\item 
	In Part~\ref{Part1Privacy}, we focus on privacy of Bitcoin and alternative cryptocurrencies focused on privacy.
	\begin{itemize}
		\item
	In Chapter~\ref{Chapter02IntroP2P}, we provide an introduction to the privacy issues in cryptocurrencies, and also explain the differences in design goals for P2P networks for file-sharing networks and for cryptocurrencies.
		\item
	In Chapter~\ref{Chapter03Clustering}, we consider a security model where a global passive adversary is listening to the whole network and tries to infer relationships between transactions.
	We propose and implement an algorithm that links related transaction based on their propagation timings in the P2P network.
	We evaluate our approach on the four cryptocurrencies in question and infer that better network-level anonymization protocol are needed to fully protect users' privacy.
		\item
	In Chapter~\ref{Chapter04Wallets}, we study the landscape of mobile wallets for various cryptocurrencies.
	We show that most of them do not satisfy our minimal privacy criteria, and more often than not the wallet developers can spy on their users.
	\end{itemize}
	\item
	Part~\ref{Part2Lightning} is dedicated to security and privacy aspects of the \textit{Lightning Network} (LN).
	LN is a so-called \textit{layer-two} (L2) protocol build on top of Bitcoin to enable faster payments with higher granularity.
	L2 networks, such as LN, present a whole new field of research regarding security and privacy.
	
	\begin{itemize}
		\item 
	In Chapter~\ref{Chapter05IntroLightning}, we provide the historical context of layer-two protocols in Bitcoin, and payment channels in particular.
	We give a technical introduction to the architecture of the Lightning Network -- the main subject of our study in this Part.
		\item
	In Chapter~\ref{Chapter06LNprobing}, we show that the balances of LN users, which are considered private, are in fact easily obtainable using a \textit{probing} technique.
	We propose and implement an algorithm that takes advantage of error reporting in the LN and detects balances with a high accuracy.
	Our evaluation on the Bitcoin testnet shows that the whole network can be probed in hours.
		\item
	In Chapter~\ref{Chapter07LNattacks}, we aim to answer the question: how likely are some of the previously described attacks on the LN depending on various assumptions about the network and the attacker's capabilities?
	We perform a simulation based on a recent LN snapshot and estimate the probability of successful attacks for various parameter combinations.
	The key takeaway is that LN (as of 2019) is relatively concentrated: compromising a small share of influential nodes raises the attack success rates significantly.
		\item
	In Chapter~\ref{Chapter08HTLClimit}, we quantify a known limitation on how LN handles concurrent payments.
	As it turns out, due to limits imposed by the Bitcoin protocol, LN can handle only up to a certain number of payments concurrently.
	We quantify the effect of this limitation on the theoretical network throughput and and its evolution during the lifetime of LN.
	\end{itemize}
	\item
	Finally, Part~\ref{Part3Ethereum} is dedicated to Ethereum -- an alternative cryptocurrency with rich programming capabilities.
	Its built-in Turing complete virtual machine (EVM) allows developers to store programs on the blockchain, whose code is executed as a response to incoming transactions.
	Such programs are referred to as \textit{smart contracts}.
	
	\begin{itemize}
		\item 
	In Chapter~\ref{Chapter09Introcontracts}, we provide the necessary background on the Ethereum architecture and code analysis in general.
		\item
	In Chapter~\ref{Chapter10Findel}, we propose Findel -- a declarative language for financial contracts build on top of Ethereum's Solidity language.
	We argue that encoding financial agreements in a functional rather than imperative paradigm is beneficial from the security viewpoint.
		\item
	In Chapter~\ref{Chapter11SmartCheck}, we introduce a tool for static analysis of Ethereum smart contracts.
	We codify all common bugs in Solidity and propose an automated tool to detect those in Solidity source code.
	We evaluate SmartCheck on a large sample of real-world Ethereum contracts deployed on mainnet.
		\item
	Finally, in Chapter~\ref{Chapter12KYC}, we investigate the question of legal requirements.
	Most cryptocurrency exchanges are compelled to implement know-your-customer (KYC) regulations, which imply that all customers provide their private information to the service provider.
	This obviously harms the users' privacy.
	We are asking a question: is it possible to preserve users' privacy insofar it is compliant with KYC regulations?
	We propose a cryptographic scheme based on Ethereum towards this goal.
	
	\end{itemize}
\end{itemize}















