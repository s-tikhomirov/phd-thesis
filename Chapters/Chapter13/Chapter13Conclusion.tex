\chapter{Conclusion}

\label{Chapter13Conclusion}

We summarize the thesis and present an outlook into the future of blockchains, cryptocurrencies, and money in general.

\section{Future directions}

\subsection{Improving P2P privacy in cryptocurrencies}

Cryptocurrency developers should introduce privacy enhancing measures at the network level, especially if the currency is meant to be privacy-preserving.
As our results show, trickling and diffusion, as they are implemented in Bitcoin and its forks, are not sufficient.

\paragraph{Better P2P propagation in Bitcon}
As we have outlines, Dandelion~\cite{Venkatakrishnan2017, Fanti2018} may be a promising way to improve the privacy of Bitcoin and other cryptocurrencies on the networking level.
Dandelion has already been implemented and is currently used in a privacy-focused cryptocurrency Grin.
Its use, however, has been shown to have weaknesses.
\todo[inline]{Discuss what Dandelion does and doesn't; ref Bogatyy's post}

\subsubsection*{Improving our clustering technique}

\paragraph{The applicability of the external quality metric}
The adjusted anonymity degree, which we used as an external quality metric, has limitations.
In particular, we didn't account for transactions from clusters which did not also contain at least one of our own transactions.
The rationale behind this is the lack of the ground truth for two "foreign" transactions: we do not know whether they should be included in the same cluster.
Consequently, our quality metric may poorly reflect the reality on large networks (such as the Bitcoin mainnet), where our transactions make up only a small part of the full network throughput.
One direction of future research may be deriving an anonymity metric which works better under these circumstances.

\paragraph{Direct comparison of relay randomization techniques}
As described in Section~\ref{sec:Ch03Background}, cryptocurrencies use different relay randomization techniques aimed at improving privacy: trickling, diffusion, or no randomization.
A natural question would be to measure the relative effectiveness of these methods.
Unfortunately, we cannot use a direct comparison between cryptocurrencies that use diffusion and trickling to make a conclusion about relative effectiveness of these methods, as the real-world networks also differ in many other parameters (such as the number of nodes and transaction rate) that also influence the attack results.
A possible direction for future research may be to quantify the effects of trickling and diffusion on privacy properties of a Bitcoin-like cryptocurrency with respect to our attack technique, holding all other parameters equal.