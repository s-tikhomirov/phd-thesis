\chapter{P2P protocols in cryptocurrencies}

\label{Chapter02IntroP2P}

Each cryptocurrency relies on a P2P network to disseminate transactions and other messages.
Ensuring that all nodes reliably obtain the relevant data is a prerequisite for consensus.
Resilience, privacy, and performance are crucial design considerations for the P2P layer.

This introductory Chapter provides the background on the network-layer aspect of cryptocurrencies.
First, we outline the design goals for a cryptocurrency P2P network.
Then, we describe the technical details of the P2P protocol in Bitcoin\footnote{We refer the reader to~\cite{BitcoinWiki, Garay2015} for more comprehensive documentation.} and selected alternative cryptocurrencies.
Finally, we classify cryptocurrency nodes from a networking perspective and provide a review of network-level attacks on cryptocurrency privacy.


\section{Design goals for a cryptocurrency P2P network}

Unlike client-server protocols, all nodes in a P2P network have equal roles.
A cryptocurrency P2P protocol has two major tasks.
First, it provides new peers with the entire set of existing blocks.
Second, it continuously disseminates new blocks and transactions as they appear.

While adhering to the same general philosophy with file-sharing, discussed in~Section~\ref{sec:FileSharingNetworks}, cryptocurrency P2P protocols aim at slightly different goals.
First, a cryptocurrency P2P protocol is tailored to disseminating one specific dataset, whereas file-sharing is content-agnostic.
Moreover, each block in the blockchain depends on the previous one.
This logical structure leads to a trade-off: downloading blocks out of order in parallel is faster but may lead to bandwidth waste if some of them turn out to be invalid.

Second, content addressing is not as essential in cryptocurrencies as in file-sharing networks, where locating the peer that hosts the required file is the key challenge.
All cryptocurrency peers store the same set of blocks (except possibly a few latest blocks).

Third, cryptocurrencies demand stringent security guarantees.
Cryptocurrency P2P protocols prioritize resilience over efficiency to defend against denial-of-service, eclipse attacks, and network data analysis.
For instance, Bitcoin peers deliberately connect to peers from a diverse set of IP~regions and autonomous systems~\cite{Naumenko2019a}.
File-sharing protocols, on the contrary, prioritize local connections to utilize the ISP's local networks instead of the global Internet~\cite{Yoshida2012,Wang2012}.


\section{P2P protocols in cryptocurrencies}

Let us now describe the P2P protocol used in Bitcoin and an alternative P2P protocol for cryptocurrencies -- Dandelion.
Cryptocurrencies based on a modified Bitcoin~Core codebase, such as Dash and Zcash, inherit its P2P protocol.
Others, such as Ethereum, implement their own one~\cite{Henningsen2019}.

\subsection{Bitcoin P2P protocol}
\label{sec:BitcoinP2PProtocol}

We now describe the technical details of the P2P protocol in Bitcoin.

\paragraph{Bootstrapping}

Any P2P network faces the bootstrapping problem: a new peer does not know any other peers.
Bitcoin solves this issue in two ways.
Bitcoin~Core provides two bootstrapping methods: \textit{DNS bootstrapping} and \textit{seed nodes}.
The code contains hard-coded DNS records that resolve into IP addresses of stable peers.
If DNS bootstrapping fails, the new peer connects to some of the \textit{seed nodes} whose IP addresses are also hard-coded.


\paragraph{Connections and peer discovery}

Bitcoin peers establish unencrypted TCP connections.\footnote{Different networks use different default ports: 8333 for Bitcoin, 18333 for Bitcoin testnet, 8233 for Zcash, 18080 for Monero, 9999 for Dash.}
A peer tries to maintain $10$~outgoing connections: eight connections for relaying all types of messages plus two dedicated connections for block propagation~\cite{Daftuar2019}.
A peer allows up to $117$~incoming connections by default.

Peers exchange information about the IP addresses of known peers.
A new peer advertises its IP address to its neighbors in an \texttt{addr} message.
Upon receiving an \texttt{addr} message, a peer may decide to relay it to some of its neighbors.
A peer can ask its neighbor which peers it is aware of using a \texttt{getaddr} message.
The neighbor responds with an \texttt{addr} message containing up to $1\,000$~addresses of recently seen peers.

After establishing the initial connections, a new peer asks the bootstrapping peers about other peers and connects to those.
It then disconnects from the bootstrapping peers to keep them available for new joining peers.
Each peer maintains a persistent database of IP addresses of known peers.
Ideally, this database should suffice for all subsequent connections to the network.
DNS bootstrapping and seed nodes remain available as a fallback mechanism.


\paragraph{Initial block download}

After connecting to the network, a new peer downloads and validates all previous blocks.
This process is known as the \textit{initial block download} (IBD).
A peer may decide to delete most blocks after validating them (\textit{pruning}).
It may also enable \textit{indexing}, allowing for fast querying of transactions in the local database.

Bitcoin~Core supports two IBD modes: \textit{blocks-first} and \textit{headers-first}.
In blocks-first IBD, a peer downloads the blocks sequentially from a single peer.
This process ensures that the peer downloads the next block only if the previous block is valid.
Blocks-first IBD is slow and depends on a single peer.
Headers-first IBD mode, introduced in~2015~\cite{Core2015}, addresses these shortcomings.
A peer first downloads all block headers (using \texttt{getheaders} and \texttt{headers} messages) and ensures that they are well-formed and contain the necessary PoW.
The peer then downloads the full blocks from multiple peers in parallel.
Note that a peer cannot validate a block by only looking at its header.
A block may contain a valid header with sufficient PoW despite containing an invalid transaction.
However, headers-first IBD is beneficial in most piratical scenarios.


\paragraph{Propagation of transactions and blocks}

Bitcoin peers exchange blocks and transactions in a three-step message exchange.
A sending peer first announces the hash of a new object with an inventory (\texttt{inv}) message.
Upon receiving an \texttt{inv}, another peer may reply with \texttt{getdata} to receive the full data.
The sending peer replies with a \texttt{block} or a \texttt{tx} message for blocks and transactions, respectively.

Developers have introduced multiple improvements to Bitcoin's P2P protocol.
In the initial protocol, each transaction is propagated twice: first on its own, then as a part of a block.
An optimized block propagation protocol called \textit{compact blocks}~\cite{Core2016} eliminates this redundancy.
Peers share block \textit{sketches} that describe its contents using short transaction identifiers.
A sketch also contains the transactions that the receiving peer lacks (as \textit{predicted} by the sending peer).
The receiving peer reconstructs the block from the sketch and requests the missing transactions with additional \texttt{getblocktxn} queries if necessary.
Erlay~\cite{Naumenko2019} is another P2P optimization that reduces the number of redundant \texttt{inv} messages that peers exchange.
Miners have special requirements for the P2P protocol and use dedicated networks~\cite{FALCON, FIBRE} for fast block dissemination.


\paragraph{Broadcast randomization}

A gossip-based propagation in a P2P network may reveal private information such as the original message sender.
An attacker may analyze the timestamps of messages received from different peers.
Bitcoin uses \textit{broadcast randomization} to protect against network attacks.
A peer introduces a random delay before announcing a transaction to each neighbor.
This mechanism, called \textit{diffusion}, replaced~\cite{Wuille} another randomization technique called \textit{trickling}.
With trickling, a peer announces a transaction to a random subset of neighbors.
Such subsets are chosen once in a fixed-length interval.
Replacing trickling with diffusion provided only a modest privacy improvement~\cite{Fanti2017}.


\subsection{Dandelion}
\label{sec:Dandelion}

Dandelion~\cite{Venkatakrishnan2017, Fanti2018} is an alternative P2P protocol for cryptocurrencies designed for stronger privacy.
It addresses the key issue with gossip protocols -- the symmetry of message propagation.
Dandelion peers propagate a message in two stages: the \textit{stem phase} and the \textit{fluff phase}.
On the stem phase, a peer only relays a message to one randomly selected neighbor.
The peer that received the message randomly chooses whether to continue the stem phase (forward the message to one random neighbor) or start the fluff phase (relay the message to multiple neighbors).
Dandelion makes a network adversary confuse the original message sender with the originator of the fluff phase.
The authors show that the protocol achieves much stronger anonymity than Bitcoin's current P2P protocol.
The drawbacks of Dandelion include increased propagation delays and sensitivity to DoS attacks at the stem phase.
As of 2020, Dandelion is not introduced in Bitcoin but used in privacy-focused cryptocurrencies Monero, Grin, and Beam.


\section{Taxonomy of nodes}
\label{sec:TaxonomyOfNodes}

Full Bitcoin nodes download, validate, and share the whole blockchain.
These activities consume bandwidth, storage, and processing power.
Alternative types of nodes offer ways to use Bitcoin with more lenient requirements.

\paragraph{Pruned nodes}
The simplest protocol change to decrease storage requirements is to discard (\textit{prune}) old blocks.
A pruned node downloads and validates all blocks, and then deletes the old blocks.
Pruning reduces storage requirements significantly.
Bitcoin~Core allows allocating as little as $550$~MB for the most recent blocks (the full Bitcoin blockchain requires $300$~GB as of September~2020).
Pruned nodes do not support transaction indexing and cannot serve old blocks to others.

\paragraph{SPV nodes}
\textit{Simplified payment verification} (SPV) is another approach, which is outlined in the original Bitcoin paper~\cite{Nakamoto2008}.
Instead of downloading full blocks, an SPV node only asks for block headers and transactions of interest.
The peers respond with the requested transactions and a Merkle proof that they are included in the blocks.
SPV is based on weaker security assumptions than the full protocol.
Recall that a peer cannot verify a block based only on its header.
A malicious peer may provide a Merkle proof that a transaction is included in an invalid block with sufficient PoW.
SPV nodes must therefore ensure they are not eclipse-attacked and query block headers from multiple independent sources.

Moreover, SPV provides weaker privacy.
In the naive implementation, the full node learns the addresses that belong to the SPV node.
To mitigate this threat, modern SPV uses \textit{Bloom filters} -- probabilistic data structures that allow checking whether an element belongs to a set.
A Bloom filter may produce false positives, wrongly reporting that an element belongs to a set, but never produces false negatives.
If an element is in the set, the filter always indicates that.
An SPV node submits a Bloom filter to a full node to specify the addresses it is interested in.
The full node replies with the transactions involving all addresses that pass the filter.
The SPV node discards the false positives locally.
The privacy guarantees of Bloom filters have also been questioned~\cite{Gervais2014}.

\paragraph{Wallets with trusted remote nodes}
Finally, one may use Bitcoin without directly connecting to its P2P network.
This mode of operation, implemented by many mobile wallets, requires a \textit{trusted node}.
A wallet stores the keys and signs transactions locally but can only publish them through a trusted node.
The trusted node can lie about the blockchain state, deny service, learn what addresses a user controls, and link them to their IP address.
The user may gain partial protection by connecting to the trusted node through Tor -- an anonymization overlay network~\cite{Dingledine2004}.
However, the administrator of the trusted node can use other methods to associate user's transactions (e.g.,~by making the wallet send a cookie along with transactions).
On the other hand, it gets harder for a global attacker to distinguish users that broadcast transactions from the same trusted node.

%There is an inherent trade-off between wallets with centralized and P2P broadcast.
%Centralized wallets may better protect the user's privacy from external adversaries, but can themselves link users' transactions and correlate them with other information obtained from the app.
%Users must also trust centralized wallet providers for availability.

\pagebreak


\section{Network-level privacy in cryptocurrencies}

We now outline various types of network-level attacks on privacy of cryptocurrency users.

First, an adversary can perform a \textit{denial-of-service} attack by flooding the network with messages overwhelm honest nodes.
The Bitcoin's P2P protocol addresses this threat: peers do not re-broadcast the same message and ban peers that send invalid data or exhibit other unexpected behavior.

Second, in an \textit{eclipse attack}, an adversary takes control of all connections between the victim and the rest of the network.
Controlling the victim's view of the network allows the attacker to provide incorrect data, censor transactions, and collect network traffic for future analysis.
Eclipse attacks may have severe consequences for miners, as the attacker inhibits or slows down block propagation.
In layer-two protocols such as Lightning\footnote{Part~\ref{Part2Lightning} of this thesis is dedicated to the privacy and security of the Lightning network. Chapter~\ref{Chapter05IntroLightning} provides an introduction to Lightning and layer-two protocols.}, network-level attacks can lead to direct loss of funds~\cite{Riard2020}.

Third, in a \textit{network data analysis attack}, an adversary extracts information from the P2P messages and infers private information about the victim.
For instance, timing differences in transaction announcements from different peers may reveal the original sender.
An adversary can establish multiple connections and collect traffic from many vantage points for more accurate results.

Fourth, the adversary can infer the \textit{network topology}, i.e.,~which pairs of nodes are connected, and use this information in further attacks.
Multiple topology estimation attacks have been described.

One attack~\cite{Miller2015} exploits some peculiarities in the update mechanism for the address database (\texttt{addrMan}) in Bitcoin~Core.
Each Bitcoin peer maintains a database of known IP addresses, along with corresponding "freshness" timestamps.
Counterintuitively, Bitcoin peers only update timestamps for peers that they maintain outgoing connections with.\footnote{After an update of Bitcoin~Core in March~2015, the attack is no longer possible.}
For incoming connections, the peer preserves the first timestamp relayed along with the address.
The authors implement a tool that exploits these rules to estimate the network topology.

Another method~\cite{Neudecker2016} infers the network topology by analyzing transaction propagation timing.
The authors test the method on the real-world Bitcoin network with high recall and precision.
They also show that an inappropriately parameterized trickling mechanism can reduce the network's resilience against topology discovery.

Finally, a measurement study~\cite{Wang2017} of Bitcoin's \textit{unreachable} peers (i.e.,~peers behind NATs and firewalls) reports, among other findings, that a large share of Bitcoin transactions originate from only two mobile applications.


\paragraph{Network-based transaction deanonymization}

In \textit{transaction deanonymization} attacks, an adversary is trying to associate the victim's identities used inside and outside of the cryptocurrency protocol.
A more concrete task towards this goal is to reveal the victim's IP address, which is often linked to their geographical location and real-world identity.
A related task is transaction clustering, whereby an attacker reveals the hidden relationships between transactions.

The first work on network-based deanonymization attacks proposes a technique based on transaction propagation times~\cite{Koshy2014}.
The adversary aggregates inputs and outputs of transactions into tuples.
A tuple contains the Bitcoin address, the first IP address to relay the transaction, and the transaction identifier.
The method considers each tuple as a "vote" in favor of a hypothesis that an IP address "owns" (i.e.,~possesses the private key of) a Bitcoin address.

A similar attack~\cite{Biryukov2014} correlates Bitcoin transactions with IP addresses.
First, the attacker prevents the victim from using Tor by abusing the Bitcoin's anti-DoS mechanism~\cite{Biryukov2015}.
Next, the attacker establishes multiple \textit{parallel} connections to all nodes that accept incoming connections.
The intuition is that the peers that advertise the victim's IP address to the attacker are the victim's immediate neighbors (\textit{entry nodes}).
The attacker then maps IP addresses to sets of entry nodes.

Another attack~\cite{Neudecker2017} clusters Bitcoin addresses and associates them with IP addresses by combining transaction graph analysis and network analysis.
The attacker determines the transaction originator using the first relayer heuristic: the transaction sender's IP is supposed to be the IP that first announces a transaction to the attacker.
This correspondence is not guaranteed because of naturally occurring network delays and broadcast randomization.
The authors suggest ways to increase the precision of the heuristic.

Network-level attacks on privacy-focused cryptocurrencies have also been described~\cite{Tramer2020}.
The time it takes for a remote node to reply to a specific request reveals whether that node is the sender of a given transaction.
The implementation of zero-knowledge proof in Zcash allows for timing side-channel attacks.
Moreover, Grin allows for linking transaction senders and receivers~\cite{Bogatyy2019}.
