\frontmatter % Use roman page numbering style (i, ii, iii, iv...) for the pre-content pages

\pagestyle{plain} % Default to the plain heading style until the thesis style is called for the body content

%----------------------------------------------------------------------------------------
%	TITLE PAGE
%----------------------------------------------------------------------------------------

\begin{titlepage}
\begin{center}

\vspace*{.06\textheight}
{\scshape\LARGE \univname\par}\vspace{1.5cm} % University name
\textsc{\Large Doctoral Thesis}\\[0.5cm] % Thesis type

\HRule \\[0.4cm] % Horizontal line
{\huge \bfseries \ttitle\par}\vspace{0.4cm} % Thesis title
\HRule \\[1.5cm] % Horizontal line
 
\begin{minipage}[t]{0.4\textwidth}
\begin{flushleft} \large
\emph{Author:}\\
\href{https://s-tikhomirov.github.io/about/}{\authorname} % Author name - remove the \href bracket to remove the link
\end{flushleft}
\end{minipage}
\begin{minipage}[t]{0.4\textwidth}
\begin{flushright} \large
\emph{Supervisor:} \\
\href{https://www.cryptolux.org/index.php/Alex\_Biryukov}{\supname} % Supervisor name - remove the \href bracket to remove the link  
\end{flushright}
\end{minipage}\\[3cm]
 
\vfill

\large \textit{A thesis submitted in fulfillment of the requirements\\ for the degree of \degreename}\\[0.3cm] % University requirement text
\textit{in the}\\[0.4cm]
\groupname\\\deptname\\[2cm] % Research group name and department name
 
\vfill

{\large \today}\\[4cm] % Date
%\includegraphics{Logo} % University/department logo - uncomment to place it
 
\vfill
\end{center}
\end{titlepage}

%----------------------------------------------------------------------------------------
%	DECLARATION PAGE
%----------------------------------------------------------------------------------------
\iffalse % not seen in other theses in our group
\begin{declaration}
\addchaptertocentry{\authorshipname} % Add the declaration to the table of contents
\noindent I, \authorname, declare that this thesis titled, \enquote{\ttitle} and the work presented in it are my own. I confirm that:

\begin{itemize} 
\item This work was done wholly or mainly while in candidature for a research degree at this University.
\item Where any part of this thesis has previously been submitted for a degree or any other qualification at this University or any other institution, this has been clearly stated.
\item Where I have consulted the published work of others, this is always clearly attributed.
\item Where I have quoted from the work of others, the source is always given. With the exception of such quotations, this thesis is entirely my own work.
\item I have acknowledged all main sources of help.
\item Where the thesis is based on work done by myself jointly with others, I have made clear exactly what was done by others and what I have contributed myself.\\
\end{itemize}
 
\noindent Signed:\\
\rule[0.5em]{25em}{0.5pt} % This prints a line for the signature
 
\noindent Date:\\
\rule[0.5em]{25em}{0.5pt} % This prints a line to write the date
\end{declaration}
\fi

\cleardoublepage

%----------------------------------------------------------------------------------------
%	QUOTATION PAGE
%----------------------------------------------------------------------------------------

\vspace*{0.2\textheight}

\noindent\enquote{\itshape If you don’t believe it or don’t get it, I don’t have the time to try to convince you, sorry.}\bigbreak

\hfill Satoshi Nakamoto
% https://bitcointalk.org/index.php?topic=532.msg6306#msg6306

%----------------------------------------------------------------------------------------
%	ABSTRACT PAGE
%----------------------------------------------------------------------------------------

\begin{abstract}
\addchaptertocentry{\abstractname} % Add the abstract to the table of contents
This thesis explores various questions related to security and privacy of blockchain systems.

Bitcoin~\cite{nakamoto2008bitcoin} is a breakthrough system, allowing for the first time to transfer digital units of value without any trusted party.
Alternative cryptocurrencies inspired by Bitcoin aim at better addressing the issues of privacy, scalability, and feature set.
In particular, Ethereum introduces smart contracts in a Turing complete language and a stateful virtual machine, greatly expanding the design space of blockchain application, but expanding the attack surface as well.

% a part is a sequence of chapters
The first part of the thesis explores Ethereum.
Ethereum's expressive language, Solidity, provides lots of opportunities for developers to write insecure code.
We propose a simpler language based on Solidity, called Findel, to encode financial agreements in a more declarative fashion.
We study the vulnerabilities in real-world Ethereum contracts and present a tool that automates bug detection in Solidity using static analysis.
Finally, we describe a proposal to limit the privacy-damaging effects of existing legal requirements (know-your-customer, or KYC) if a financial service is implemented on top of Ethereum.

In the second part, we discuss the networking layer of Bitcoin and related cryptocurrencies.
We introduce an attack on privacy that allows an adversary to correlated transactions issued by the same user.
We test this technique on Bitcoin, as well as on the major privacy focused cryptocurrency.
We provide a separate overview of privacy-related functionality in mobile wallets.

The third part is dedicated to a promising approach at scaling blockchain, namely, off-chain protocols.
We study the Bitcoin's Lightning Network as the most prominent example of this approach.
We measure the potential effects of certain subset of Lightning nodes on privacy of the users.
Finally, we introduce a probing attack that lets an adversary reveal intermediary balances of payment channels it is not a part of -- the information supposed to be private.
	
\end{abstract}

%----------------------------------------------------------------------------------------
%	ACKNOWLEDGEMENTS
%----------------------------------------------------------------------------------------

\begin{acknowledgements}
\addchaptertocentry{\acknowledgementname} % Add the acknowledgements to the table of contents
\begin{itemize}
	\item advisor
	\item jury
	\item coauthors
	\item colleagues
	\item Uni staff, incl administrative
	\item Luxembourg
	\item Zcash foundation
	\item family
	\item Satoshi
\end{itemize}
\end{acknowledgements}

%----------------------------------------------------------------------------------------
%	LIST OF CONTENTS/FIGURES/TABLES PAGES
%----------------------------------------------------------------------------------------

\tableofcontents % Prints the main table of contents

\listoffigures % Prints the list of figures

\listoftables % Prints the list of tables

%----------------------------------------------------------------------------------------
%	ABBREVIATIONS
%----------------------------------------------------------------------------------------

\begin{abbreviations}{ll} % Include a list of abbreviations (a table of two columns)

\textbf{PoW} & \textbf{P}roof \textbf{o}f \textbf{W}ork\\
\textbf{LN} & \textbf{L}ightning \textbf{N}etwork\\

\end{abbreviations}

%----------------------------------------------------------------------------------------
%	PHYSICAL CONSTANTS/OTHER DEFINITIONS
%----------------------------------------------------------------------------------------

%\begin{constants}{lr@{${}={}$}l} % The list of physical constants is a three column table

% The \SI{}{} command is provided by the siunitx package, see its documentation for instructions on how to use it

%Speed of Light & $c_{0}$ & \SI{2.99792458e8}{\meter\per\second} (exact)\\
%Constant Name & $Symbol$ & $Constant Value$ with units\\

%\end{constants}

%----------------------------------------------------------------------------------------
%	SYMBOLS
%----------------------------------------------------------------------------------------

%\begin{symbols}{lll} % Include a list of Symbols (a three column table)

%$a$ & distance & \si{\meter} \\
%$P$ & power & \si{\watt} (\si{\joule\per\second}) \\
%Symbol & Name & Unit \\

%\addlinespace % Gap to separate the Roman symbols from the Greek

%$\omega$ & angular frequency & \si{\radian} \\

%\end{symbols}

%----------------------------------------------------------------------------------------
%	DEDICATION
%----------------------------------------------------------------------------------------

\dedicatory{For/Dedicated to/To my\ldots} 