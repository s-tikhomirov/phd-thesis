\chapter{Quantitative analysis of Lightning network privacy}

\label{Chapter07LNattacks}

We evaluate the possibility of various attacks on Lightning, considering various subsets of nodes potentially compromised.

\todo[inline]{Update with camera-ready version}
\todo[inline]{Split between Chapter 05, 07,08}

Payment channel networks have been introduced to mitigate the scalability issues inherent to permissionless decentralized cryptocurrencies such as Bitcoin.
Launched in 2018, the Lightning Network (LN) has 
been gaining popularity and 
consists today of more than $5000$ nodes and $30000$ payment channels 
that jointly hold $895$~bitcoins ($7.6M$~USD as of February~2020).
This adoption has motivated research from both academia and industry.

Payment channels suffer from security vulnerabilities, such as the wormhole attack~\cite{Malavolta2019}, anonymity issues~\cite{Malavolta2017}, and scalability limitations related to  the upper bound on the number of concurrent payments per channel~\cite{EmelyanenkoK2017}, which have been pointed out by the scientific community but never quantitatively analyzed. 

In this work, we first analyze the proneness of the LN to the wormhole attack and attacks against anonymity. 
We observe that an adversary needs to control only $2\%$ of LN nodes to learn sensitive payment information (e.g., sender, receiver and payment amount) or to carry out the wormhole attack. 
Second, we study the management of concurrent payments in the LN and quantify its negative effect on scalability. 
We observe that for micropayments, the forwarding capability of up to $50\%$ of channels is restricted to 
a value smaller than the overall channel capacity.
This phenomenon not only hinders scalability but also opens the door for DoS attacks: We estimate that 
a network-wide DoS attack costs within $1.5M$~USD, while isolating the biggest community from the rest of the network costs only $225k$~USD.

Our findings should prompt the LN community to consider the security, privacy and scalability issues of the network studied in this work 
when educating users about path selection algorithms, as well as to adopt multi-hop payment protocols that provide stronger security, privacy and 
scalability guarantees. 

\section{Background}
\label{sec:background}

\begin{figure*}[tb]
\includegraphics[width=\textwidth]{htlc-figure}
	\caption{An HTLC-based payment in the LN. The node $u_1$ pays $u_5$ using $u_2$, $u_3$ and $u_4$ as intermediaries. 
	Here we assume that each node charges a fee of $0.1$ and time is measured in days.\label{fig:htlc}}
\end{figure*}

The Lightning Network (LN) has emerged as the alternative to the scalability issue of Bitcoin  with the highest adoption in practice~\cite{Cuen2019}.
%LN has experienced rapid growth since its launch in early 2018~, 
As of February~2020, LN facilitates the off-chain exchange of nearly $900$~BTC.
The principles of the LN can be used to improve the scalability of other cryptocurrencies. 
For instance, similar networks operate with Litecoin~\cite{1MLLitecoin} and Ethereum~\cite{RaidenWebsite}. 
In this section, we introduce the basic notions of the LN and refer the reader 
to~\cite{Gudgeon2019} for further reading.
% TowardsBitcoinPaymentNetworks, BitcoinMagazineUnderstandingLightning, 

\subsubsection*{LN nodes} A node in the LN is governed by a pair of signing and verification keys from 
the ECDSA signature scheme, 
and identified by the hashed value of the verification key.  
Additionally, the owner can assign a handcrafted identifier (alias) to their node.
Operations from a node are authorized with a digital signature 
created with the corresponding signing key.
Thus, whoever holds the signing key is the owner of a node.
One user can potentially own several nodes.

%\vspace{-0.2cm}
\subsubsection*{LN channels} A LN channel (i.e.,~an edge) is jointly controlled by the two counterparties and its capacity is determined by the amount of coins 
deposited when created.
While the total capacity of the channel stays constant during its lifetime,
the balance of each counterparty varies according to two operations:
(i) single channel updates, where the two users agree on an updated balance; and 
(ii) multi-hop transactions, where the balance 
of several channels forming a path are simultaneously updated. 

%\vspace{-0.2cm}
\subsubsection*{LN transactions} A multi-hop transaction (or simply a transaction hereby) leverages a 
path of channels between a sender and a receiver (who might not share a channel between them).
A transaction must ensure the atomicity of the transfer: 
either all balances along the path are updated or none of them are.
For that, the LN relies on Hash Time-Lock Contracts (HTLCs), 
 excerpts from the  Bitcoin's scripting language that 
permit a node ($u_1$) to lock $x$~coins in a channel between two nodes ($u_1$ and $u_2$) 
and release them according to the encoded conditions.
The terms for the HTLC($u_1, u_2, y, x, t$) are defined with a hash value $y := H(r)$, 
where $r$ is chosen uniformly at random, 
an amount $x$ of coins, and a timeout $t$, as follows: 
(i) If $u_2$ reveals a value $r$ such that $H(r) = y$ before $t$ expires, $u_1$ pays $x$  to $u_2$; 
(ii) if $t$ expires, $u_1$ receives $x$  back.

LN relies on HTLCs to enable multi-hop transactions.  
All HTLCs along the path use the same hash value $y=H(r)$ aiming to achieve atomicity expecting that 
none of the intermediate balances can be updated before the receiver reveals $r$, and all of them can be updated after that.
An illustrative example of an HTLC-based transaction is depicted in~\cref{fig:htlc}.
Here, the user $u_1$ transfers $1$ bitcoin to $u_5$ using $u_2$, $u_3$ and $u_4$ as intermediaries. 
For that, $u_5$ locally chooses a value $r$ 
uniformly at random, computes the cryptographic challenge for the HTLC as $y := H(r)$, 
and sends $y$ to the sender (step 1).
The message encoding $y$ is called an \textit{invoice}.
Then, the payment starts with a commit phase (steps 2-5) where every pair of nodes, 
starting from the sender, establishes an HTLC using $y$.
After the commit phase is finished, the transaction enters the release phase.
Here, the receiver reveals $r$ to $u_4$ to fulfill the contract (step 6), 
triggering thereby the release phase where every pair of nodes fulfills their 
contract from the receiver to the sender (steps 6-9).

It is important to note two aspects here.
First, every intermediary user charges a fee for the forwarding service provided. 
For instance, $u_2$ receives $1.3$~coins but only forwards $1.2$~coins, getting a fee of $0.1$~coins. 
Second, the time parameter of the contracts throughout the path is decreasing to ensure that no user loses coins. 
For instance, the HTLC between $u_1$ and $u_2$ sets a timeout of four days 
whereas the timeout in the HTLC between $u_2$ and $u_3$ is only three days.
This facilitates that 
$u_2$ has enough time to settle the contract with $u_1$ after receiving $r$ from $u_3$.
%\todo[inline]{Mention that this introduces a DoS risk?}
% There is an inherent trade-off: channels with short timelocks open the risk of not being able to dispute a fraudulent transaction in case of blockchain congestion, and long timelocks open up a DoS vector, where an attacker can route many unsettled payments through a channel and effectively block it until the timelock expires.

%\todo[inline]{Preventing and onion routing paragraphs could be deleted}
%\paragraph{Preventing cheating}
%A critical problem for off-chain protocols is invalidating an old state.
%In a channel between Alice and Bob, after Alice has sent some \coins to Bob, she may try to close the channel on-chain, effectively double-spending.
%To prevent this, in every LN transactions the parties exchange secrets which would allow them to take all money from the channel should the other party publish an old state.
%
%\paragraph{Onion routing}
%A channel can keep track of multiple unsettled HTLCs.
%In a chain of channels "Alice - Bob - Charlie - Dave", Alice can be paying Dave, and at the same time Bob can be paying Dave.
%For each payment being forwarded, a node only knows the immediate previous and next hops, but neither the initial sender nor the final recipient.

\subsubsection*{LN implementations} 
The development of the LN, which was originally introduced in~\cite{Poon2016}, is guided by a set of request for comments (RFC) documents called "Basics of Lightning Technology" or BOLTs~\cite{BOLT}, 
which are then followed by several implementation teams.
The three most advanced implementations available today are LND~\cite{LND}, 
c-lightning~\cite{clightning}, and Eclair~\cite{Eclair}.
Additionally, there exist implementations at earlier stages of development:
Electrum~\cite{ElectrumWebsite, ElectrumLightningAnnounce}, lit~\cite{lit}, lpd~\cite{lpd}, ptarmigan~\cite{ptarmigan}, and rust-lightning~\cite{rustlightning}.
Our analysis is concerned with the definition of the LN as described in the BOLTs and thus the results 
apply equally to every implementation. 


\section{Datasets}
\label{sec:datasets}



We obtained a snapshot of LN on 2020-02-25 from \url{https://ln.bigsun.xyz} and parsed with the (anonymized) scripts available at~\cite{Tikhomirov2019}.
%\footnote{Our experiments are based on data from \url{https://ln.bigsun.xyz/}.
%	The (anonymized) scripts are available at~\cite{LightningPrivacy}.}
This snapshot consists of $5929$~nodes and $35233$~channels.
We model this data as an undirected multi-graph (i.e., may contain multiple edges between each pair of nodes), 
representing the fact that several channels can be shared by two LN nodes.
We only considered the largest connected component, 
which contains $5862$~nodes $(98.87\%)$ and $35196$~channels $(99.89\%)$.
We observe that this subgraph contains a representative sample of the LN.
We refer to this dataset as \emph{LN20}.

Based on \emph{LN20}, 
LN nodes have an average degree of $12.01$ and a median degree of $3$ (see~\cref{fig:node-degree-histogram,fig:channel-capacity-histogram}).
The majority of nodes have very few channels, 
whereas there is a small number of nodes with many channels. 
In particular, there are more than 1744 nodes with degree 1, and the most connected node has 1198~channels.
The capacity is also unequally distributed.
 %\footnote{Public key \texttt{03ffdd7fd4656a55a63a4b6e154325859681718ba2fac40e51cb61752506bb8c7b}.}
These observations motivate the methodology in our experiments in~\cref{sec:sec-priv-attacks}. 

We also derive a series of historic snapshots which represent the state of LN on the first day of each month from April~2018 to February~2020.
We refer to this dataset as \textit{LNHist} 
and use it in our study of concurrent payments on LN performance over time (\cref{sec:attack}).
%See Appendix~\ref{sec:historic} for our study of the temporal evolution of the LN based on LNHist.


\begin{figure}[tb]
	\centering
	\includegraphics[width=\columnwidth]{node-degree-histogram}
	\caption{Node degree distribution.\label{fig:node-degree-histogram}}
\end{figure}

\begin{figure}[tb]
	\centering
	\includegraphics[width=\columnwidth]{channel-capacity-histogram}
	\caption{Channel capacity distribution.\label{fig:channel-capacity-histogram}}
\end{figure}

\subsubsection*{Ethical considerations} 
Our analysis is based solely on publicly available data. 
We do not interfere with the LN activity, nor deanonymize any of its nodes. 


\input{Chapters/Chapter07/resilience}

\input{Chapters/Chapter07/attacks}

\section{Related work}
\label{sec:related-work}

%We can roughly divide the related literature in two groups: academic analyses of the concept of payment-channel networks (PCN)
%and experimental analyses of the Lightning Network data.

Multiple research works have shed light on various aspects of payment-channel networks, such as security~\cite{Malavolta2019, Kiayias2019}, privacy~\cite{Malavolta2017, HerreraJoancomarti2019}, concurrency~\cite{Malavolta2017}, routing~\cite{Malavolta2017a, Roos2018, Sivaraman2018, Prihodko2016}, liquidity~\cite{Dandekar2011,MorenoSanchez2018}, efficiency~\cite{Decker2018}, and incentive compatibility~\cite{Engelmann2017}.
These works mainly share the lack of a quantitative analysis of the impact of their findings in the current LN.
%In tour work , we have empirically analyzed the potential severity of the wormhole attack, 
%as well as attacks on value privacy and relationship anonymity, in the current Lightning Network. 

A group of papers more closely related to ours conveys experimental analyses of various aspects of the LN.
Herrera-Joancomart\'{i} et al.~\cite{HerreraJoancomarti2019} describe an adversarial 
strategy to determine the current balance of a channel in the network.
Tang et al.~\cite{Tang2019} study 
the tradeoffs between balance privacy and routing effectiveness. 
Martinazzi~\cite{Martinazzi2019} and Seres et al.~\cite{Seres2019} study the evolution of topological aspects of the LN graph.
Conoscenti et al.~\cite{Conoscenti2019} study the dependency of the LN on payment hubs 
and the rebalancing mechanisms that ameliorate the effect of depleted channels.
Tochner et al.~\cite{Tochner2019} analyze a DoS attack vector based on route hijacking. 
P{\'{e}}rez{-}Sol{\`{a}} et al.~\cite{PerezSola2019} introduce the LockDown attack where the adversary 
prevents a LN node from transacting by depleting the capacity in all its channels.
In comparison, our HTLC depletion attack achieves the same result (a victim node can not forward payments), but exploits the HTLC limit at each channel rather than its capacity.
%\todo[inline]{Pedro:I would not put this text in a footnote. We need to say why related work is different from this}
%Our experiments show that the isolation of a node (or part of a network) 
%exploiting the HTLC limit is cheaper for the adversary than the attack proposed by P{\'{e}}rez{-}Sol{\`{a}} et al. 
Finally, concurrently to our research, Mizrahi and Zohar~\cite{Mizrahi2020} study the HTLC limit and its effects.
Their work, however, does not account for the way LN handles payments below the dust limit.
%\todo[inline]{Pedro: Same here. What's the difference with this work? Aren't they missing the point that 1satoshi payments are not carried out with HTLC?}


%their estimated cost of attacking one node is $15$~EUR ($16.5$~USD), whereas the cost of attacking the whole LN in our case is estimated at $3000$~USD, i.e.,~around $0.6$~USD per node.


\section{Conclusions}
\label{sec:conclusions}

The Lightning Network (LN) has emerged as the most widely deployed solution for the scalability issue affecting current blockchains such as Bitcoin. 
%Its substantial growth in the number of nodes, payment channels and their capacity has attracted attention from academia and industry.
Despite its conceptual appeal and growing adoption,  several works~\cite{Malavolta2017, Malavolta2019} have identified 
 security, anonymity and scalability limitations. A quantitative 
analysis of their impact, however, is missing and this paper aims at filling this gap.

We quantitatively study for the first time the proneness of the current Lightning Network to the 
wormhole attack as well as attacks against value privacy and relationship anonymity. 
We observe that a moderately resourceful adversary controlling only $2\%$ of the total node count can carry out these attacks with high success probability.

We also quantitatively analyze the negative effect on scalability produced by the limit on concurrent payments in the LN. 
We calculate that the limited concurrency in the LN implies that an adversary can block the complete LN investing around $1.5M$~USD ($18.5\%$~of the network capacity), and this cost can be substantially reduced by targeting highly valuable channels (e.g., high-capacity channels or those connecting the biggest communities in the network).
%\todo[inline]{Pedro: This is not an encouraging conclusion :)}

%First, we quantitatively analyze the effect of the wormhole attack as well as attacks on value privacy and relationship anonymity. 
%Second, we quantitatively analyze the restriction of the LN scalability gain due to the limited concurrency supported by the LN protocol. 
%In particular, we observe that more than up to $50\%$ of LN channels can support fewer concurrent micropayments than in principle allowed by their capacity. 
%For instance, the adversary investing $225$k~USD can prevent nodes in the largest community from transacting with the rest of the network. 

% the probability of three privacy attacks depending on payment amounts and the number of compromised nodes.
%Our results show that a payment's probability of being forwarded through a malicious path does not depend on its size, but large payments are less likely to succeed at all.
%Among the three attacks we considered, value privacy is the most sensitive to the number of compromised nodes.
%With just a few highly-connected hubs compromised, LN users risk having their payment amounts leaked to the adversary.
%Relationship anonymity attack and, finally, the wormhole attack are less likely with the same number of malicious nodes.

%Then, we have described an understudied drawback of LN -- the limit on the number of concurrent in-flight payments (HTLCs) in a channel.
%This limitation is explained by the interaction between Lightning and Bitcoin protocols.
%We have shown, first, that this is the key factor limiting LN performance for small payments.
%Our results indicate that some of the widely discussed LN use cases (such as micropayments) might not be viable even with sufficient channel capacity.
%We describe a DoS attack vector based on the HTLC limit and show that it is more efficient for the attacker to block channels by depleting their HTLC limit rather than capacity, as has been already described in the literature.
%In this state of affairs, we prompt the LN community to implement countermeasures to alleviate these security, anonymity and scalability issues. 
%For instance,  we argue that implementing anonymous multi-hop locks (AMHL) instead of the currently used HTLC construction would 
%mitigate the security, privacy  and scalability issues.  %We hope that this work helps Lightning and Bitcoin communities identify and address the most pressing bottlenecks of these technologies and help them achieve their full potential.

