
\section{Security challenges in Ethereum} \label{SecurityChallenges}

Ethereum allows to encode and trustlessly enforce complex financial agreements.
This opens up new business models and constitutes a dramatic change in the digital economy.
Smart contracts operate in a highly adversarial environment and are hard to debug once deployed.
Multiple high-profile contracts have been compromised~\cite{Sirer2016, Palladino2017}.
This highlights the importance of software tools to ensure correctness and security of smart contracts.

Security is a primary concern in Ethereum programming for multiple reasons:
\begin{itemize}
	\item \textbf{Unfamiliar execution environment}.
	Ethereum differs from centrally managed execution environments, be that mobile, desktop, or cloud.
	Developers are not used to their code being executed by a global network of anonymous, mutually distrusting, profit-driven nodes.
	\item \textbf{New software stack}.
	The Ethereum stack (the Solidity compiler, the EVM, the consensus layer, etc) is under development, with security vulnerabilities still being discovered~\cite{chriseth2017}.
	\item \textbf{Very limited ability to patch contracts}.
	A deployed contract can not be patched\footnote{Though workarounds exist, such as proxy contracts redirecting calls to an adaptable address of the latest version of the main contract.}.
	This makes a popular ``move fast and break things`` motto inapplicable: a contract must be correct before deployment.
	\item \textbf{Anonymous financially motivated attackers}.
	Compared to many cybercrimes, exploiting smart contracts offers higher gains (the prices of cryptocurrencies have been increasing rapidly), easier cashing out (ether and tokens are instantly tradable), and lower risk of punishment due to anonymity.
	\item \textbf{Rapid pace of development}.
	Blockchain companies strive to release their products fast, often at the expense of security.
	\item \textbf{Suboptimal high-level language}.
	Some argue that Solidity itself inclines programmers towards unsafe development practices~\cite{ydtm2016}.
\end{itemize}
