\section{Conclusions}
\label{sec:conclusions}

The Lightning Network (LN) has emerged as the most widely deployed solution for the scalability issue affecting current blockchains such as Bitcoin. 
%Its substantial growth in the number of nodes, payment channels and their capacity has attracted attention from academia and industry.
Despite its conceptual appeal and growing adoption,  several works~\cite{Malavolta2017, Malavolta2019} have identified 
 security, anonymity and scalability limitations. A quantitative 
analysis of their impact, however, is missing and this paper aims at filling this gap.

We quantitatively study for the first time the proneness of the current Lightning Network to the 
wormhole attack as well as attacks against value privacy and relationship anonymity. 
We observe that a moderately resourceful adversary controlling only $2\%$ of the total node count can carry out these attacks with high success probability.

We also quantitatively analyze the negative effect on scalability produced by the limit on concurrent payments in the LN. 
We calculate that the limited concurrency in the LN implies that an adversary can block the complete LN investing around $1.5M$~USD ($18.5\%$~of the network capacity), and this cost can be substantially reduced by targeting highly valuable channels (e.g., high-capacity channels or those connecting the biggest communities in the network).
%\todo[inline]{Pedro: This is not an encouraging conclusion :)}

%First, we quantitatively analyze the effect of the wormhole attack as well as attacks on value privacy and relationship anonymity. 
%Second, we quantitatively analyze the restriction of the LN scalability gain due to the limited concurrency supported by the LN protocol. 
%In particular, we observe that more than up to $50\%$ of LN channels can support fewer concurrent micropayments than in principle allowed by their capacity. 
%For instance, the adversary investing $225$k~USD can prevent nodes in the largest community from transacting with the rest of the network. 

% the probability of three privacy attacks depending on payment amounts and the number of compromised nodes.
%Our results show that a payment's probability of being forwarded through a malicious path does not depend on its size, but large payments are less likely to succeed at all.
%Among the three attacks we considered, value privacy is the most sensitive to the number of compromised nodes.
%With just a few highly-connected hubs compromised, LN users risk having their payment amounts leaked to the adversary.
%Relationship anonymity attack and, finally, the wormhole attack are less likely with the same number of malicious nodes.

%Then, we have described an understudied drawback of LN -- the limit on the number of concurrent in-flight payments (HTLCs) in a channel.
%This limitation is explained by the interaction between Lightning and Bitcoin protocols.
%We have shown, first, that this is the key factor limiting LN performance for small payments.
%Our results indicate that some of the widely discussed LN use cases (such as micropayments) might not be viable even with sufficient channel capacity.
%We describe a DoS attack vector based on the HTLC limit and show that it is more efficient for the attacker to block channels by depleting their HTLC limit rather than capacity, as has been already described in the literature.
%In this state of affairs, we prompt the LN community to implement countermeasures to alleviate these security, anonymity and scalability issues. 
%For instance,  we argue that implementing anonymous multi-hop locks (AMHL) instead of the currently used HTLC construction would 
%mitigate the security, privacy  and scalability issues.  %We hope that this work helps Lightning and Bitcoin communities identify and address the most pressing bottlenecks of these technologies and help them achieve their full potential.
